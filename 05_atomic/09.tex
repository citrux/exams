\question{Атом водорода по квантовой механике. Уровни и спектры атомов 
щелочных металлов.}

Рассмотрим одноэлектронную задачу с потенциальной энергией вида
\( \ds U(r) = -k\frac{Ze^2}{r} \) (гиперболическая потенциальная яма). Будем
решать её в сферических координатах.

Уравнение Шрёдингера:
\[
    \frac{1}{r^2}\pder{}{r}\left(r^2\pder{\psi}{r}\right) + \frac{1}{r^2
    \sin\theta}\pder{}{\theta}\left(\sin\theta\pder{\psi}{\theta}\right) +
    \frac{1}{r^2\sin^2\theta}\ppder{\psi}{\phi} + \frac{2m}{\hbar^2}\left(E +
    k\frac{Ze^2}{r}\right)\psi = 0.
\]

Из этого уравнение можно выделить три следствия:
\begin{enumerate}
    \item это уравнение имеет решение при любых значениях \( E > 0 \). Это
    соответствует случаю, когда несвязанный с атомом электрон пролетает мимо
    ядра;
    
    \item это уравнение имеет решение при некоторых дискретных значениях
    \( E < 0 \):
    \[
        E_n = -\frac{m_ee^4}{2\hbar^2}\frac{Z^2}{n^2},
    \]
    причем эти значения точно совпадают со значениями, найденными по теории
    Бора;
    
    \item волновые функции, являющиеся решениями этого уравнения, зависят от
    трёх целочисленных параметров: \( \psi = \psi_{nlm}(r, \theta, \phi) \);
    число \( n \) -- главное квантовое число(входит в формулу для энергии
    \( E_n \) и определяет энергию электрона в атоме), \( l \) -- орбитальное
    квантовое число (определяет модуль момента импульса электрона в атоме),
    \( m \) -- магнитное квантовое число (определяет проекцию момента импульса
    электрона в атоме на выбранное направление):
    \[
        n = 1,\,2,\,3,\,\ldots; \quad l = 0,\,1,\,\ldots,\,n - 1; \quad m = 0,\,
        \pm 1,\,\pm 2,\,\ldots,\,\pm l.
    \]
\end{enumerate}
    
Энергия электрона зависит только от главного квантового числа,
следовательно, каждому собственному значению \( E_n \) соответствует
несколько собственных волновых функций \( \psi_{nlm} \), отличающихся друг
от друга числами \( l \) и \( m \). Это означает, что атом водорода может
иметь одно и то же значение энергии, находясь в нескольких различных
состояниях. Состояния с одинаковой энергией называются вырожденными, а число
состояний с определённым значением энергии называется кратностью вырождения:
\( \ds g = \sum_{l = 0}^{n-1} (2l + 1) = n^2 \).
    
Состояния, в которых \( l = 0 \) называют \( s \)-состояниями, в которых
\( l = 1 \) -- \( p \)-состояниями, \( l = 2 \) -- \( d \), \( l = 3 \) --
\( f \), \( l = 4 \) -- \( g \) и далее по алфавиту.
    
Перед этими символами указывается главное квантовое число, например, \( 2s \),
\( 3p \).

В квантовой механике доказывается, что испускание и поглощение фотона происходит
при переходе электрона с одного уровня на другой при условии, что \( \Delta l =
\pm 1 \). Это условие является следствием закона сохранения момента импульса
фотона.

Переходы на уровень \( 1s \) образуют серию Лаймана:
\[
    2p \to 1s,\ 3p \to 1s,\ \ldots,\ np \to 1s \quad (n = 2,\,3\,\ldots).
\]
Переходы на уровень \( 2s \) и \( 2p \) образуют серию Бальмера:
\[
    3p \to 2s,\ 3s \to 2p,\ 3d \to 2p,\ \ldots,\ np \to 2s,\ ns \to 2p,\ nd \to
    2p \quad (n = 3,\,4\,\ldots).
\]

Вернемся к исходной задаче. Её решением являются волновые функции, которые можно
представить в виде:
\[
    \psi_{nlm} = R_{nl}(r) Y_{lm}(\theta, \phi).
\]
Угловая функция \( Y_{lm}(\theta, \phi) \) является решением уравнения на
нахождение собственных значений квадрата момента импульса
\( \hat{L}^2\psi = L^2\psi \):
\[
    Y_{lm}(\theta, \phi) = \varTheta_{l|m|}(\theta)\cdot\e^{\i m\phi}.
\]
Для состояния \( 1s \): \( R(\rho) = \e^{-\rho} \), \( l = 0 \), \( m = 0 \),
\( \varTheta = 1 \);

\( 2s \): \( R(\rho) = (2 - \rho)\e^{-\rho/2} \), \( l = 1 \), \( m = 0 \),
\( \varTheta = \cos\theta \);

\( 2p \): \( R(\rho) = \rho\e^{-\rho/2} \), \( l = 1 \), \( |m| = 1 \),
\( \varTheta = \sin\theta \), где \( \rho = r/r_1 \).

Нормируем: \( \ds \int\psi^*_{nlm}\psi_{nlm}\,dV = 1 \), \( dV = r^2\sin\theta\,
dr\d\theta\d\phi \). Подставляя функцию \( \psi_{nlm} \) имеем:
\[
    \int R_{nl}^2 r^2\,dr \int\limits_{(4\pi)} Y^*_{lm} Y_{lm}\d\varOmega = 1.
\]

Из-за того, что \( Y_{lm}(\theta, \phi) \) является решением уравнения на
нахождение собственных значений квадрата момента импульса, то она нормирована.
Имеем:
\[
    \int\limits_0^\infty R_{nl}^2 r^2\,dr = 1; \quad \int\limits_{(4\pi)}
    Y^*_{lm}Y_{lm}\d\varOmega = 1.
\]

Плотность вероятности нахождения электрона на расстоянии \( r \) от ядра:
\[
    \rho(r) = \der{W}{r} = R_{nl}^2 r^2.
\]
Радиусы орбит по Бору совпадают с наиболее вероятными расстояниями электрона от
ядра. Для наглядной иллюстрации о положении электрона в атоме вводят понятие
электронного облака: плотность распределения этого электронного облака
пропорциональна плотности вероятности нахождения электрона в данной точке
пространства. Главное квантовое число \( n \) определяет размеры облака: чем
оно больше, тем облако больше.

\newpage
