\chapter{Отражение и прохождение частиц через потенциальный барьер.}
В квантовой механике имеется вероятность:
\begin{itemize}
    \item отразиться от потенциального барьера при \( E > U_0 \);
    \item имеется вероятность пройти сквозь потенциальный барьер при
    \( E < U_0 \).
\end{itemize}

Рассмотрим случай \( E < U \). Решим уравнение Шрёдингера для областей 1,2 и 3.
Для областей 1 и 3 уравнение Шрёдингера имеет вид
\[
    \dder{\psi}{x} + \frac{2mE}{\hbar^2}\psi = 0,\ \frac{2mE}{\hbar^2} = k^2.
\] 
Его общее решение для области 1 имеет вид
\[
    \psi_1 = A_1 e^{ikx} + B_1 e^{-ikx},
\]
а в области 3
\[
    \psi_3 = A_3 e^{ikx} + B_3 e^{-ikx}.
\]
Решение в виде \( \psi_1 \) можно трактовать как суперпозицию двух бегущих волн:
первое слагаемое представляет собой волну, бегущую в положительном направлении
оси \( Ox \), а второе -- волну бегущую в обратном направлении. Поэтому
\( B_3 = 0 \).

Теперь рассмотрим уравнение Шрёдингера в области 2:
\[
    \dder{\psi}{t} + \frac{2m}{\hbar^2}(E - U_0)\psi = 0.
\]
Его решение имеет вид
\[
    \psi_1 = A_2 e^{\beta x} + B_2 e^{-\beta x}.
\]
Из условия гладкости \( \psi \)-функции получим выражения для связи коэффициентов:
\begin{align*}
    A_1 + B_1 = A_2 + B_2, & A_2 e^{\beta l} + B_2 e^{-\beta l} = A_3 e^{ikl};\\
    ik A_1 - ik B_1 = \beta A_2 - \beta B_2, & \beta A_2 e^{\beta l} - \beta
    B_2 e^{-\beta l} = ik A_3 e^{ikl}. 
\end{align*}

Введём понятие коэффициентов отражения и прохождения барьера:
\[
    R = \frac{|B_1|^2}{|A_1|^2},\ 
    D = \frac{|A_3|^2}{|A_1|^2},\ 
    R + D = 1.
\]

Введём обозначения:
\[
    b_1 = \frac{B_1}{A_1},\ a_2 = \frac{A_2}{A_1},\ b_2 = \frac{B_2}{A_1},\ 
    a_3 = \frac{A_3}{A_1},\ n = \frac{\beta}{k}.
\]

Тогда получим систему уравнений:
\[
    \left\{
    \begin{array}{l}
        1 + b_1 = a_2 + b_2, \\
        i - ib_1 = n a_2 - n b_2, \\
        a_2 e^{\beta l} + b_2 e^{-\beta l} = a_3 e^{ikl}, \\
        na_2 e^{\beta l} - nb_2e^{-\beta l} = ia_3 e^{ikl}.
    \end{array}
    \right.
\]

Решая её, получим:
\[
    \left\{
    \begin{array}{l}
        2i = (n+i)a_2 - (n-i)b_2, \\
        (n-i)a_2 e^{\beta l} - (n+i)b_2 e^{-\beta l} = 0.
    \end{array}
    \right.
\]
Отсюда,
\[
    a_2 = \frac{2i(n+i)e^{-\beta l}}{(n+i)^2e^{-\beta l} - (n-i)^2e^{\beta l}},\  
    b_2 = \frac{2i(n-i)e^{\beta l}}{(n+i)^2e^{-\beta l} - (n-i)^2e^{\beta l}},\ 
    a_3 = \frac{4nie^{-ikl}}{(n+i)^2 e^{-\beta l} - (n-i)^2 e^{\beta l}}.
\]
С учётом \( \beta \gg 1 \), упрощая выражение, получим:
\[
    a_3 = \frac{4nie^{-(ik+\beta)l}}{-(n-i)^2}, \ 
    D = |a_3|^2 = \frac{16n^2}{(n^2 + 1)^2} e^{-2\beta l} \approx e^{-2\beta l}. 
\]

Однако, барьеры бывают не только прямоугольными [рисунок 4].

Явление прохождения частицы через потенциальный барьер называется туннельным
эффектом.Он является следствием соотношения неопределённостей Гейзенберга.
Туннельный эффект может объяснить многие явления, наблюдаемые на практике, такие
как холодная эммиссия электронов из металлов и \( \alpha \)-распад.

Холодная эмиссия: при приложении к металлу электрического поля существенно
возрастает количество электронов. вылетающих из металла. Это легко объясняется 
туннельным эффектом. [рисунок 5]

\( \alpha \)-распад: потенциальный барьер обусловлен кулоновским взаимодействием
ядра, образовавшегося при \( \alpha \)-распаде с \( \alpha \)-частицей.
[рисунок 6]
\newpage
