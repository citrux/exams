\chapter{Операторы физических величин. Квантование момента импульса.}

Под оператором подразумевается правило, посредством которого одной функции
сопоставляеется другая функция:
\[
    f = \hat{Q}\phi.
\]
Например,таковым является лапласиан:
\[
    \Delta = \ppder{}{x} + \ppder{}{y} + \ppder{}{z}.
\]
Уравнение Шредингера может быть записано в виде
\[
    \hat{H}\Psi = E\Psi.
\]
Вообще, уравнение вида
\begin{equation}
    \hat{Q}\Psi = q\Psi.
    \label{eq:1}    
\end{equation}
это уравнение на нахождение собственных значений и собственных функций оператора
\( \hat{Q} \).

Один из постулатов квантовой теории утверждает, что при измерении физической
величины q, представляемой оператором  \( \hat{Q} \) могут получаться только
результаты, совпадающие с собственными значениями этого оператора, то есть
являющиеся решением уравнения \ref{eq:1}.
\[
    \hat{x} = x,\ \hat{p}_x = ih\pder{}{x},\ \hat{\vec{p}} = ih\nabla. 
\]
Ещё один из постулатов квантовой теории утверждает, что среднее значение любой
физической величины определяется соотношением
\[
    \midnum{Q} = \int \Psi^*\hat{Q}\Psi\,dV.
\]

Формулы классической физики для связи между величинами в квантовой теории стоит
рассматривать как формулы, связывающие операторы этих величин.

Рассмотри оператор момента импульса. В классической физике
\( \vec{L} = \vec{r}\times\vec{p} \). Для оператора момента импульса в квантовой
теории получим выражение
\[
    \left\{\right.
\]

В квантовой теории доказывается, что одновременно могут быть измеримы модуль
момента импульса и одна из его проекций.
\[
    \hat{L^2}\Psi = L^2\Psi
\]
Решение имеет вид
\[
    L^2 = l(l+1)\hbar
\]
То есть модуль момента импульса имеет дискретный спектр. \( l \) -- орбитальное
квантовое число.

\[
    -ih\pder{\psi}{\phi} = L_z\psi.
\]
Решение уравнения имеет вид
\[
    \psi = exp(\frac{iL_z\phi}{\hbar}).
\]
Так как функция периодична, то
\[
    L_z = m\hbar.
\]
Ось Oz  называется выделенным направлением. Таким образом, мы получили
собственные значения оператора момента импульса на выделенное направление.
Так как \( |L_z| < |L| \), то \( |m| < \sqrt{l(l+1)} \).

Заметим, что момент импульса не может совпадать с выделенным направлением в
пространстве.

\newpage
