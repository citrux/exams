\question{Принцип тождественности одинаковых частиц. Фермионы и бозоны. 
Принцип Паули. Строение электронных оболочек. Объяснение периодической 
системы элементов Менделеева. Правила Хунда.}

С точки зрения квантовой теории частицы одной природы теряют индивидуальность и
оказываются неразличимы. Это носит название принципа неразличимости
(тождественности) одинаковых частиц.

Рассмотрим две частицы: \( q_1(\vec{r}_1,\,m_s) \) и \( q_2(\vec{r}_2,\,
m_s) \). Согласно принципу тождественности одинаковых частиц
\( |\psi(q_1,\,q_2)|^2 = |\psi(q_2,\,q_1)|^2 \). Это соотношение может быть
представлено двумя случаями:
\begin{enumerate}
    \item \( \psi(q_1,\,q_2) = \psi(q_2,\,q_1) \) -- волновая функция
    симметрична относительно перестановки координат. Такие волновые функции
    характерны для частиц, обладающих целым спином \( (s = 0,\,1,\,2,\,\ldots)
    \) -- бозонов.
    
    \item \( \psi(q_1,\,q_2) = -\psi(q_2,\,q_1) \) -- волновая функция
    антисимметрична относительно перестановки координат. Такие волновые функции
    характерны для частиц с полуцелым спином -- фермионов.
\end{enumerate}

Бозоны могут находится в пределах данной системы в одинаковом состоянии в
неограниченном количестве, а фермионы могут находится в одинаковом квантовом
состоянии только по одиночке. Отсюда вытекает принцип Паули:

\begin{quote}
\emph{в атоме не может быть двух электронов, обладающих одинаковой
совокупностью квантовых чисел.}
\end{quote}

Состояние электрона в атоме характеризуется четырьмя квантовыми числами: главным
\( n = 1,\,2,\,\ldots \), орбитальным \( l = 0,\,1,\,\ldots,\,n - 1 \),
магнитным \( m_l = -l,\,-l + 1,\,\ldots,\, l \) и спиновым \( m_s = -1/2,\,
1/2 \). Таким образом, максимальное количество электронов на одном
энергетическом уровне равно \( 2n^2 \).

Электроны, для которых главное квантовое число \( n \) одинаково, образуют
оболочку. Каждая оболочка состоит из подоболочек -- совокупностей электронов,
для которых одинаковы квантовые числа \( n \) и \( l \):

\begin{table}[h!]
    \center
    \begin{tabular}{|C{.15}|C{.12}|*{2}{C{.05}|}*{3}{C{.03}|}*{4}{C{.02}|}
    C{.03}|}\hline
        \( n \) & 1 & \multicolumn{2}{c|}{2} & \multicolumn{3}{c|}{3}
        & \multicolumn{4}{c|}{4} & \ldots \\ \hline
        % --------------------------------------------------
        Оболочка & K & \multicolumn{2}{c|}{L} & \multicolumn{3}{c|}{M}
        & \multicolumn{4}{c|}{N} & \ldots \\ \hline
        % --------------------------------------------------
        \multirow{2}{*}{Подоболочка} & K & L\(_1 \) & L\(_2 \) & M\(_1 \)
        & M\(_2 \) & M\(_3 \) & N\(_1 \) & N\(_2 \) & N\(_3 \) & N\(_4 \)
        & \ldots \\ \cline{2-12}
        & 1s & 2s & 2p & 3s & 3p & 3d & 4s & 4p & 4d & 4f & \ldots \\ \hline
    \end{tabular}
\end{table}

Для полностью заполненной подоболочки характерно равенство нулю суммарного
орбитального и суммарного спинового магнитных моментов. Следовательно, момент
импульса полностью заполненной подоболочки равен нулю. Значит, при определении
чисел \( L \) и \( S \) всего атома полностью заполненные подоболочки не
принимаются во внимание.

\subquestion{Объяснение периодической системы элементов Менделеева}

Принцип Паули дает объяснение периодической повторяемости свойств атомов.
Наблюдаемая периодичность свойств атомов объясняется поведением внешних
валентных электронов. Как оказалось, эта периодичность связана с определенной
периодичностью электронной конфигурации атомов.

\begin{minipage}{.35\textwidth}
    \flushleft
        \begin{tabular}{*{2}{C{.12}}*{4}{|C{.12}}}
            & \( ^2\mathrm{S}_{1/2} \) & \( ^1\mathrm{S}_0 \)
            & \( ^2\mathrm{S}_{1/2} \) & \( ^1\mathrm{S}_0 \)
            & \( ^2\mathrm{P}_{1/2} \) \\[-1.5em]
            \multirow{2}{*}{2p} &&&&& \multirow{2}{*}{\( \up \)} \\ \cline{2-6}
            \multirow{2}{*}{2s} &&& \multirow{2}{*}{\( \up \)}
            & \multirow{2}{*}{\( \updown \)} & \multirow{2}{*}{\( \updown \)} \\
            \cline{2-6}
            \multirow{2}{*}{1s} & \multirow{2}{*}{\( \up \)}
            & \multirow{2}{*}{\( \updown \)} & \multirow{2}{*}{\( \updown \)}
            & \multirow{2}{*}{\( \updown \)} & \multirow{2}{*}{\( \updown \)} \\
            \cline{2-6}
            & H & He & Li & Be & B \\
        \end{tabular}
\end{minipage} \hfill
\begin{minipage}{.5\textwidth}
    H: \( L = 0 \), \( S = 1/2 \), \( J = 1/2 \) \\
    He: \( L = 0 \), \( S = 0 \), \( J = 0 \) \\
    Li: \( L = 0 \), \( S = 1/2 \), \( J = 1/2 \) \\
    Be: \( L = 0 \), \( S = 0 \), \( J = 0 \) \\
    B: \( L = 1 \), \( S = 1/2 \), \( J = 1/2,\,3/2 \)
\end{minipage}

Подоболочка 3d оказывается энергетически выше подоболочки 4s, то есть сначала
заполняется подоболочка 4s, а затем заполняется 3d, или, другими словами, при
незаполненной оболочке M начинает заполняться N.

\begin{table}[h!]
    \center
    \caption{Возможные состояния электрона в атоме}
    \begin{tabular}{|C{.1}|*{4}{C{.05}|}C{.15}|*{2}{C{.065}|}} \hline
        Оболочка & \( n \) & \( l \) & \( m_l \) & \( m_s \) & Подоболочка
        & \multicolumn{2}{c|}{Число \( e^- \)} \\ \hline
        % --------------------------------------------------
        K & 1 & 0 & 0 & \( \updown \) & K (1s) & \multicolumn{2}{c|}{2} \\\hline
        % --------------------------------------------------
        \multirow{4}{*}{L} & \multirow{4}{*}{2} & 0 & 0 & \( \updown \)
        & L\(_1 \) (2s) & 2 & \multirow{4}{*}{8} \\ \cline{3-7}
        & & \multirow{3}{*}{1} & --1 & \( \updown \)
        & \multirow{3}{*}{L\(_2 \) (2p)} & \multirow{3}{*}{6} & \\ \cline{4-5}
        & & & 0 & \( \updown \) & & & \\ \cline{4-5}
        & & & 1 & \( \updown \) & & & \\ \hline
        % --------------------------------------------------
        \multirow{9}{*}{M} & \multirow{9}{*}{3} & 0 & 0 & \( \updown \)
        & M\(_1 \) (3s) & 2 & \multirow{9}{*}{18} \\ \cline{3-7}
        & & \multirow{3}{*}{1} & --1 & \( \updown \)
        & \multirow{3}{*}{M\(_2 \) (3p)} & \multirow{3}{*}{6} & \\ \cline{4-5}
        & & & 0 & \( \updown \) & & & \\ \cline{4-5}
        & & & 1 & \( \updown \) & & & \\ \cline{3-7}
        & & \multirow{5}{*}{2} & --2 & \( \updown \)
        & \multirow{5}{*}{M\(_3 \) (3d)} & \multirow{5}{*}{10} & \\ \cline{4-5}
        & & & --1 & \( \updown \) & & & \\ \cline{4-5}
        & & & 0 & \( \updown \) & & & \\ \cline{4-5}
        & & & 1 & \( \updown \) & & & \\ \cline{4-5}
        & & & 2 & \( \updown \) & & & \\ \hline
    \end{tabular}
\end{table}

\subquestion{Правила Хунда}
\begin{enumerate}
    \item Минимумом энергии при данной электронной конфигурации обладает терм с
    наибольшим числом \( S \) и наибольшим возможным при данном \( S \) числом
    \( L \).
    \item Если электронная оболочка заполнена менее, чем наполовину, то число
    \( J = |L - S| \), иначе -- \( J = L + S \).
\end{enumerate}

\newpage
