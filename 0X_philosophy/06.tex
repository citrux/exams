\question{Основные проблемы и понятия философской мысли Древнего Китая.}
Специфика древнекитайской цивилизацией определялась наличием мощного централизованного государства с развитым бюрократическим аппаратом. Реальными субъектами власти выступали не жрецы, а чиновники, которые являлись носителями светского знания и образования. Доминирующим регулятором социальной практики являлась не религия, а ритуал, регламентирующий все сферы общественной жизни.

Выделяют 6 основных течений китайской мысли:
\begin{itemize}
    \item конфуцианство\\
    Основатель --- Конфуций, идеи которого изложены в книге Лунь Юй. Центральное место занимают вопросы о нравственной природе человека, семье и управлении государством. Центральное понятие --- жень(человеколюбие), определяющее идеальные отношения между людьми в соответствии с принципом взаимности: чего не желаешь себе, того не делай другим. Основная внутренняя мотивация человека связана с долгом как честным выполнением предписанных судьбой обязанностей. При этом каждый должен соответствовать своему статусу и не претендовать на неподобающее место. Началом, образующим все человеческие качества, выступает семья, поэтому почтительное отношение к старшим лежит в основе ритуала. Государственная иерархия выстраивается по подобию семьи (правитель как отец своих подданных).
    \item моизм выдвигал идею всеобщей любви и взаимной выгоды как основных политических принципов
    \item легизм высказывал идею государства, построенного на справедливых законах и широко использующего поощрения и наказания
    \item школа имён (логика) центрировалась вокруг проблемы соотношения действительности и имени (понятия, названия), настаивая на корректном употреблении последних
    \item школа инь-ян была ориентирована на натурфилософию. Инь --- тёмное, женское начало, ян --- мужское, светлое. Основные состояния космоса и Поднебесной описываются комбинациями инь и ян. Соединение инь и ян происходит в ци --- универсальной жизненной энергии. Данные понятия дополняются учением о пяти элементах: воде, огне, металле, дереве и земле.
    \item даосизм --- основная оппозиция конфуцианству. Основатель --- Лао Цзы (Дао дэ цзин). Центральное понятие --- дао (путь), универсальный закон возникновения и исчезновения отдельных явлений и мира в целом. Предназначение человека --- следовать дао. Основной принцип у вэй заключается в отсутствии целеполагания и растворении своего Я в естественном порядке вещей. Цель пути --- бессмертие.
\end{itemize}

Китайскую философию характеризует:
\begin{itemize}
    \item отсутствие метафизических систем и направленность на регуляцию социального опыта
    \item традиционализм, как отражение идеала социальной стабильности с одной стороны, и как принцип организации философских школ с другой,
    \item литературно-афористичный стиль изложения.
\end{itemize}
