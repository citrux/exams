\question{Категория материи. Формы движения материи. Движение и развитие.}
Материя -- фундаментальная исходная категория фил., от того или иного ее понимания зависит решение практ. всех других ф. проблем. От лат. materia -- вещество. Материя сущ. только в многообразии конкретных объектов, через них, а не наряду с ними. Категория материи является фундаментальным философским понятием.Способ определить материю - выделить такой предельно общий признак, который характеризует любые виды материи независимо от того, познаны они уже или еще только будут познаны в будущем. Таким общим признаком является свойство <<быть объективной реальностью, существовать вне нашего сознания>>. Определяя материю посредством этого признака, диалектический материализм неявно
предполагает бесконечное развитие материи и ее неисчерпаемость.

Движение -- важнейший атрибут, способ существования материи. Д. включает в себя все происходящие в природе и обществе процессы. В самом общем виде Д. -- это изменение вообще, всякое взаимодействие материальных объектов и смена их состояний.

Иерархия форм движения по Энгельсу:
\begin{enumerate}
    \item механическое
    \item физическое
    \item химическое
    \item биологическое
    \item социальное
\end{enumerate}

Развитие, в отличие от движения, обозначает закономерные необратимые качественные изменения. В зависимости от направления развития выделяют прогрессивное, регрессивное и нейтральное, а в зависимости от способа --- эволюционное и революционное.