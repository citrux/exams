\question{Основные проблемы и понятия философской мысли Древней Индии.}
 Мифологически-религиозные представления и зачатки философской мысли Древней Индии изложены в ведах.

 В ведах говорится, что есть три сферы вселенной: небо, земля и воздушное пространство между ними. Им соответствуют три группы богов (всего 33 божества). Но есть высшее духовное начало, брахман, стоящее выше богов. Что касается человека, то тело считается смертным, а душа (атман) --- вечной. Душа переселяется из одной телесной оболочки в другую в соответствии с кармой. Благочестивое поведение, в конечном счёте, может освободить душу от связи с телесностью.

 Философские школы Древней Индии разделяют на ортодоксальные (признающие авторитет вед) и неортодоксальные.

 Неортодоксальные:
 \begin{itemize}
    \item джайнизм\\
    существует 5 субстанций:
    \begin{itemize}
        \item материя
        \begin{itemize}
            \item земля
            \item вода
            \item огонь
            \item воздух
        \end{itemize}
        \item время
        \item пространство
        \item покой
        \item движение
    \end{itemize}
    Каждое живое существо имеет душу, уоторая привязана к материи тончайшей оболочкой (карманашира). Задача человека --- достигнуть освобождения через разъединение души и материи. Путь к этому через правильную веру, правильное познание и правильное поведение. Джайнизм говорит о пяти видах познания --- от обычного к ясновидению и телепатии, подчеркивает особую роль ненанесения вреда жизни.
    \item буддизм, в основе которого лежит концепция срединного пути, согласно которой следует избегать крайностей. Центральным пунктом является признание 4 благородных истин и восьмеричного пути. Истины гласят: жизнь в мире полна страданий, существуют причины этих страданий, можно прекратить эти страдания и существует путь, ведущий к прекращению страданий. Этот путь включает в себя правильные взгляды, правильную решимость, правильную речь, правильное поведение и т.д. В итоге наступает нирвана --- угасание страстей, а вместе с ними и страданий.
    \item чарвака-локаята\\
    единственный надёжный источник познания --- чувственное восприятие, поэтому вера в богов, переселение души и судьбу считается ошибочной. Душа --- результат особой комбинации материальных элементов. В жизни сочетаются страдания и наслаждения и нужно стремиться к максимуму наслаждений на основе духовной дисциплины.
\end{itemize}
 
 Ортодоксальные:
 \begin{itemize}
    \item санкхья\\
    первооснова мира пракрити, в которой есть три гуны, дисбаланс которых приводит к явлениям. Так же существует чистый дух --- пуруши --- сознание как таковое. Эволюция вселенной начинается с контакта пракрити и пуруши. Большая роль уделяется страданиям, которые существуют из-за отсутствия совершенного знания. Для того, чтобы избежать страданий, нужно понять, в чём заключается наше истинное я.
    \item йога --- призывает к ограничению деятельности ума, предусматривается дисциплина ума и тела, воздержание от греха, сосредоточение.
    \item вайшештка --- коренной причиной страдания является незнание.
    \item ньяя --- на первом плане вопросы о познании, его источниках, отличиях ложного познания от истинного.
    \item миманса --- развивается учение о пяти источниках правильного познания.
    \item веданта --- истинная реальность заключена в духовном начале, и через познание себя происходит освобождение.
\end{itemize}