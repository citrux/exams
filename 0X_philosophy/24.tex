\question{Человек как предмет философского анализа.}

Основные трактовки человека:
\begin{itemize}
    \item человек -- одна из основных проблем философской рефлексии;
    \item рассмотрение человека как особой темы становится универсальной тенденцией развития различных наук:
    \begin{itemize}
        \item политэкономии и социологии;
        \item биологии и медицины;
        \item астрономии и географии;
        \item этнографии и антропологии;
        \item и т.д.
    \end{itemize}
    \item в искусстве все большее место стала занимать идея преломления природных и социальных явлений 
        сквозь призму видения их человеком;
    \item повышение интереса к философскому анализу проблемы человека продиктовано сегодня
    \begin{itemize}
        \item новым этапом НТР;
        \item развитием мирового сообщества;
        \item экологической ситуацией;
        \item и т.д.
    \end{itemize}
    \item невозможно однозначно соотнести определение человека с каким-либо родовым понятием;
    \item философское постижение человека происходит через осмысление его бытия в мире;
\end{itemize}

Человек и наука:
\begin{itemize}
    \item связь человека с неким феноменом объясняется все человеческое бытие в целом;
    \item наука ориентирована лишь на эмпирический уровень человеческого бытия;
    \item философия абстрагируется от действительности, чтобы понять не только то, что есть, но и как должно 
        быть;
\end{itemize}

История философии человека:
\begin{itemize}
    \item истории философии человек понимался традиционно в единстве таких его основных модусов:
    \begin{itemize}
        \item тело;
        \item душа;
        \item дух.
    \end{itemize}
    \item образ тела в философии
    \begin{itemize}
        \item физическая субстанция человеческой жизни, выступающая как элемент природы;
        \item спектр человеческих чувств и состояний, как совесть, стыд, смех, плач и т.п. 
    \end{itemize}
\end{itemize}

Душа рассматривается как интегративное начало, промежуточное звено, соединяющее тело и дух, придающее 
человеку целостность. Для современной философии душа -- наиболее сложная и противоречивая тема, 
рассматриваемая в двух основных ракурсах:
\begin{itemize}
    \item как жизненный центр тела, являющийся той силой, которая, будучи сама бессмертной,  очерчивает срок 
        телесного существования (в связи с признанием существования или несуществования души 
        в философии возникали вопросы о смерти и бессмертии, бытии и небытии);
    \item как индивидуализирующее свойство человека в обществе, описываемое в философии через 
        проблемы свободы воли, творчества, рока и судьбы.
\end{itemize}

Дух и человек:
\begin{itemize}
    \item дух воплощает в себе фундаментальную идею <<человечности>>;
    \item дух выступает как родовая человеческая способность, соотносящаяся с разумом, сознанием и 
        социальностью;
    \item понятии духа отражается не только феномен <<духовности>>, как интегративного начала культуры и 
        общества, но и личностные характеристики отдельного человека;
    \item человека нельзя упрощенно представить как диаду (тело –- дух) или триаду (тело –- дух –- душа);
    \item человек -- уникальная целостность с трудной дифференциацией телесного, душевного и духовного 
        уровней; 
\end{itemize}