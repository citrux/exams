\question{Многообразие форм человеческого знания. Проблемы истины. Знание и вера.}

Формы знания:
\begin{itemize}
    \item Научное -- объективное, истинное знание.
    \item Ненаучное -- разрозненное, несистематизированное знание, которое находится в противоречии с 
        существующей научной картиной мира.
    \item Донаучное -- прототип, предпосылочная база научного знания.
    \item Паранаучное -- несовместимое с имеющимся научным знанием.
    \item Лженаучное -- знание, сознательно использующее домыслы и предрассудки.
    \item Антинаучное -- знание утопично и сознательно искажающее представления о действительности.
    \item Личностное -- знание, являющееся достоянием отдельной личности.
\end{itemize}

Тезисы
\begin{itemize}
    \item к исторически \emph{первым формам человеческого знания} относят \textbf{игровое познание}, которое 
        строится на основе условно принимаемых правил и целей;
    \item \emph{Особую разновидность знания}, являющегося достоянием отдельной личности, представляет 
        \textbf{личностное знание};
    \item \textbf{мифологическое знание} -- это особый вид целостного знания, в рамках которого человек 
        стремится создать целостную картину мира, опираясь на совокупность эмпирических сведений, верований, 
        различных форм образного освоения мира;
    \item \textbf{религиозное знание} -- это целостно-мировоззренческое знание обусловлено эмоциональной 
        формой отношения людей к господствующим над ними высшими силами (природными и социальными);
    \item \textbf{художественное знание} -- это знание, опирающиеся на художественный опыт -- это наглядное 
        знание.
\end{itemize}

Особенности обыденного познания:
\begin{itemize}
    \item складывается стихийно под воздействием ежедневного опыта;
    \item не предполагает постановку задач, которые бы шли дальше повседневной практики;
    \item обусловлено, социальной, профессиональной, национальной, возрастной особенностью носителя;
    \item передача знаний предполагает личное общение с носителем этого знания;
    \item осознано не в полном объёме;
    \item низкий уровень формализации.
\end{itemize}

Особенности научного знания:
\begin{itemize}
    \item строгая доказательность, обоснованность, достоверность результатов;
    \item ориентация на объективную истинность, проникновение в сущность вещей;
    \item универсальный надличностный характер;
    \item воспроизводимость результата;
    \item логически организовано и системно;
    \item обладает особым, высокоформализованым языком;
    \item скептическое отношение к авторитетам.
\end{itemize}

Проблемы истины
\begin{itemize}
    \item является центральной во всей теории познания;
    \item отождествляется с самой сущностью, является одним из самых важнейших мировоззренческих понятий;
    \item находится в одном ряду с такими ключевыми явлениями, как \emph{Добро}, \emph{Зло}, 
        \emph{Справедливость}, \emph{Красота};
    \item каждом этапе своего развития человечество располагало только \emph{относительной истиной};
    \item знание каждой исторической эпохи содержит в себе элементы \emph{абсолютной истины}, поскольку оно 
        имеет объективно истинное содержание;
\end{itemize}

Способы толкования истины:
\begin{itemize}
    \item[Онтологическое] <<Истина -- то, что есть>>;
    \item[Гносеологическое] <<Истина -- когда знания соответствуют действительности>>;
    \item[Позитивистское] <<Истина должна подтверждаться опытом>>;
    \item[Прагматическое] <<Истина -- полезность, эффективность знания>>;
    \item[Конвенциональное] <<Истина -- это соглашение>>.
\end{itemize}

Знание и вера
\begin{itemize}
    \item Знание
    \begin{itemize}
        \item претендует на адекватное отражение действительности;
        \item делает истину доступной для субъекта посредством доказательства;
        \item рассматривается как результат познавательной деятельности;
        \item считается, что именно научное знание говорит от имени истины и позволяет субъекту с 
            определенной мерой уверенности ею распоряжаться;
        \item как способ приобщения субъекта к истине обладает объективностью и универсальностью.
    \end{itemize}
    \item Вера
    \begin{itemize}
        \item есть сознательное признание чего-либо истинным на основании преобладания субъективной 
            значимости;
        \item обнаруживает себя в непосредственном, не требующем доказательства принятии тех или иных 
            положений, норм, истин;
        \item проявляется в состоянии убежденности и связана с чувством одобрения или неодобрения;
        \item внутреннее духовное состояние.
    \end{itemize}
\end{itemize}