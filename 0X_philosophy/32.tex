\question{Общество. Культура и цивилизация. Основные сферы жизни общества.}

\subquestion{Культура и цивилизация}
Культура~-- исторически определённый уровень развития общества, творческих сил и способностей
человека, выраженный в типах и формах организации жизни и деятельности людей, в их взаимоотношениях,
а также в создаваемых ими материальных и духовных ценностях.

Цивилизация~-- совокупность материальных и духовных достижений общества в его историческом развитии,
уровень общественного развития и материальной культуры, достигнутый в том или ином обществе; степень
и характер развития культуры определённых эпох и народов.

Цивилизация выражает нечто общее, рациональное, стабильное. Она представляет собой систему отношений,
закрепленных в праве, в традициях, способах делового и бытового поведения. Они образуют механизм,
гарантирующий функциональную стабильность общества. Цивилизация определяет общее в сообществах,
возникающих на базе однотипных технологий.

Культура~-- есть выражение индивидуального начала каждого социума. Исторические этносоциальные
культуры есть отражение и выражение в нормах поведения, в правилах жизни и деятельности, в традициях
и привычках не общего у разных народов, стоящих на одной цивилизационной ступени, а того, что
специфично для их этносоциальной индивидуальности, их исторической судьбы, индивидуальных и
неповторимых обстоятельств их прошлого и сегодняшнего бытия, их языка, религии, их географического
местоположения, их контактов с другими народами и т.д. Если функция цивилизации~-- обеспечение
общезначимого, стабильного нормативного взаимодействия, то культура отражает, передает и хранит
индивидуальное начало в рамках каждой данной общности.

Если культура характеризует меру развития человека, то цивилизация характеризует общественные условия
этого развития, социальное бытие культуры. 

\subquestion{Сферы жизни общества}
Сфера жизни общества~-- определенная совокупность устойчивых отношений между социальными субъектами.

Каждая сфера включает в себя:
\begin{itemize}
    \item определенные виды деятельности человека;
    \item социальные институты;
    \item сложившиеся отношения между людьми.
\end{itemize}

Традиционно выделяют четыре основные сферы общественной жизни:
\begin{enumerate}
    \item социальная~-- это отношения, которые возникают при производстве непосредственной
        человеческой жизни и человека как социального существа. Сюда входят отношения между
        различными профессиональными и социально-демографическими слоями населения, а также
        национальных общностей.
        
        Функционирование социальной сферы связано с удовлетворением круга социальных потребностей,
        таких как решение проблем здравоохранения, образования, обеспечения уровня жизни, реализации
        права на труд и т.п.
        
    \item Экономическая~-- это совокупность отношений людей, возникающих при создании и перемещении
        материальных благ. Это сфера функционирования производства, непосредственного воплощения в
        жизнь достижений научно-технического прогресса, реализации всей совокупности производственных
        отношений людей, в том числе отношений собственности на средства производства, обмена
        деятельностью и распределения материальных благ.
        
        Производственные отношения и производительные силы в совокупности составляют экономическую
        сферу жизни общества:
        \begin{itemize}
            \item производительные силы~-- люди (рабочая сила), орудия труда, предметы труда;
            \item производственные отношения~-- производство, распределение, потребление, обмен.
        \end{itemize}
        
        В экономической сфере непосредственно воплощаются в жизнь экономическое сознание людей, их
        материальная заинтересованность в результатах своей производственной деятельности, а также их
        творческие способности. Здесь же реализуется деятельность институтов управления экономикой.
        
    \item Политическая~-- это отношения людей, связанные прежде всего с властью, которые обеспечивают
        совместную безопасность.
        
        Элементы политической сферы можно представить таким образом:
        \begin{itemize}
            \item политические организации и институты;
            \item политические нормы~-- политические, правовые и моральные нормы, обычаи и традиции;
            \item политические коммуникации~-- отношения, связи и формы взаимодействия между
                участниками политического процесса, а также между политической системой и обществом;
            \item политическая культура и идеология.
        \end{itemize}

        Потребности и интересы формируют определенные политические цели социальных групп. На этой
        целевой основе возникают политические партии, общественные движения, властные государственные
        институты, осуществляющие конкретную политическую деятельность. 
        
    \item Духовная~--– это сфера отношений людей по поводу разного рода духовных ценностей, их
        создания, распространения и усвоения всеми слоями общества. При этом под духовными ценностями
        подразумеваются не только произведения искусства, но и знания людей, наука, моральные нормы
        и т.д.
        
        Духовная жизнь общества складывается из повседневного духовного общения людей и из таких
        направлений их деятельности, как познание, в том числе научное, образование и воспитание, из
        проявлений морали, искусства, религии.
        
        Большое значение в данном отношении имеет деятельность учреждений, выполняющих функции
        образования и воспитания, а также атмосфера семейного воспитания человека, круг его
        сверстников и друзей, его духовное общение с другими людьми. Немаловажную роль в формировании
        духовности человека играет искусство.
\end{enumerate}
