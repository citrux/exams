\question{Русская философская мысль второй половины XIX -- начала XX веков (идеи одного из русских философов).}

Религиозная философия:
\begin{enumerate}
	\item С. Н. Булгаков 
	\begin{itemize}
		\item выдвинул идею объединения всех христианских церквей в единую христианскую <<экуменическую>> церковь;
		\item причина всех бед -- разобщение: в обществе -- на экономическую, политическую и духовную сферы; в религии -- разные христианские церкви;
		\item был весьма религиозен -- считал, что люди несут ответственность перед богом после смерти.
	\end{itemize}
	\item П. А. Флоренский
	\begin{itemize}
		\item мир рассматривал как единое целое, целостный мир антиномичен (соткан из противоречий);
		\item предполагал, что будущие открытия прольют свет на взаимоотношения материи и духа;
	\end{itemize}
\end{enumerate}

Русский космизм:
\begin{enumerate}
	\item В. И. Вернадский 
	\begin{itemize}
		\item рассматривал факторы, влияющие на геологические процессы на земле, выявил, что с каждым годом человек меняет планету больше, чем процессы в атмосфере, литосфере, гидросфере и биосфере, это привело его к понятию ноосферы;
		\item ноосфера или сфера разума -- часть биосферы, литосферы, атмосферы и гидросферы, изменённая людьми, в совокупности с их духовной и материальной культурой;
		\item ноосфера постоянно расширяется, Вернадский надеялся, что однажды она превратится в ноокосмос.
	\end{itemize}
	\item К. Э. Циолковский
	\begin{itemize}
		\item астрономические исследования и рост числа открытых звёзд и звёздных скоплений привели к тому, что Циолковский задался вопросом о бесконечности Вселенной, Вселенная по Циолковскому бесконечна и имеет простую иерархичную структуру -- атомы складываются в обычные тела, тела -- в планеты и звёзды, планеты и звёзды -- в галактики, галактики -- в скопления галактик и т. д. до бесконечности;
		\item атомы Циолковский считал неуничтожимыми, по этой причине люди бессмертны, ведь атомы, из которых они состоят воплотятся, в новых живых существах, пусть даже через миллионы или миллиарды лет;
		\item по этой причине Вселенная не могла родиться, а была всегда, потому что если бы она появилась, то можно было бы спросить, а что было до неё -- рождаются и умирают звёзды, галактики, скопления галактик (например, все известные нам на данный момент, но не Вселенная);
		\item Земля из-за бесконечности Вселенной просто не может быть уникальной, существует множество земель и на каждой живут разумные существа, однажды люди встретятся с ними, но перед этим нужно выйти в космос;
		\item выход в космос невозможен без модификации человеческих тел, новые люди должны свободно существовать там, где не будет привычных нам продуктов питания, а будет только излучение, так появилась идея о <<человеке-растении>>.
	\end{itemize}
\end{enumerate}