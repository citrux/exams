\question{Немецкая классическая философия. Концептуальные идеи Гегеля.}
Идеалистическая диалектика Гегеля представила универсальную логику становления и развития разума, природы и истории. Говоря о том, что <<физика должна сделать предметом мышления само мышление>>, Гегель в качестве начала отсчёта берёт абсолютную идею, которая выступает как чистое бытие. Идея выступает не только как первоэлемент мышления, но и как цель, поскольку познание устремлено к истине, также имеющей идеальную природу.

Познание всегда требует своего предмета, поэтому идея опредмечивает себя в действительности, являющейся инобытием идеи.

Путь познания связан с переходом от отдельных истин к более полному знанию. Система Гегеля выстраивается следующим образом:
\begin{enumerate}
     \item логика (идеи как чистые категории)
     \item философия природы (идеи инобытии)
     \item философия духа (человек и история, посредством которых идея возвращается к себе из инобытия)
 \end{enumerate}

 Основная заслуга Гегеля связана с разработкой диалектического метода. Характеристики гегелевской диалектики:
 \begin{itemize}
     \item представление о существовании объективных законов развития в природе, обществе, культуре
     \item учение о противоречии как источнике развития и описание механизмов возникновения и разрешения противоречий (тезис -- антитезис -- синтез)
     \item разработка диалектической системы категорий (количество и качество, сущность и явление)
     \item идеалистический характер диалектики (природные и исторические процессы производны от логики саморазвития абсолютной идеи)
     \item абсолютизация триады и финалистский характер развития (wtf?)
 \end{itemize}