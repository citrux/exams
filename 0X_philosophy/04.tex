\question{Исторические типы философии.}
\begin{enumerate}
    \item Древневосточная философия.\\
        Особенности: глубокая разработка проблем мистического познания;
        тесная связь с социально-политическими реалиями;
        в основе нравственной культуры взаимоотношений человека с окружающим миром~-- идеи космизма.
    \item Античная философия.\\
        Особенности: противопоставление знания и мнения;
        космоцентризм: <<космос>>~-- мир, порядок, всё сущее, <<логос>>~-- основа мира, слово, мысль,
        <<человек>>~-- один из элементов космоса, у него есть свое место и предназначение;
        отделение философии от мифологии, уход от антропоморфизма.
    \item Философия Средневековья.\\
        Особенности: сплав теологии и античной философии;
        Бог~-- первопричина, творец всего сущего, его воля господствует над миром.
    \item Философия эпохи Возрождения.\\
        Особенности: антропоцентризм, гуманизм;
        Бог~-- творец, человек~-- творец культуры, наделен (в отличие от остальной природы) способностью
        мыслить и творить.
    \item Философия Нового времени.\\
        Особенности: гносеологизм~-- доступ к пониманию мира есть познание;
        логоцентризм~-- идеал познания~-- четкое, строго рациональное мышление;
        антропоцентризм;
        наукоцентризм;
        стремление подчинить общественную жизнь законам;
        ориентация на практику, методы и способы познания.
    \item Классическая немецкая философия.\\
        Особенности: вера в могущество разума;
        антропоцентризм, неотчуждаемость прав личности;
        мировой закон вечного развития;
        целостная концепция диалектики;
        история~-- целостный процесс, в ее основе~-- труд;
        роль философии~-- быть совестью культуры.
    \item Неклассическая философия (Западноевропейская философия XIX-XX вв.).\\
        Особенности: иррационализм;
        центральные аспекты духовной жизни: воля, чувство, интуиция, бессознательное, воображение,
        инстинкт;
        позитивизм, философия жизни, экзистенциализм.
    \item Русская философия.\\
	    Особенности:
	    <<византизм>>, православие;
	    слияние восточной и западной философии;
	    дополнение рационального познания внерациональным.
\end{enumerate}
	    
Иногда из западноевропейской философии выделяют также следующее направление:
\begin{enumerate}
    \item Марксистско-Ленинская философия.\\
        Особенности: изучение законов развития общества;
        основа общественной жизни~-- материальное производство;
        государство~-- продукт классовых противоречий;
        обоснование революции;
        идеальное общество~-- коммунистическое общество.
\end{enumerate}
