\question{Немецкая классическая философия: общая характеристика. Философия Канта.}

Немецкая классическая философия, основные произведения которой выходят в 1770-1830 гг., образована 5 авторами:
\begin{itemize}
    \item И. Кантом,
    \item И. Г. Фрихте,
    \item Ф. В. И. Шеллингом,
    \item Г. В. Ф. Гегелем,
    \item и Л. Фейербахом.
\end{itemize}

Значимость НКФ определяется тем, что она стала:
\begin{enumerate}
    \item высшей степенью развития классической рационалистической традиции, ориентированной на разум как принцип организации действительности и механизм её познания;
    \item преддверием неклассического типа философствования, поскольку ряд её положений (критицизм Канта, диалектика Гегеля, антропологизм Фейербаха) стали основами для развития новых философских стратегий.
\end{enumerate}

Критическая философия И. Канта пересматривает идеалы предшествующей гносеологии, усматривающей истину в объективной действительности, и обосновывает необходимость критического исследования самого разума как источника познания. Кант приходит к следующим выводам:
\begin{itemize}
    \item Человеческое познание, начинающееся с определенного вопроса или предположения, всегда активно по отношению к природной действительности, т.к. выявляет в ней то, что оно хочет или готово увидеть, поэтому знание исходит не столько из объекта, сколько из субъекта.
    \item Объективный мир --- мир вещей в себе, он открывает нам лишь те или иные свои проявления (феномены), в то время как подлинная сущность (ноумен) остаётся для нас неизвестной.
    \item Возможность научной истины (в математике и физике) обеспечивается наличием априорных форм чувственности и рассудка. Априорные формы и категории образуют трансцендентальные основания разума и познания
    \item Истины о душе, боге и вселенной находятся в компетенции не теоретического разума, а практического, поскольку они важны для определения смысла и целей человеческого существования и поведения. При этом высшим нравственным законом выступает априорный категорический императив.
\end{itemize}

Кант сместил центры познавательного интереса с объекта на субъект.