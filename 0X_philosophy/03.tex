\question{Мировоззрение (структура, уровни, принципы). Форма мировоззрения.}
Мировоззрение --- это система устойчивых взглядов человека на мир и своё место в нём.

Для мировоззрения характерны следующие черты:
\begin{itemize}
	\item активная форма знания, определяющая жизненную позицию и деятельность человека
	\item целостность
	\item всеобщность
\end{itemize}

Выделяют два уровня мировоззрения:
\begin{enumerate}
	\item обыденно-практический --- здесь формируется повседневное мироощущение, представляющее эмоционально-психическую сторону отношения человека к миру, и мировосприятие, фиксирующее его чувственно-познавательную сторону и задающего целостное видение объекта. На этом уровне мир предстаёт эмоционально-окрашенным, чувственное отношение к нему может быть оптимистичным или пессимистичным
	\item рационально-теоретический --- создаётся обобщенная или концептуально оформленная модель мировоззрения в форме миропонимания, выражающая интеллектуальную сторону отношений человека к миру и раскрывающего их сущность в виде понятий, категорий, теорий, концепций и т.д.
\end{enumerate}

Основные структурные элементы мировоззрения:
\begin{itemize}
	\item знания --- более или менее достоверная информация о реальности
	\item ценности --- смысловые ориентации, наделенные личностной, социальной или культурной значимостью
	\item убеждения --- личностная форма знаний, принимаемых человеком за истину и определяющих его позицию
\end{itemize}

Формы мировоззрения:
\begin{itemize}
	\item мифологическое --- исторически первая форма мировосприятия, понимания мира и самого себя первобытным человеком. Это специфическая система фантастических представлений об окружающей человека природной и социальной реальности. Эту форму отличают следующие признаки:
	\begin{itemize}
		\item синкретизм ---  неразрывное единство человека и природы
		\item антропоморфизм --- перенесение человеческих качеств на неживой и живой мир природы
		\item вера во множество сверхъестественных существ, олицетворяющих разнообразный природный мир
		\item доминирование чувственно-эмоционального восприятия над рациональным познанием
	\end{itemize}
	\item религиозное мировоззрение связано с возникновением монотеистических религий и включает два уровня:
	\begin{itemize}
		\item психологический, отражающий взгляды, чувства, настроение, характер поведения и специфику действий верующих людей
		\item идеологический, представляющий систему догматических знаний о божественной реальности.
	\end{itemize}
	Специфика религиозного мировоззрения проявляется в том, что:
	\begin{itemize}
		\item мир удваивается добавлением сакрального, как прообраза и цели деятельности человека, между мирами устанавливается граница
		\item сущность бога духовна, он творец мира и находится вне его
		\item источником знаний является священное писание и вера, которая выше разума
		\item религиозная картина мира иерархична: бог -- человек -- животные -- растения -- неживой мир.
	\end{itemize}
	\item философское мировоззрение отличается своей направленностью на интеллектуальный поиск истины, что предполагает:
	\begin{itemize}
		\item осмысление предельных оснований бытия и мышления
		\item обоснование ценностного отношения людей к действительности
		\item стремление к систематической целостности знания о мире, человеке и характере их взаимоотношений
		\item логическую аргументацию выдвигаемых положений
		\item опору на разум и внутренний опыт человека
	\end{itemize}
	Философия рассматривается в качестве теоретического ядра мировоззрения.
\end{itemize}