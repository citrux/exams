\question{Проблемы познания в философии. Уровни и формы познания.}

Познание -- процесс целенаправленного активного отображения действительности в сознании человека.

Гносеология -- раздел философии, который изучает процессы познания.

Существуют две точки зрения на процесс познания:
\begin{itemize}
	\item гностицизм (мир познаваем, а человек обладает безграничными возможностями познания);
	\item агностицизм (либо мир непознаваем, либо возможности познания ограничены).
\end{itemize}

Иммануил Кант выдвинул последовательную теорию агностицизма:
\begin{itemize}
	\item человек обладает ограниченными возможностями познания;
	\item мир непознаваем в принципе, так как человек может познать внешнюю сторону предметов, но никогда не сможет постичь внутреннюю сущность -- <<вещей в себе>>.
\end{itemize}

В рамках гностицизма выделяют:
\begin{enumerate}
	\item эмпиризм (Бэкон), мешают познанию
	\begin{itemize}
		\item идол рода (ограниченность органов чувств);
		\item идол пещеры (субъективность восприятия мира);
		\item идол рынка или площади (неправильное употребление слов);
		\item идол театра (некритические заимствования).
	\end{itemize}
	основной метод -- индукция (от частного к общему),  эмпиризм вырос из средневекового номинализма;
	\item рационализм (Декарт),
	основной метод -- дедукция (от общего к частному),  рационализм вырос из средневекового реализма,
	<<нелепо полагать несуществующим то, что мыслит, ... я мыслю, следовательно существую ... и ... оно ... вернейшее из всех заключений>>, Декарт предполагал также, что есть врождённые идеи (идея бога, идея чисел и фигур, и общие понятия (из ничего ничего не происходит)); 
	\item сенсуализм (к сенсуалистам можно было бы отнести Локка, если бы он не считался эмпириком, сочувствовавшим рационалистам), основной метод -- ощущение, врождённые идеи отрицаются. 
\end{enumerate} 

2 формы познания:

\begin{itemize}
	\item чувственное (изучает внешние свойства, элементы: ощущение (отдельная часть), восприятие (объект как целое), представление (когда нет предмета));
	\item рациональное (изучает внутренние свойства, элементы: понятие, суждение, умозаключение);
\end{itemize}

Различают два уровня научного познания:
\begin{itemize}
	\item эмпирический (наблюдение, эксперимент, сравнение, измерение, описание);
	\item теоретический (выдвижение, построение и разработку научных гипотез и теорий; формулирование законов; выведение логических следствий из законов; сопоставление друг с другом различных гипотез и теорий, теоретическое моделирование, а также процедуры объяснения, предсказания и обобщения).
\end{itemize}