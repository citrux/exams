\question{Проблема сознания в философии: происхождение и сущность.}

Актуальность проблемы сознания в философии объясняется
\begin{itemize}
    \item без выяснения природы человеческого сознания нельзя определить место и роль человека в мире, 
        особенности его взаимоотношений с окружающей действительностью;
    \item вопрос о сущности сознания, о её связи с бытием является одним из важнейших мировоззренческих и 
        методологических аспектов каждого философского направления;
    \item все проблемы современной общественной практики органически связаны с исследованиями сознания:
    \begin{itemize}
        \item проблемы общественного развития;
        \item взаимодействия человека и техники;
        \item отношение научно-технического прогресса и природы;
        \item проблем воспитания;
        \item общения людей.
    \end{itemize}
\end{itemize}

При описании структуры сознания обычно выделяются следующие сознания:
\begin{itemize}
    \item[во-первых], оно включает в себя информацию о внешнем мире, объекте;
    \item[во-вторых], оно направлено и на самого носителя, субъекта сознания, т.е. сознание выступает в 
        качестве самосознания.
\end{itemize}

Сознание включает несколько основных структурных блоков, главными из которых являются:
\begin{itemize}
    \item познавательные процессы, к которым относятся ощущения, восприятия, представления, мышление, 
        память, язык и речь;
    \item эмоциональные состояния -- позитивные и негативные, активные и пассивные и т.д.;
    \item волевые процессы -- принятие и исполнение решений, волевые усилия.
\end{itemize}

Сущность сознания как высшей формы отражения выражается в том, что:
\begin{itemize}
    \item чувственное отражение наполняется более глубоким и осознанным содержанием;
    \item сознание отражает мир не в чувственно-наглядных, а в идеальных образах;
    \item человеческое отражение носит не приспособительный, а активно-преобразовательный характер.
\end{itemize}

\emph{Отражение} -- всеобщее свойство материи, заключающееся в воспроизведении признаков, свойств и 
отношений отражаемого объекта. 