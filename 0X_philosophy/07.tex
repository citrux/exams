\question{Восточная и западная философия: сравнительная характеристика.}

Непредвзятый анализ показывает, что хотя различия между восточной и западной филосо­фией существуют,
они не являются чрезмерными. Говорят, что западная философия рациональна, научна, натуралистична,
ориентирована на прогресс и активную преобразующую деятельность, а восточная философия религиозна,
мистична, ин­туитивна, нацелена на эстетико-этическую просвещенность.

Под философией древнего Востока подразумеваются философии древ­ней Индии и древ­него Китая, под
философией древнего Запада~-- древней Греции и древнего Рима.

Сходство в проблематике философии древнего мира Запада и Востока:
\begin{enumerate}
    \vspace*{-2ex}
    \itemsep-1ex
    \item философия за­рождается в лоне мифологии как первой формы общест­венного созна­ния;
    \item философия зарождается как форма общественного сознания с возник­новением клас­сового
        общества и государства;
    \item философия обращена к общечеловеческим ценностям. В центре познания~-- проблемы
        добра и зла; прекрасного и безобраз­ного; справедливо­сти и несправедливости; дружбы,
        товарищества, любви и ненависти; счастья, наслаждения и др.
    \item Мировоззренче­ский характер философского знания. Философские идеи, взгляды, теории,
        системы являются либо идеалисти­ческими, либо материалистическими, а иногда
        эклектическими (со­едине­ниями этих двух типов мировоззре­ний). Однако, философские
        взгляды философов не высту­пают однозначно~-- только как мате­риалистические или только
        как идеалистические. В них сочетаются те и другие идеи.
    \item Стремление философии к научному поиску истинного зна­ния, имею­щего методоло­гическую
        значимость (методологическая функция филосо­фии). С помощью философских учений,
        концепций, идей осуществляется ана­лиз самых раз­личных явлений, даются практические
        рекомендации.
    \item Философы вырабатывают свой собственный метод исследования, ана­лиза, объясне­ния
        явлений. Два основных философских метода~-- диа­лектический и метафизиче­ский~--
        использу­ются философами стихийно. Философы схваты­вают общую взаимосвязь явлений, их
        противоречивость, движение и развитие, единство и многообразие мира.
    \item Философия оказала огромное влияние на последующую философ­скую мысль, куль­туру,
        развитие человеческой цивилизации.
    \vspace*{-2ex}
\end{enumerate}

Традиционно признается, что западная культура экстенсивна и экстравертирована, т.е. направлена на
преобразование окружающего мира, в то время как Восток более ориентирован на постижение внутреннего
мира человека, т.е. интровертирован.

Основные различия приведены в таблице~\ref{tab:q07tab01}.
\begin{table}
    \centering
    \small
    \caption{Отличия востока и запада}
    \label{tab:q07tab01}
    \begin{tabular}{|C{.13}*{2}{|m{.41\textwidth}}|} \hline
        Предмет &
            \centering\arraybackslash Восток &
            \centering\arraybackslash Запад \\ \hline
        Проблематика &
            Внимание сконцентрировано на про­блеме человека &
            Многопроблемная филосо­фия \\
        Проблема человека &
            Исследуется с точки зрения практики, жизнедеятельно­сти людей, их образа жизни &
            Исследуется не через психиче­ское бытие и этикет, а че­рез общие принципы бытия и позна­ния \\
        Отношение к религии &
            Тесное взаимодействие с религией &
            Приверженность научной ме­тодологии, отмежевание от рели­гии, сильна атеистическая тенден­ция \\
        Разработка категориального аппарата &
            Органически воспринимаются многие категории, предложенные мифологией и «Ригведами» (инь, янь, ци). Часто рассматри­ваются категории души и физического тела, материи и души, сознания и его состояний &
            Разрабатывается на основе преемственности его разви­тия, вы­ражающего суть философских сис­тем и их борьбу \\
        Учение о строении материи &
            Проблема дискретности материи, ее строения не ставится &
            Возникла атомистика как учение о дискретном строении ма­терии \\
        Проблема познания &
            Проблемой логики занима­лась лишь индий­ская школа Ньяя &
            Познание рассматривается не только как эмпирическое, но и как логи­че­ское \\
        Социальная проблематика &
            Проблемы <<вселенского человека>>, от которого пошло все, рассмотрение общечеловеческих ценно­стей &
            Более широкий спектр проблем. Обращение к человеку в единстве с обществом. Преобладает этическая тематика \\
        Смысл духовной цивилизации &
            Обращение к бытию личности через уход от материального мира &
            Откры­тость изменениям, поискам истины в различных направлениях \\ \hline
    \end{tabular}
\end{table}
