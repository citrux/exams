\question{Восточная и западная философия: сравнительная характеристика.}

Непредвзятый анализ показывает, что хотя различия между восточной и западной философией существуют,
они не являются чрезмерными. Говорят, что западная философия рациональна, научна, натуралистична,
ориентирована на прогресс и активную преобразующую деятельность, а восточная философия религиозна,
мистична, интуитивна, нацелена на эстетико-этическую просвещенность.

Под философией древнего Востока подразумеваются философии древней Индии и древнего Китая, под
философией древнего Запада~-- древней Греции и древнего Рима.

Сходство в проблематике философии древнего мира Запада и Востока:
\begin{enumerate}
    \vspace*{-1.8ex}
    \itemsep-1.2ex
    \item философия зарождается в лоне мифологии как первой формы общественного сознания;
    \item философия зарождается как форма общественного сознания с возникновением классового
        общества и государства;
    \item философия обращена к общечеловеческим ценностям. В центре познания~-- проблемы
        добра и зла; прекрасного и безобразного; справедливости и несправедливости; дружбы,
        товарищества, любви и ненависти; счастья, наслаждения и др.
    \item Мировоззренческий характер философского знания. Философские идеи, взгляды, теории,
        системы являются либо идеалистическими, либо материалистическими, а иногда
        эклектическими (соединениями этих двух типов мировоззрений). Однако, философские
        взгляды философов не выступают однозначно~-- только как материалистические или только
        как идеалистические. В них сочетаются те и другие идеи.
    \item Стремление философии к научному поиску истинного знания, имеющего методологическую
        значимость (методологическая функция философии). С помощью философских учений,
        концепций, идей осуществляется анализ самых различных явлений, даются практические
        рекомендации.
    \item Философы вырабатывают свой собственный метод исследования, анализа, объяснения
        явлений. Два основных философских метода~-- диалектический и метафизический~--
        используются философами стихийно. Философы схватывают общую взаимосвязь явлений, их
        противоречивость, движение и развитие, единство и многообразие мира.
    \item Философия оказала огромное влияние на последующую философскую мысль, культуру,
        развитие человеческой цивилизации.
    \vspace*{-1.8ex}
\end{enumerate}

Традиционно признается, что западная культура экстенсивна и экстравертирована, т.е. направлена на
преобразование окружающего мира, в то время как Восток более ориентирован на постижение внутреннего
мира человека, т.е. интровертирован.

Основные различия (Проблема: Восток / Запад):
\begin{enumerate}
    \vspace*{-1.8ex}
    \itemsep-1.2ex
    \item проблематика: внимание сконцентрировано на проблеме человека / многопроблемная философия;
    \item проблема человека: исследуется с точки зрения практики, жизнедеятельности людей, их образа
        жизни / исследуется через общие принципы бытия и познания;
    \item отношение к религии: тесное взаимодействие с религией / приверженность научной
        методологии, отмежевание от религии, сильна атеистическая тенденция;
    \item разработка категориального аппарата: органически воспринимаются многие категории,
        предложенные мифологией и <<Ригведами>> (инь, янь, ци), рассматриваются категории души и
        физического тела, материи и души, сознания и его состояний / разрабатывается на основе
        преемственности его развития, выражающего суть философских систем и их борьбу;
    \item учение о строении материи: проблема дискретности и строения материи не ставится / возникла
        атомистика;
    \item проблема познания: проблемой логики занималась лишь индийская школа Ньяя / познание
        рассматривается не только как эмпирическое, но и как логическое;
    \item социальная проблематика: проблемы <<вселенского человека>>, от которого пошло все,
        рассмотрение общечеловеческих ценностей / более широкий спектр проблем, обращение к человеку
        в единстве с обществом, этика;
    \item смысл духовной цивилизации: обращение к бытию личности через уход от материального мира /
        открытость изменениям, поискам истины в различных направлениях.
\end{enumerate}
