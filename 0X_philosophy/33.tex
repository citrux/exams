\question{Философия истории. Многовариантность исторического процесса.}

Одним из первых философов истории считается Августин Аврелий, рассматривавший человеческую историю как закономерный процесс спасения человечества после грехопадения.

История как закономерный одновариантный процесс:
\begin{itemize}
	\item формационный подход (Ленин, Маркс, Энгельс) (история -- это закономерный процесс смены общественно-экономических формаций, в каждой формации есть \textit{базис} -- экономика, и \textit{надстройка} -- система общественных отношений, изменения в базисе приводят к конфликту с надстройкой и изменению надстройки), формации:
	\begin{enumerate}
		\item \textit{первобытно-общинная}: примитивный труд, личная свобода, отсутствие частной собственности, слабая общественная организация;
		\item \textit{рабовладельческая}: частная собственность, общественная организация, публичная власть, социальное неравенство;
		\item \textit{феодальная}: крупная земельная собственность, личная свобода, но экономическая (политическая) зависимость, появление ремесленных центров;
		\item \textit{капиталистическая}: промышленность, механизация, личная свобода рабочих, но их экономическая зависимость от буржуазии;
		\item \textit{коммунистическая}: государственная собственность, труд на общественное благо, равномерное распределение продуктов производства, высокий уровень развития производственных сил.
	\end{enumerate}
	\item цивилизационный подход (Тойнби):
	\begin{enumerate}
		\item центральное понятие -- цивилизация -- устойчивая общность людей, объединённых общими традициями, образом жизни, географическими и историческими рамками; 
		\item цивилизации бывают основными (шумерская, вавилонская, христианская и др.), которые влияют на другие цивилизации (в основном религиозно) и локальными -- замкнутыми в национальных границах (американская, русская, германская и т. д.);
		\item движущие силы истории -- вызов, брошенный цивилизации извне, ответ на вызов и великие личности; развитие строится по принципу <<вызов-ответ>>;
		\item каждая цивилизации проходит 4 стадии: зарождение, рост, надлом, смерть (дезинтеграция).
	\end{enumerate}
	\item культурологический подход (Шпенглер):
	\begin{enumerate}
		\item центральное понятие -- культура -- совокупность религии, традиций, материальной и духовной жизни, цивилизация здесь -- высший уровень развития культуры, формируется перед смертью культуры; 
		\item 8 культур --  индийская, китайская, вавилонская, египетская, античная, арабская, русская, западноевропейская;
		\item культура зарождается, живёт и умирает.
	\end{enumerate}
	\item Ясперс:
	\begin{enumerate}
		\item центральное понятие -- \textit{историческая ситуация} -- термин по сути характеризует географическое положение, особенности структуры общества и экономическое развитие, а также духовную жизнь общества (всё-таки Ясперс -- экзистенциалист); 
		\item по мнению Ясперса в истории в 800-200 гг. до н. э. сложились одинаковые исторические ситуации в Китае, Индии, Персии, Палестине, Древней Греции, что привело к зарождению основ религии и переходу от мифологии к религии, последнее связано с поиском смысла существования и философией, данный период он назвал \textit{осевой эпохой мировой истории};
		\item дальнейшее развитие человечества он видит в объединении с помощью \textit{философской веры} -- синтезом религии и науки, которые открывают смысл жизни и истории.
	\end{enumerate}
\end{itemize}

История как многовариантный процесс:
\begin{itemize}
	\item Кьеркегор:
	\begin{enumerate}
		\item выступал за личную историю каждого отдельного индивидуума, выбор пути у каждого свой и пока он есть человек -- свободен.
	\end{enumerate}
	\item Хайдеггер:
	\begin{enumerate}
		\item полагал, что у каждого народа своя история и она состоит в сохранении единства 4 начал -- земли, неба, смертных и богов;
		\item выпадение одного звена из этой цепочки приводит к гибели народа -- последний распадается на массы людей, у каждой из которых формируется своя история. 
	\end{enumerate}
	\item Сартр:
	\begin{enumerate}
		\item считает, что экзистенциальные условия (обычно это различные духовные переживания) существенно расширяют и дополняют позиции марксизма на исторический процесс; это приводит к тому, что в истории появляется случайный фактор, который нарушает закономерности.
	\end{enumerate}
\end{itemize}