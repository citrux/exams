\question{Греческая натурфилософия. Ранние философские школы: поиск первоосновы мира.}

Натурфилософия обозначает философию природы. Целью натурфилософов было найти \textit{<<конечные причины>>} и фундаментальные закономерности природных явлений. 

Отличие натурфилософии от философии науки заключается в том, что первая основывается на умозрительных заключениях и построения натурфилософов в этом смысле слабо отличаются от мифов.

Отличие от мифологии заключается в том, что вместо вопроса <<\textit{кто} всё создал?>> они задали вопрос <<\textit{что} стало первопричиной всего?>> 

Для натурфилософии характерны:
\begin{enumerate}
	\item космоцентризм (космос, как окружающий мир, является центром изучения);
	\item поиск первоначала, от которого всё произошло;
	\item гилозоизм (одушевление неживой природы).
\end{enumerate}

К натурфилософам относят представителей \textit{Милетской школы} (VI в. до н. э.):
\begin{enumerate}
	\item \textit{Фалес}:
	\begin{itemize}
		\item считал первоначалом воду -- <<архэ>>;
		\item Землю считал плоским диском на воде;
		\item предполагал, что все вещи имеют души;
		\item допускал наличие множества богов;
		\item в центр Вселенной помещал Землю;
		\item *** а также определил продолжительность года и доказал ряд математических теорем, что делает его нормальным чуваком среди философской братии.
	\end{itemize}
	\item \textit{Анаксимандр} ученик кадра сверху:
	\begin{itemize}
		\item считал первоначалом <<апейрон>> -- вечную, неизмеримую бесконечную субстанцию из которой всё вышло и в которую всё превратится;
		\item мир -- цилиндры: <<Холод в соединении с влагою и сухостью образовал Землю, имеющую форму цилиндра, основание которого относится к высоте как 3:1, и занимающую центр вселенной. Солнце находится в высшей небесной сфере, больше Земли в 28 раз и представляет полый цилиндр, из которого изливаются огненные потоки; когда отверстие закроется, происходит затмение. Луна тоже цилиндр и в 19 раз больше Земли; при наклонном положении ее получаются лунные фазы, а затмение происходит тогда, когда она совсем перевернется>>;
		\item выступал за самозарождение жизни.
	\end{itemize}
	\item \textit{Анаксимен} ученик кадра сверху:
	\begin{itemize}
		\item считал первоначалом воздух, он же повсюду; как следствие все вещества полагал состоящими из воздуха, просто где-то воздуха больше где-то меньше; превращения воздуха: воздух -- вода -- ил -- почва -- камень и т. д.; 
		\item космология: <<пределы мироздания состоят из земли и хрусталя; звезды суть материальные тела, облитые огнем; солнце, движением которого обусловливаются времена года, представляет такой же плоский круг, как и висящая в воздухе Земля, вокруг которой все двигается>>;
	\end{itemize}
\end{enumerate}

Другая фигура среди натурфилософов \textit{Гераклит Эфесский}:
\begin{itemize}
	\item этот был фаталистом, по его мнению всё происходит согласно некоторой закономерности, подчиняясь судьбе-необходимости -- \textit{Логосу} (всеобщему закону); 
	\item мир по его мнению непрерывно меняется, все мы слышали выражение <<В одну и ту же реку нельзя войти дважды>> оказывается оно вырвано из контекста <<В одну и ту же реку нельзя войти дважды и нельзя дважды застигнуть смертную природу в одном и том же состоянии ... Солнце не только новое каждый день, но вечно и непрерывно новое>>;
	\item движущая сила всех изменений -- война, борьба противоположностей, в ходе которой противоположности соединяются.
\end{itemize}

Есть и ещё несколько искателей первоначал, которых почему-то не всегда относят к натурфилософам:

\textit{Пифагорейцы}
\begin{itemize}
	\item первоначало -- число; 
	\item мир познаётся через число и потому сводится к числу;
	\item 1 -- мельчайшая частица всего;
	\item пифагорейцы искали протокатегории "чётное-нечётное", "светлое-темное" и т. д.
\end{itemize}

Представители \textit{Элейской школы} (элеаты) (VI в. до н. э.):
\begin{enumerate}
	\item \textit{Парменид}:
	\begin{itemize}
		\item всё сводит к двум первоначалам -- Эфирному огню (чистому свету, тёплому) и густой тьме (ночи, холоду);
		\item в центре философии -- бытие: <<бытие есть, небытия нет, ибо его невозможно ни познать, ни высказать>>, <<мыслимо только сущее>>;
		\item космогония: бытие вечно, так как из ничего не может возникнуть что-то, бытие неизменно, всё заполнено бытием, поэтому всё непрерывно;
		\item космология: Вселенная состоит из концентрических кругов, которые лежат слоями вокруг Земли, все круги окружает небесная твердь, в центре правит великая богиня Правды и необходимости.
	\end{itemize}
	\item \textit{Ксенофан}:
	\begin{itemize}
		\item Бог = Природа (Космос) = Бытие.
	\end{itemize}
	\item \textit{Зенон} продолжатель дела Парменида и Ксенофана:
	\begin{itemize}
		\item доказывал иллюзорность многообразия мира и движения с помощью апорий (с греч. тупик, трудность); 
		\item апории: <<Дихотомия>>, <<Ахилл и черепаха>>, <<Стрела>>, <<Движущиеся тела>>, все об одном и том же -- чтобы достичь цели нужно преодолеть половину пути, потом половину от половины и т. д., поэтому движения нет;
		\item ценность апорий Зенона заключается в том, что он впервые поставил вопрос о смысле бесконечного и бесконечно делимого.
	\end{itemize}
\end{enumerate}

И последний яркий представитель ранних греческих философских школ -- \textit{Демокрит} (460-370 гг. до н. э.):
\begin{itemize}
	\item этот придумал атомы, идея состояла в том, что если бы бытие делилось бесконечно, как это делал Зенон, то это означало бы что оно целиком состоит из пустоты и ничего нет, поэтому атом -- то что неделимо, чтобы хоть что-то было; 
	\item атомы находятся в постоянном движении, всё состоит из атомов;
	\item атомы бывают различной формы и размера (круглые, продолговатые и т. д.);
	\item атомы могут сталкиваться и соединяться, что и привело к созданию Земли, вещей, живых существ;
	\item Вселенная бесконечна и бесконечно в ней количество миров;
	\item душа человека состоит из огненных атомов (то есть самых быстрых и мелких!), душа удерживается в теле за счёт того, что мы вдыхаем воздух, со смертью тела душа также умирает, разумная часть души находится в районе грудной клетки;
	\item процесс познания связан с тем, что все вещи испускают атомы -- образы вещей и поглощаются атомами души, в философии это называют \textit{принципом истечения};
	\item ничего в мире не происходит и не образуется без причины это порой трактуется как крайний детерминизм, то есть все наши будущие действия заранее предопределены.
\end{itemize}
