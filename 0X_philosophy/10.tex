\question{Основные идеи и положения философии Платона и Аристотеля.}

Философия Платона:
\begin{itemize}
    \item общие понятия нашего сознания (идеи/эйдосы) имеют самостоятельно существование вне призрачного мира 
        материальных вещей;
    \item источник истинного знания -- в воспоминаниях о мире идей, который душа созерцала до своего 
        <<телесного рождения>>;
    \item существуют два мира: мир идей (эйдосов) и мир вещей; любая вещь является лишь отражением своей 
        идеи, может стремиться к ней, но никогда не достигнет её;
    \item противопоставляет душу и тело как две разнородные сущности;
    \item эмпирическое исследование является маловажным;
    \item путь к истинному знанию -- диалектика понятий;
\end{itemize}

Платон выделяет три начала души:
\begin{enumerate}
    \item \textbf{Разумное начало}, обращённое на познание и всецело сознательную деятельность.
    \item \textbf{Яростное начало}, стремящееся к порядку и преодолению трудностей.
    \item \textbf{Страстное начало}, выражающееся в бесчисленных вожделениях человека.
\end{enumerate}

Философия Аристотеля:
\begin{itemize}
    \item идеи неотделимы от вещей и имеют свое бытие в них;
    \item идея и реальное явление существуют не отдельно друг от друга, а в неразрывном сочетании;
    \item идея -- форма дающая материи смысл;
    \item не существует особого мира идей и в начале жизни душа подобна чистой вощёной дощечке для письма 
        (tabula rasa), на которой ещё ничего не начертано; затем она постепенно заполняется <<отпечатками>>, 
        приобретаемыми из опыта;
    \item миря явлений обладает подлинной реальностью и содержит в себе истину;
    \item душа при помощи индукции переходит к более сложным общим понятиям;
    \item эмпирическое исследование -- краеугольный камень философии;
    \item путь к истинному знанию -- рациональная логика.
\end{itemize}

Учение Аристотеля чаще всего подразделяют на четыре раздела:
\begin{enumerate}
    \item Логика (излагается в группе трудов, получивших общее название <<Органон>>).
    \item Теоретическая философия. Она в свою очередь делится на:
    \begin{enumerate}
        \item <<Первую философию>> -- учение об основах истинного бытия. В своде сочинений Аристотеля 
            <<первая философия» была размещена после его сочинений по физике и из-за этого чисто случайного 
            обстоятельства получила название «метафизики», получившее затем у философов гораздо более важный 
            смысл. 
        \item Математику –- учение о количестве и протяжении;
        \item Физику –- учение о движении предметов. Понятие <<движения>> Аристотель толкует весьма широко, 
            понимая его как любое изменение качества.
    \end{enumerate}
    \item Практическая философия –- учение о принципах человеческой жизни и деятельности. Она состоит из:
    \begin{enumerate}
        \item Этики (науки о цели жизни отдельного человека).
        \item Экономики (науке о домашнем хозяйстве).
        \item Политики (науки о государстве).
    \end{enumerate}
    \item Поэтическая философия: трактаты Аристотеля о поэзии и риторике.
\end{enumerate}