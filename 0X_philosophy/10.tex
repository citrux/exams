\question{Основные идеи и положения философии Платона и Аристотеля.}

Непосредственным предшественником философии Аристотеля было учение Платона, однако их сразу разделило 
важнейшее теоретическое отличие. По системе Платона, общие понятия нашего сознания (идеи) имеют 
самостоятельно существование вне призрачного мира материальных вещей. По Аристотелю же, идеи неотделимы от 
вещей и имеют свое бытие в них. Аристотель полагает, что идея и реальное явление существуют не отдельно друг 
от друга, а в неразрывном сочетании. Идея -- только форма, дающая материи смысл.

Согласно философии Платона, источник истинного знания -- в воспоминаниях о мире идей, который душа созерцала 
до своего <<телесного рождения>>. Но Аристотель считает, что никакого особого мира идей нет, и в начале жизни 
душа подобна чистой вощёной дощечке для письма (tabula rasa), на которой ещё ничего не начертано. Затем она 
постепенно заполняется <<отпечатками>>, приобретаемыми из опыта. Аристотель, в отличие от Платона, убеждён, 
что мир явлений не есть <<ложный призрак>>, а обладает подлинной реальностью и содержит в себе истину. 
Получая от него простые чувственные впечатления, душа при помощи индукции переходит к более сложным общим 
понятиям.

В соответствии со всеми этими взглядами, эмпирическое исследование, маловажное для Платона, делается у 
Аристотеля краеугольной основой философии. Платон считает путем к истинному знанию диалектику понятий, 
Аристотель -- рациональную логику. Эта отрасль знания образует краеугольный камень его философии. Аристотель 
разработал её в такой полноте, что впоследствии к ней не было сделано почти никаких дополнений.

Учение Аристотеля чаще всего подразделяют на четыре раздела:
\begin{enumerate}
    \item Логика (излагается в группе трудов, получивших общее название <<Органон>>).
    \item Теоретическая философия. Она в свою очередь делится на:
    \begin{enumerate}
        \item <<Первую философию>> -- учение об основах истинного бытия. В своде сочинений Аристотеля 
            <<первая философия» была размещена после его сочинений по физике и из-за этого чисто случайного 
            обстоятельства получила название «метафизики», получившее затем у философов гораздо более важный 
            смысл. 
        \item Математику –- учение о количестве и протяжении;
        \item Физику –- учение о движении предметов. Понятие <<движения>> Аристотель толкует весьма широко, 
            понимая его как любое изменение качества.
    \end{enumerate}
    \item Практическая философия –- учение о принципах человеческой жизни и деятельности. Она состоит из:
    \begin{enumerate}
        \item Этики (науки о цели жизни отдельного человека).
        \item Экономики (науке о домашнем хозяйстве).
        \item Политики (науки о государстве).
    \end{enumerate}
    \item Поэтическая философия: трактаты Аристотеля о поэзии и риторике.
\end{enumerate}

%% urls
% http://rushist.com/index.php/philosophical-articles/2425-filosofiya-aristotelya-kratko
% http://rushist.com/index.php/philosophical-articles/2245-filosofiya-platona-kratko