\question{Особенности развития русской философии. Основные этапы и идеи.}

Русская философия -- феномен мировой философской мысли, так как развивалась независимо от европейской и мировой философии без влияния эмпиризма, рационализма, идеализма.

Характерные черты:
\begin{itemize}
	\itemsep-1ex
	\item сильно религиозное влияние;
	\item специфическая форма выражения философских мыслей -- художественное творчество, публицистика;
	\item целостность -- все занимались всем комплексом философских идей;
	\item большая роль проблем морали и нравственности.
\end{itemize}

Основные проблемы:
\begin{itemize}
	\itemsep-1ex
	\item проблема человека;
	\item космизм, как восприятие космоса как единого организма;
	\item мораль и нравственность;
	\item проблема выбора исторического пути России;
	\item проблема власти, государства, социальной справедливости, идеального общества.
\end{itemize}

Этапы и идеи:
\begin{enumerate}
	\itemsep-1ex
	\item Период зарождения древнерусской и раннехристианской философии (IX-XIII вв.), представители и идеи:
	\begin{itemize}
		\itemsep-.5ex
		\item Иларион (<<Слово о законе и благодати>>) -- анализ роли христианства в настоящем и будущем Руси;
		\item Владимир Мономах (<<Поучение>>) -- моральные и нравственные ценности;
		\item Климент Смолятич (<<Послание пресвитеру Фоме>>) -- проблемы разума и познания;
		\item Филипп Пустынник (<<Плач>>) -- соотношение плотского и духовного.
	\end{itemize}
	\item Период становления Московской Руси (XII-XVII вв.), представители и идеи:
	\begin{itemize}
		\itemsep-.5ex
		\item Филофей (XVI в.) -- идея <<Москва --- Третий Рим>>;
		\item Юрий Крижанич -- выступал против схоластики, занимался вопросами познания, в качестве первопричины всего сущего видел бога;
	\end{itemize}
	\item Философия XVIII в., представители и идеи:
	\begin{itemize}
		\itemsep-.5ex
		\item Татищев, Кантемир, Прокопович -- вопросы устройства монархии, права императора и его божественность;
		\item Ломоносов -- сторонник механистического материализма, атомизм, деизм (допускал наличие бога, но не наделял его сверхъестественной силой);
		\item Радищев -- сторонник борьбы против самодержавия, сторонник народовластия, свободы;
	\end{itemize}
	\item Философия XIX в., декабристы, монархисты, западники и славянофилы, революционеры-де\-мо\-кра\-ты, атеисты, космисты, ... идеи:
	\begin{itemize}
		\itemsep-.5ex
		\item Соловьёв -- идея объединения всех сторон бытия, прогресс как всеобщая связь поколений, воскрешение всех живых (духовное воскрешение) и мертвых -- цель человечества, идея Всеобщей Божественной Мудрости -- Софии;
		\item Л. Н. Толстой -- попытался создать новую религию: Бог -- добро, любовь, разум, совесть; смысл жизни -- самосовершенствование; необходимо отказаться от насилия и государства -- не ходить на работу, не участвовать в политической жизни.
	\end{itemize}
	\item Философия XX в., космизм, русское зарубежье, ленинизм, идеи:
	\begin{itemize}
		\itemsep-.5ex
		\item Питирим Сорокин -- создал теорию стратификации и социальной мобильности, страты -- социальные группы, выделенные по некоторым признакам, переходы между стратами -- социальная мобильность, она делает демократию устойчивой, история -- процесс смены ценностей, развитие науки и техники -- угроза для человечества.
	\end{itemize}
\end{enumerate}
