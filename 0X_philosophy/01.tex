\question{Проблема происхождения философии. Миф и логос.}
Философия как одна из самых древних форм духовной культуры насчитывает 25 столетий своей истории. Возникнув в странах Древнего Востока, своего наивысшего расцвета она достигла в Древней Греции.

Философия возникает в определённых социокультурных условиях:
\begin{enumerate}
	\item формирование товарно-денежных отношений
	\item выделение ремесла из сельского хозяйства
	\item становление государства и безличного писанного права
	\item социальная дифференциация общества
	\item отделение умственного труда от физического
\end{enumerate}

Духовным источником возникновения философии может рассматриваться переход от мифа к логосу --- рациональному и достоверному знанию. Теоретическая мысль возникает там, где некоторые люди стали иметь досуг ---  свободное время.

Основные концепции генезиса философии:
\begin{enumerate}
	\item мифогенная --- в процессе рационализации мифов и перевода чувственно-конкретного мифологического мышления в понятийно-логическую форму
	\item гносеогенная --- в результате развития и обобщения протонаучного знания
	\item гносеогенно-мифогенная --- совокупность развитой мифологии, зачатков наук и обыденного мышления.
\end{enumerate}