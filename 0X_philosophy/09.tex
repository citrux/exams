\question{Проблемы человека и познания: споры Сократа и софистов.}

\emph{Софисты}, или мудрецы, современники Сократа -- это профессиональные риторы-преподаватели, философы, 
стремившиеся познать основы мироздания и изложить открывшиеся знания в пространных учениях. 

Основные тезисы
\begin{itemize}
    \item софисты -- <<платные учителя мудрости>>, впервые взявшие деньги за обучение философии;
    \item \emph{главной темой учений} -- исследование первопричин бытия, его составляющих частей и 
        движущих начал;
    \item признаком неистинности такого рода философии Сократ считал коренные, неразрешимые 
        противоречия её отдельных учений;
    \item Сократ приходит к выводу о принципиальной непознаваемости абсолютных истин;
    \item ограниченный человеческий разум неспособен вместить в себя всех вселенских тайн, и это знание 
        может быть дано ему только путём откровения (Сократ восстал против бесплодного умствования софистов);
    \item вес философской мысли (по Сократу) должен быть перенесен с недоступных <<дел божественных>> на 
        <<дела человеческие>>, которые зависят от свободной воли человека и находятся в его власти. 
    \item Сократ пытается подвигнуть своих слушателей к исканию истинных норм человеческих отношений 
        посредством самоиспытания и самопознания;
    \item по Сократу истинно только то знание, которое может быть применено на практике;
    \item вопрос о существовании объективной истины:
    \begin{itemize}
        \item софисты -- отсутствии истины вне человека;
        \item Сократ -- нахождение всеобщей истины, применимой к каждому из нас;
    \end{itemize}
    \item различия в отношении Сократа и софистов к мифу, к мифологическим образам и преданиям:
    \begin{itemize}
        \item софисты стремились к аллегорическому истолкованию мифов, пытались найти в мифических 
            представлениях разумный смысл;
        \item Сократ -- попытки интерпретировать мифы о богах и героях является бесполезным занятием;
    \end{itemize}
\end{itemize}