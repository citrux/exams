\question{Информационная революция и становление информационного общества.}

В истории развития цивилизации произошло несколько информационных революций
\begin{itemize}
    \item \emph{Первая} революция связана с изобретением письменности, что привело к гигантскому 
        качественному и количественному скачку. Появилась возможность передачи знаний от поколения к 
        поколениям.
    \item \emph{Вторая} вызвана изобретением книгопечатания, которое радикально изменило индустриальное 
        общество, культуру, организацию деятельности.
    \item \emph{Третья} обусловлена изобретением электричества, благодаря которому появились телеграф, 
        телефон, радио, позволяющие оперативно передавать и накапливать информацию в любом объеме.
    \item \emph{Четвертая} связана с изобретением микропроцессорной технологии и появлением персонального 
        компьютера.
\end{itemize}

Четвёртый период характеризуюттри фундаментальные инновации:
\begin{itemize}
    \item переход от механических и электрических средств преобразования информации к электронным;
    \item миниатюризация всех узлов, устройств, приборов, машин;
    \item создание программно-управляемых устройств и процессов.
\end{itemize}

Для этой стадии развития общества и экономики характерно:
\begin{itemize}
    \item увеличение роли информации, знаний и информационных технологий в жизни общества;
    \item возрастание числа людей, занятых информационными технологиями, коммуникациями и производством 
        информационных продуктов и услуг, рост их доли в валовом внутреннем продукте;
    \item нарастающая информатизация общества с использованием телефонии, радио, телевидения, сети Интернет, 
        а также традиционных и электронных СМИ;
    \item создание глобального информационного пространства, обеспечивающего:
    \begin{itemize}
        \item эффективное информационное взаимодействие людей;
        \item их доступ к мировым информационным ресурсам;
        \item удовлетворение их потребностей в информационных продуктах и услугах.
    \end{itemize}
    \item развитие электронной демократии, информационной экономики, электронного государства, электронного 
        правительства, цифровых рынков, электронных социальных и хозяйствующих сетей.
\end{itemize}