\question{Иррационализм в философии XIX века.}

Иррационализм -- направление в философии, настаивающее на ограниченности человеческого ума в постижении мира. 

Иррационализм предполагает существование областей миропонимания, недоступных разуму, и достижимых только 
через такие качества, как \emph{интуиция}, \emph{чувство}, \emph{инстинкт}, \emph{откровения}, 
\emph{вера} и т. п.

Основоположником европейского иррационализма является \textbf{Шопенгауэр}. 

Основное по Шопенгауэру
\begin{itemize}
    \item мир может обнаруживаться человеком и как воля, и как представление;
    \item воля -- это абсолютное начало всякого бытия, некая космическая и биологическая по своей природе 
        сила, сотворившая мир и человека;
    \item человек -- раб воли, поскольку служит не себе, а Абсолюту;
    \item воля заставляет человека жить, каким бы бессмысленным ни было его существование;
    \item воля заманивает индивида призраками счастья и такими соблазнами, как, например, сексуальное 
        наслаждение;
    \item человек имеет для воли косвенное значение так как служит средством для ее сохранения;
    \item у человека есть только один выход -- погасить в себе волю к жизни;
    \item каждый человек располагает тремя высшими благами жизни -- здоровьем, молодостью и свободой.
\end{itemize}

Предшественниками иррационализма в философии были \emph{Ф.~Г.~Якоби}, и, прежде всего, 
\emph{Г.~В.~Й.~Шеллинг}. Но, как утверждал \emph{Фридрих Энгельс}, работа Шеллинга Философия откровения 
(1843) представляет собой <<первую попытку сделать из преклонения перед авторитетами, гностических фантазий 
и чувственной мистики свободную науку мышления>>.

Ключевым элементом иррационализм становится в философиях \emph{С.Кьеркегора}, \emph{А. Шопенгауэра} и 
\emph{Ф. Ницше}. Влияние этих философов обнаруживается в самых различных направлениях философии (прежде 
всего немецкой), начиная с философии жизни, неогегельянства, экзистенциализма и рационализма вплоть до 
идеологии немецкого национал-социализма. Даже критический рационализм \emph{К. Поппера}, часто называемый 
автором самой рациональной философией, характеризовался как иррационализм.

Необходимо мыслить \emph{дислогично}, соответственно, \emph{иррационально}, чтоб познать 
\emph{иррациональное}. 

\textbf{Логика} -- рациональный способ познания категорий бытия и небытия.

Основное по Ницще
\begin{itemize}
    \item жизнь как <<воля к власти>>;
    \item всё живое стремится к власти;
    \item неравенство создаёт естественное разграничение;
    \item жизнь -- это борьба всех против всех, в ней побеждает сильнейший;
    \item насилие есть чистое проявление прирожденной воли человека к власти;
    \item главная причина краха современной ему цивилизации философ видел в засилии интеллекта, в 
        преобладании его над волей;
    \item там, где интеллект возвышается над волей, она обречена на неминуемое разложение, поэтому разум 
        должен быть подчинен воле и работать как орудие власти;
    \item пытался разорвать границы чисто теоретического познания и ввести в него в качестве регулятора 
        практическую жизнь;
    \item одним из первых сказал о наступлении нигилизма;
\end{itemize}