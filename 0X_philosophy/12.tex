\question{Антропоцентризм, гуманизм и натурфилософия в философской мысли европейского Возрождения.}

Философию Возрождения отличал ярко выраженный антропоцентризм. Если в Средневековье человек рассматривался в соотношении с богом, то для Возраждения характерно понимание человека как самостоятельной личности. Аскетизму и вере в потусторонний мир были противопоставлены светские интересы.

Гуманизм:
\begin{itemize}
    \item Франческо Петрарка --- родоначальник; развитие души и тела человека, ЗОЖ, равенство всех людей, сжечь попов.
    \item Никколо Макиавелли --- новая концепция государства, в противоположность теократической. Государство для самозащиты людей друг от друга, а то всех хочется побольше собственности.
    \item Эразм Роттердамский --- <<Похвала глупости>>, необходимость радикальной реформы церкви.
    \item Томас Мор --- <<Утопия>>, к чёрту частную собственность, даёшь общественную!
\end{itemize}

В эпоху Возрождения складывается новая натурфилософия, для которой характерны следующие черты:
\begin{itemize}
    \item пантеизм --- идея взаимопроникновения природы и бога
    \item идея тождества микро- и макромира, органистическое мировоззрение, трактующее природу по аналогии с человеком
    \item гилозоизм --- идея оживлённости и даже одушевлённости всего бытия
    \item идея самодеятельности материи 
\end{itemize}
Отдельные натурфилософы высказывают оригинальные идеи:
\begin{itemize}
    \item Николай Кузанский: нет никакого центра мира
    \item Леонардо да Винчи: научное познание должно основываться на опыте и, исходя из опыта, выявлять причинные связи явлений
    \item Бернардино Телезио: началом вещей служит материя, которая сама по себе пассивна; кроме неё есть активные силы --- тепло и холод, которые придают вещам свойства и формы.
    \item Джордано Бруно: вселенная вечна, материя самодвижется, существуют другие формы жизни во вселенной, мне лижут пятки языки костра\ldots
    \item Галилео Галилей: разработал принципы механики, сделал ряд важных астрономических открытий (мамой клянусь, она вертится!), отказался от гилозоистических представлений в пользу геометрически-механистических; говорил о необходимости сочетания анализа и синтеза в научном исследовании.
\end{itemize}