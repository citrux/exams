\question{Предмет философии. Структура и специфика философского знания.}
Предметом философии в самом общем виде можно полагать целостное познание предельных оснований бытия природы, человека, общества и культуры.

В структуре философского знания выделяют следующие основные разделы:
\begin{itemize}
	\item онтология --- философия бытия, учение о наиболее общих принципах и основаниях всего сущего
	\item гносеология --- учение о познании
	\item аксиология --- учение о ценностях
	\item \ldots
\end{itemize}

Специфика философии раскрывается в сравнении с другими формами духовной культуры:
\begin{itemize}
	\item по предметной направленности философия сближается с религией, поскольку они представляют из себя ценностно-ориентированные формы знания и выходят за пределы эмпирической реальности. С мифологией философию объединяет ориентация на целостное воспроизведение мира и человека. В то же время этот критерий отличает философию от науки, так как научное знание предметно и фрагментарно
	\item по характеру знания, языковым средствам и специфике рефлексии философия близка к науке, так как для них характерно построение знания в теоретической форме, использование отвлеченных понятий, опора на критерии разума и логическую доказательность выводов
	\item по критерию природы текста и статусу автора философия ближе к искусству, так как и для философского, и для художественного текста принципиальное значение имеет авторский стиль произведения и позиция самого автора
	\item \ldots
\end{itemize}