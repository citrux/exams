\question{Философские течения XX-XXI вв.: прагматизм, геменевтика, феноменология, постмодернизм.}

Прагматизм -- направление идеалистической философии, которое имеет своей главной целью не поиск абстрактной истины, а выработку конкретных средств по решению сложных жизненных ситуаций.

Представители Ч. Пирс, У. Джемс, Джон Дьюи.  

Идеи Дьюи:
\begin{itemize}
	\item отвергал идею первотолчка, поиски первопричины считал бессмысленными;
	\item центральное понятие философии -- опыт, т. е. всё, что имеется в человеческом сознании и врождённое, и приобретенное;
	\begin{enumerate}
		\item человеческое сознание -- носитель опыта;
		\item опыт бывает чувственный, сверхчувственный (спиритический, духовный), религиозный, моральный, художественный, социальный, культурный и др.
	\end{enumerate} 
	\item \textit{цель философии} -- помочь человеку в потоке опыта двигаться к поставленной цели и достигать её, для этого общество должно совершенствоваться, то есть должна происходить \textit{социальная реконструкция} и \textit{совершенствование мышления};
	\item при решении проблем выделяет этапы:
	\begin{enumerate}
		\item неопределённая ситуация (ощущение затруднения, поиск источника и формулировка проблемы);
		\item проблематическая ситуация (ставится цель, изучаются возможные трудности);
		\item выдвижение гипотезы (формулировка последовательности действий);
		\item критический анализ гипотезы (предвидение результатов);
		\item проверка гипотезы на практике.
	\end{enumerate}
\end{itemize}

Герменевтика -- направление в философии, которое исследует теорию и практику истолкования, интерпретации и понимания.

Основные вопросы: Как возможно понимание? как устроены бытие, сущность которого состоит в понимании?

Главная идея герменевтики: существовать -- значит быть понятым.

Изначально герменевтика была направлена на изучение и анализ древних текстов, к первым герменевтам в этом плане можно отнести Августина Блаженного и Фому Аквинского.

Центральные понятия герменевтики: 
\begin{enumerate}
	\item \textit{герменевтический треугольник} -- взаимоотношения между автором текста, читателем и самим текстом;
	\item \textit{герменевтический круг} -- циклический процесс понимания.
\end{enumerate}

Видным представителем данного направления был Шлейермахер, он считал, что правильное толкование текста и следовательно его понимание возможно только в случае родственности душ автора и читателя. Два пути понимания:
\begin{enumerate}
	\item путь дивинации -- вжиться в душу автора произведения;
	\item путь сравнения -- сопоставление фактов и других данных.
\end{enumerate}

Понимание происходит с помощью чередования сравнения и дивинации. Шлеймахер говорил что существует два метода толкования текстов -- грамматический и психологический. В рамках первого выявляется <<дух языка>>, в рамках второго -- <<дух автора>>. Полное понимание наступает тогда, когда читатель поймёт и логику языка, и душу.

Феноменология -- направление в философии, развиваемое Гуссерлем и его последователями. Представляет собой учение о феноменах, т. е. возникающих в сознании смыслах предметов.

Основные особенности:
\begin{itemize}
	\item предлагается отказаться от субъективизма и объективизма, которые по мнению Гуссерля привели к кризису науки (по-моему у всех этих субъективистов личный кризис, Фрейд даже знает какой);
	\item по мнению Гусерля феномен, как структура сознания, объединяет и субъект и объект, а сознание всегда направлено на объект, то есть \textit{интенциально};
	\item сознание представляет собой временной поток, внутренне-организованный и независимый от объекта;
	\item процесс познания по Гуссерлю называется феноменологической редукцией и представляет собой переход к структурам <<чистого сознания>>, например, научным идеализациям, путём <<вынесения за скобки>> внешнего мира.
	\item в качестве примера Гуссерль рассматривает научные идеализации (типа точки и прямой) как предельные субъективные творения; 
\end{itemize}

Постмодернизм -- направление в философии, задача которого сломить многовековой диктат разума, показать, что его претензии на познание истины являются гордыней и ложью, использованными разумом для оправдания своих тоталитарных притязаний.

Основные принципы постмодерна:
\begin{itemize}
	\item объективная сущность – иллюзия;
	\item истина неоднозначна, множественна;
	\item обретение знания – это бесконечный процесс пересмотра словаря;
	\item действительность формируется под воздействием человеческих желаний и поступков;
	\item человеческое познание не отражает мир, а интерпретирует его, и ни одна интерпретация не имеет преимуществ перед другой.
\end{itemize}

Одним из первых мыслителей философии постмодерна является француз Жан Франсуа Лиотар. Он считает, что отличие постмодернистской философии от марксистской – в утверждении идеи выбора из нескольких альтернатив, которые представляются не столько в познанном, сколько в исторической конфигурации жизненных практик, в социальной сфере.

Американский философ Ричард Рорти выдвинул мнение, что вся до сих пор существовавшая философия искажала личностное бытие человека, ибо лишала его творчества.

Рорти совмещает прагматизм с аналитической философией, утверждая, что предметом философского анализа должны быть социум и формы человеческого опыта. Для него социум – общение людей, а главное в нем – интересы личности, «собеседника».

Деррида критикует понимание бытия как присутствия. Он считает, что живого настоящего как такового не существует: оно распадается на прошлое и будущее.

Многие постмодернисты предлагают новый тип философствования – философствование без субъекта.