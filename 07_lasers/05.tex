\question{Ширина спектральной линии излучения. Однородное и неоднородное 
уширения линий.}

Любые процессы, сокращающие время жизни частиц на уровнях, приводят к уширению 
линий соответствующих переходов. Так как определение энергии состояния должно 
проводиться за время, не превышающее время жизни в этом состоянии \( \tau \). 
Тогда неточность определения энергии в соответствии с соотношением 
неопределенности <<энергия -- время>>
\[
	\Delta E \Delta t \gtrsim \hbar
\]
не может быть меньше \( \hbar/\tau \). Неопределенность энергии состояния 
приводит к неопределенности частоты перехода, равной \( 1/2\pi\tau \). 
Постоянная времени \( \tau \) является мерой времени, необходимого для того, 
чтобы возбужденная система отдала свою энергию. Значение \( \tau \) 
определяется скоростями спонтанного излучения и безызлучательных 
релаксационных переходов.

В отсутствие внешних воздействий спонтанное излучение определяет время жизни 
состояния. Поэтому наименьшая возможная, так называемая естественная ширина 
линии \( \Delta\nu_0 \) определяется вероятностью спонтанного перехода 
\( A_{21} \):

\[
	\Delta\nu_0 = \frac{A_{21}}{2\pi}
\]

% нужен нормальный переход

Конечность времени жизни частицы в возбужденном энергетическом состоянии 
ведёт к уширению уровней энергии. Излучение с уширенных уровней приобретает 
спектральную ширину. ...

Уширение линии, обусловленное конечностью времени жизни состояний, связанных 
рассматриваемых переходов, называется однородным. Каждый атом, находящийся в 
в соответствующем состоянии, излучает при переходе сверху вниз линию с полной 
шириной \( \Delta\nu_\text{л} \) и спектральной формой \( q(\nu) \). 
Аналогично каждый атом, находящийся в соответствующем нижнем состоянии, 
поглощает при переходе снизу вверх излучение в спектре с полной шириной 
\( \Delta\nu_\text{л} \) и в соответствии со спектральной зависимостью 
\( q(\nu) \). При однородном уширении вне зависимости от его природы 
спектральная зависимость \( q(\nu) \) есть единая спектральная характеристика 
как одного атома, так и всей совокупности атомов.

Примерами однородного уширения являются естественная ширина линии и 
столкновительное уширение в газах. 

Экспериментально наблюдаемые спектральные линии могут явиться бесструктурной 
суперпозицией нескольких спектрально неразрешимых однородно уширенных линий. 
В этих случаях каждая частица излучает или поглощает не в пределах всей 
экспериментально наблюдаемой линии. Такая спектральная линия называется 
неоднородно уширенной. Причиной неоднородного уширения может быть любой 
процесс, приводящий к различию в условиях излучения (поглощения) для части 
одинаковых атомов исследуемого ансамбля частиц, или наличие в ансамбле атомов 
с близкими, но различными спектральными свойствами, однородно уширенные 
спектральные линии которых перекрываются лишь частично.

Классическим примером неоднородного уширения является доплеровское уширение, 
характерное для газов при малых давлениях и (или) высоких частотах.