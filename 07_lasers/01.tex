\question{Предмет квантовой электроники. История возникновения и развития 
квантовой электроники. Современные достижения и возможности квантовой 
электроники.}

Квантовая электроника -- это область физики, изучающая методы усиления 
и генерации электромагнитного излучения путём использования эффекта 
индуцированного испускания излучения в термодинамически неравновесных 
квантовых системах, свойства получаемых таким образом усилителей и 
генераторов и их применения. Наиболее известными приборами квантовой 
электроники являются мазеры и лазеры. Поэтому в узком смысле слова можно 
говорить о квантовой электроники как о науке о мазерах и лазерах, имея при 
этом в виду, что мазеры -- это квантовые усилители и генераторы когерентного 
электромагнитного излучения радиочастотного (СВЧ) диапазона, а лазеры 
относятся к оптическому диапазону.

Целесообразно курс основ квантовой электроники понимать как курс основ физики 
лазеров, дополненный изложением принципов действия наиболее интересных из 
них. При этом чрезвычайно важная область применения лазерного излучения может 
быть затронута только в виде ссылок на наиболее представительные примеры таких 
применений. 

С самого начала квантовая физика была известна своей революционностью. В 
начале 20 столетия в науке появилась квантовая теория, которой А. Эйнштейн 
воспользовался для объяснения дискретных свойств светового излучения, о 
котором на протяжении долгого времени было известно, что он распространяется 
как непрерывные волны. Первоначально эти взгляды встретили непонимание 
большинства физиков, даже Планка Эйнштейну пришлось убеждать в реальности 
квантов. В рекомендацию для избрания Эйнштейна в Прусскую Академию наук 
(1912), подписанную Планком и рядом других крупнейших физиков Германии, 
авторы включили извинение за <<легкомысленную>> веру Эйнштейна в существование 
фотонов: <<То, что он в своих рассуждениях иногда выходит за пределы цели, 
как, например, в своей гипотезе световых квантов, не следует слишком сильно 
ставить ему в упрёк. Ибо, не решившись пойти на риск, нельзя осуществить 
истинно нового, даже в самом точном естествознании>>. Постепенно, однако, 
накопились опытные данные, убедившие скептиков в дискретности электромагнитной 
энергии.

Уже в самом первом квантовом приборе механизм усиления был настолько необычным, 
что он не мог возникнуть как логическое развитие принципов электроники. Однако 
идея мазерного усиления возникла независимо в трех различных 
радиоспектроскопических лабораториях на базе исследований, единодушно 
отвергавшихся промышленными лабораториями.

В промышленных лабораториях надеялись, что новая область физики даст 
значительные практические результаты. Однако спустя несколько лет четыре 
промышленные лаборатории, первыми начавшие работу в этой области, прекратили 
ее, и исследования по радиоспектроскопии полностью сосредоточились в 
университетах.

Несмотря на отсутствие заинтересованности большинства ученых в исследованиях и 
разработках новых типов приборов, А. М. Прохоров, еще не будучи лауреатом 
Нобелевской премии, дальновидно оценил возможности и перспективы изучения 
лазеров. Прохоров некоторое время занимался СВЧ-техникой, однако затем решил 
переключиться на лазеры и заставил коллектив подчиниться своему решению, разбив 
в лаборатории приборы по старой тематике. Последовавший скандал лишил коллектив 
половины сотрудников (уволились), но оставшиеся начали заниматься новым для 
себя делом. В результате Нобелевская премия досталась именно за лазеры.

Только после появления оптического мазера, позднее названного лазером, 
достоинства изобретения стали достаточно понятны, так что ряд лабораторий 
сильно заинтересовались ими и приступили к интенсивному освоению новой области. 
С тех пор квантовая электроника развилась до своего нынешнего уровня и стала 
приносить значительные доходы.