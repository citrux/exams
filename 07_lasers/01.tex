\question{Предмет квантовой электроники. История возникновения и развития
квантовой электроники. Современные достижения и возможности квантовой
электроники}

\subquestion{Предмет квантовой электроники}

\emph{Квантовая электроника} -- это область физики, изучающая методы усиления
и генерации электромагнитного излучения путём использования эффекта
вынужденного излучения в термодинамически неравновесных квантовых системах,
свойства получаемых таким образом генераторов и усилителей и их применения.
Другими словами, это наука о мазерах и лазерах.

\subquestion{Фундаментальные физические предпосылки создания квантовой
электроники}
\begin{enumerate}
    \item[1900 г.] Планк получил верный результат для теплового излучения исходя
        из квантов излучения;
    \item[1905 г.] Эйнштейн пришёл к гипотезе световых квантов;
    \item[1913 г.] Бор связал частоту излучения с разностью уровней энергии;
    \item[1916 г.] Эйнштейн вывел формулу Планка, постулировав вынужденное
        излучение;
    \item[1924 г.] Бозе и Эйнштейн получили статистику Бозе-Эйнштейна, которой
        подчиняются фотоны;
    \item[1927 г.] Дирак строго обосновал существование вынужденного излучения
        и его свойства;
\end{enumerate}

\subquestion{История возникновения и развития}
\begin{enumerate}
    \item[1954 г.] -- были даны непосредственные теоретические основы
        квантовой электроники и создан первый прибор -- пучковый аммиачный
        мазер;
    \item[1955 г.] -- Басов и Прохоров предложили трёхуровневый метод накачки;
    \item[1960 г.] -- первый лазер (рубиновый, гелий-неоновый);
    \item[1962 г.] -- полупроводниковый лазер;
    \item[1964 г.] -- молекулярный \( \mathrm{CO_2} \) лазер;
\end{enumerate}

\subquestion{Современные достижения и возможности квантовой электроники}
На сегодняшний день лазеры позволяют получать импульсы до 10~МВт длительностью
от 10~нс до микросекунд (с помощью модуляции добротности можно получить и
фемтосекунды) интенсивностью до 1~ТВт/\(\text{см}^2\). КПД при этом может
достигать 15\%.

Области применения лазеров:
\begin{itemize}
    \item научно-исследовательская деятельность;
    \item промышленность;
    \item термические технологии (сварка, резка, скрабирование);
    \item для разделения изотопов;
    \item для оптического контроля;
    \item для связи (оптоволокно, атмосфера);
    \item в медицине;
\end{itemize}

