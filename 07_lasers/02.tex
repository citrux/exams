\question{Взаимодействие лазерного излучения с веществом.}

\begin{itemize}
    \itemsep -4pt
    \item \emph{самоиндуцированная прозрачность} --
        свет проходит в среде без потерь за счёт насыщения перехода;
    \item \emph{самофокусировка} --
        показатель преломления вещества при больших интенсивностях зависит
        от интенсивности излучения -- за счёт этого пучок фокусируется;
    \item \emph{индуцированная поляризация}.
        Двумя важнейшими свойствами индуцированной поляризации являются
        возникновение поляризации на частоте, отличной от частоты внешнего
        поля, и её нелинейный характер. Эти два главных свойства
        индуцированной поляризации лежат в основе большинства нелинейных
        оптических явлениях;
    \item \emph{сдвиг атомных уровней}
        \begin{itemize}
            \itemsep -2pt
            \item \emph{Эффект Штарка} -- смещение и расщепление электронных
                термов атомов во внешнем электрическом поле. Дипольный
                момент атома во внешнем электрическом поле приобретает
                дополнительную энергию, которая и вызывает смещение
                термов атомов.
            \item \emph{Эффект Зеемана}, обусловленный тем, что в присутствии
                магнитного поля квантовая частица, обладающая спиновым магнитным
                моментом, приобретает дополнительную энергию
                \( \D E = -\vec{\mu}\cdot\vec{B} \), пропорциональную его
                магнитному моменту \( \vec{\mu} \). Приобретённая энергия
                приводит к снятию вырождения атомных состояний по магнитному
                квантовому числу \( m_j \) и расщеплению атомных линий.
        \end{itemize}
    \item \emph{Многофотонное возбуждение} представляет собой процесс, в котором
        электрон в квантовой системе переходит из одного связанного состояния
        в другое связанное состояние в результате поглощения нескольких
        фотонов внешнего поля. При этом предполагается, что между этими двумя
        состояниями другие связанные электронные состояния отсутствуют, а
        если присутствуют, то в них не происходит реального перехода электрона
        при поглощении им фотона по причине отсутствия резонанса между
        энергией фотона и энергией перехода или запретом такого перехода;
    \item \emph{ионизация};
    \item \emph{давление света} --
        при нормальном падении светового пучка на единицу плоской поверхности
        непрозрачного тела согласно второму закону Ньютона сила давления
        света определяется соотношением \( G = dp/dt = F(1+\rho)/c \), где
        \( F \) -- интенсивность излучения, \( \rho \) -- коэффициент
        отражения, \( p \) -- импульс, \( c \) -- скорость света
    \item \emph{нагрев};
    \item \emph{оптический пробой}, возникающий в прозрачных средах~-- в газах,
        плазме, жидкостях, кристаллах и стёклах, представляет собой единое
        явление, в основе которого лежит процесс превращения прозрачной среды в
        сильно поглощающую среду под действием мощного лазерного излучения;
    \item \emph{возбуждение звука в жидкостях} --
        при взаимодействии лазерного излучения с прозрачными средами возникает
        широкий круг различных явлений, приводящих к возбуждению упругих
        колебаний среды в очень большом диапазоне частот от инфразвука до
        гиперзвука;
    \item \emph{образование плазмы} --
        при взаимодействии лазерного излучения с поверхностью твёрдого
        непрозрачного тела по мере увеличения температуры, то есть по мере
        увеличения поглощенной энергии излучения, могут происходить процессы
        нагревания, плавления и испарения твёрдого тела. Дальнейшее увеличение
        поглощенной энергии должно привести к ионизации паров и образованию
        плазмы.
\end{itemize}
