\question{Твердотельные лазеры. Лазер на ИАГ}
Отличается от лазера на неодимовом стекле лишь тем, что вместо стеклянной 
матрицы используется кристалл иттриево-алюминиевого граната. Он обладает 
большей теплопроводностью, чем стекло, однако, размеры активной среды 
получаются существено меньше. Поэтому выходная мощность такого лазера в 
одиночном импульсе (1..10 Дж) существенно меньше, чем у лазеров на стекле. 
Но в отличие от стеклянных, ИАГ-лазеры можно использовать в непрерывном и 
импульсно-периодическом режимах. Этот лазер позволяет получать длинные 
импульсы (0,5..10 мс) с частотой до 100 Гц в импульсно-периодическом режиме и 
короткие (\(\le 10\) мкс) с частотой до 100 кГц в непрерывном режиме 
модуляцией добротности.

Мощность генерации в непрерывном режиме может достигать 400 Вт. КПД лазеров на 
ИАГ составляет 2..3\%.