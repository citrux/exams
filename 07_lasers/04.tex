\question{Соотношение между коэффициентами Эйнштейна. Формула Планка.}

Рассмотрим ансамбль квантовых частиц, находящихся в термостате при 
температуре \( T \). Пусть рассматриваемая квантовая система обладает двумя 
уровнями энергии \( E_2 > E_1 \), при переходах между которыми поглощается 
или излучается квант энергии \( h\nu \). При термодинамическом равновесии 
ансамбль не теряет и не приобретает энергии. Следовательно, в единицу времени 
во всём ансамбле общее число переходов из верхнего энергетического состояния 
в нижнее должно быть равным общему числу переходов из нижнего состояние в 
верхнее. Общее число переходов определяется числом частиц на уровнях энергии 
или населенностью уровней. 

При тепловом равновесии распределение частиц по уровням подчиняется формуле 
Больцмана:
\begin{equation}
	\frac{n_2}{g_2} = \frac{n_1}{g_1}\exp
		\left[ -\frac{E_2 - E_1}{kT}\right]
	\label{eq4.1.4}
\end{equation}
где \( g_2 \) и \( g_1 \) -- кратность вырождения уровней 2 и 1, 
\( k \) -- постоянная Больцмана.

Полное число переходов \( 2 \rightarrow 1 \) равно произведению числа частиц 
\( n_2 \) в состояние \( 2 \) на вероятность перехода \( 2 \rightarrow 1 \) 
в единицу времени для одной частицы. Вероятность самопроизвольного перехода 
частицы из верхнего состояния в нижнее пропорциональна времени. За время 
\( dt \) эта вероятность составляет по предположению
\[
	dw^\text{спонт} = A_{21} dt
\]
где \( A_{21} \) -- коэффициент Эйнштейна для спонтанного излучения. Таким 
образом вероятность спонтанного испускания излучения в единицу времени 
постоянна и равна по определению соответствующему коэффициенту Эйнштейна 
\( A_{21} \):
\begin{equation}
	W_{21}^\text{спонт} = A_{21}
	\label{eq4.1.6}
\end{equation}

Частицы рассматриваемого ансамбля находятся в поле их собственного излучения, 
плотность энергии которого в единичном спектральном интервале составляет 
\( \rho_\nu \). Это поле индуцирует переходы из верхнего состояния в нижнее 
и обратно. Вероятности этих переходов пропорциональны \( \rho_\nu \). 
Комбинируя (\ref{eq4.1.6}), (\ref{eq4.1.4}), (\ref{eq3.1.3}) и (\ref{eq3.1.2}), 
можем из условия равновесия
\[
	g_1 B_{12} \rho_\nu \exp\left[ -\frac{E_1}{kT} \right] = 
	g_2 \left( B_{21}\rho_nu + A_{21} \right)
		\exp\left[ -\frac{E_2}{kT} \right]
\]
найти соотношения между коэффициентами \( A_{21}, B_{12}, B_{21} \). Это 
уравнение позволяет легко найти плотность энергии поля излучения 
рассматриваемой равновесной квантовой системы:
\[
	\rho_\nu = \frac{A_{21}}{B_{21}}
		\left[ 
			\frac{g_1 B_{12}}{g_2 B_{21}}\exp\frac{E_2 - E_1}{kT} - 1 
		\right]^{-1}
\]

Эйнштейн постулировал, что излучение, испускаемое и поглощаемое при 
равновесных переходах между энергетическими состояниями рассматриваемой 
равновесной квантовой системы, описывается формулой Планка для 
равновесного излучения абсолютно чёрного тела. Тогда для свободного 
пространства
\[
	\rho_\nu = \frac{8\pi\nu^2}{c^3}\frac{h\nu}{\exp[h\nu/kT] - 1}
\]
где \( c \) -- скорость света.

Если сопоставить две эти формулы с условием Бора, то получим что 
постулат Эйнштейна совместим с постулатом Бора. Дальнейшее сравнение приведёт 
к выводу:
\[
	g_1 B_{12} = g_2 B_{21}
\]

Это соотношение говорит о равновероятности индуцированных излучения и 
поглощения. Далее, вероятность спонтанного излучения пропорциональна 
коэффициенту Эйнштейна для индуцированного излучения:
\[
	A_{21} = \frac{8\pi\nu^2}{c^3}h\nu B_{21}
\]

Таким образом, для описания термодинамического равновесия между системой 
квантовых частиц и полем её излучения Эйнштейн ввёл индуцированные полем 
равновероятные переходы из верхнего состояния в нижнее и из нижнего в верхнее. 
Требование равновесия приводит к такому соотношению между спонтанными и 
индуцированным излучениями, при котором для одной частицы вероятность 
переходов в единицу времени с испусканием квантов излучения равна
\[
	W^\text{изл} = \left( \frac{8\pi\nu^2}{c^3} + \rho_\nu \right)B_{21}
\]