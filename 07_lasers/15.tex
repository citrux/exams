\question{Характеристики и особенности лазерного излучения}

\subquestion{Энергетические характеристики}

Основными энергетическими характеристиками лазеров являются его
мощность~\( P \) и КПД~\( \eta \).

Мощность лазерного излучения можно достаточно точно и просто оценить, зная
характеристики среды (\( K_0 \) и \( I_s \)) и резонатора (\( L_p \), \( \xi \),
\( \chi \)) с помощью соотношения \eqref{eq_1.91}. Если площадь поперечного
сечения активной среды \( s \), то мощность
\[
  P = Is,
\]
где \( I \)~-- средняя интенсивность излучения, прошедшего через выходное
зеркало резонатора.

Для лазеров, работающих в импульсном и импульсно-периодическом режимах
генерации, усредненная~\( \average{P} \) и импульсная~\( P_\text{и} \) мощности
определяются через энергию импульса~\( \eps_\text{и} \), его
длительность~\( \tau_\text{и} \) и частоту следования
импульсов~\( f_\text{и} \):
\[
  \average{P} = \eps_\text{и} f_\text{и}, \quad
    P_\text{и} = \eps_\text{и} / \tau_\text{и}.
\]

Общий энергетический КПД лазера можно представить в виде
\[
  \eta = W_\text{изл.} / W_\text{затр.} = \eta_\text{кв} \eta_\text{в}
    \eta_\text{р} \eta_\text{с.о},
\]
где \( W_\text{изл.} \)~-- энергия излучения, \( W_\text{затр.} \)~--
потребляемая энергия, \( \eta_\text{кв} \)~-- квантовый КПД,
\( \eta_\text{в} \)~-- КПД возбуждения, \( \eta_\text{р} \)~-- КПД резонатора,
\( \eta_\text{с.о} \)~-- КПД систем обслуживания.

Квантовый КПД~-- это отношение энергии генерируемого кванта~\( \eps_q \) к
энергии возбуждения наиболее высокого уровня~\( \eps_l \), участвующего в
процессе создания инверсной заселенности:
\[
  \eta_\text{кв} = \eps_q / \eps_l.
\]
Значение квантового КПД зависит от конкретной схемы уровней и изменяется от
\( 10^{-3} \) до 0,9.

КПД резонатора определяется отношением вероятности процесса вынужденного
излучения к сумме вероятностей всех процессов тушения верхнего лазерного уровня
и нерезонансных потерь квантов в резонаторе. Ее можно оценить так:
\[
  \eta_\text{р} \simeq I / (I + I_s) = 1 - K_\text{п} / K_0.
\]
Обычно значение КПД резонатора составляет 0,3..0,7.

КПД возбуждения активной среды равен отношению
\[
  \eta_\text{в} = \eps_2 M_2 V_a / W^*,
\]
где \( \eps_2 \)~-- энергия возбуждения верхнего лазерного уровня, \( M_2 \)~--
скорость его заселения, \( V_a \)~-- объем возбуждаемой активной среды,
\( W^* \)~-- полная мощность, затраченная на возбуждение среды. Величина
\( \eta_\text{в} \) зависит от способа и конкретной схемы возбуждения, обычно
составляет 0,1..0,7.

КПД систем обслуживания учитывает необходимые для работы лазера энергетические
затраты. Обычно он учитывает КПД источников питания, затраты на прокачку рабочей
смеси, охлаждение и т.~п. Обычно составляет 0,5..0,9.

\subquestion{Особенности лазерного излучения}

\begin{itemize}
  \item \emph{Монохроматичность} лазерного излучения характеризует способность
    лазеров излучать в узком диапазоне длин волн и определяется величиной
    \( \D\nu / \nu_0 \).

  \item \emph{Когерентность} лазерного излучения характеризует связанные с ним
    колебания электромагнитного поля: сдвиг фазы для двух произвольных точек
    остается постоянным во времени.
  
  \item \emph{Поляризация} характеризует ориентацию вектора электрического поля
    в электромагнитной волне. Для создания определенного типа поляризации в
    резонатор вводят какой-либо поляризующий элемент.
    
  \item \emph{Высокая направленность} характеризует угловую расходимость
    лазерного пучка в пространстве. Она определяется полным углом расходимости
    \( 2\Theta \), в пределах которого содержится определенная доля мощности
    лазерного излучения.
\end{itemize}
