\input{../.preambles/00-lectures}
\input{../.preambles/10-russian}
\input{../.preambles/20-math}

\renewcommand{\div}{\mathrm{div}\,}
\newcommand{\grad}{\mathrm{grad}\,}
\newcommand{\header}[1]{\vspace*{.3em}\emph{#1}\vspace*{.2em}}
\newcommand{\D}{\,\Delta}
\newcommand{\pnder}[3]{\frac{\partial^{#1} #2}{\partial #3^{#1}}}
\renewcommand{\kappa}{\varkappa}
\newcommand{\ds}{\displaystyle}
\newcommand{\e}{\mathrm{e}}

\begin{document}
\emph{1. Тепловое излучение. Закон Релея-Джинса. Гипотеза Планка. 
Кванты излучения. Функция Кирхгофа. Законы Стефана-Больцмана и 
смещения Вина.}

\newpage % ---------------------------------------------------------------------

\emph{2. Фотоны. Фотоэффект. Закон Эйнштейна. Опыт Боте. Тормозное 
рентгеновское излучение. Эффект Комптона. Давления света.}

\newpage % ---------------------------------------------------------------------

\emph{3. Ядерная модель атома и опыты Резерфорда. Теория рассеяния 
альфа-частиц. Дифференциальное эффективное сечение рассеяния.}

\newpage % ---------------------------------------------------------------------

\emph{4. Спектральные закономерности. Комбинационный принцип. Теория 
атома Бора. Спектр водорода. Недостатки теории Бора. Изотопический 
сдвиг атомных уровней. Опыты Франка и Герца.}

\newpage % ---------------------------------------------------------------------

\emph{5. Гипотеза де Бройля и её экспериментальное подтверждение. 
Корпускулярно-волновой дуализм свойств вещества. Опыты Девиссона-
Джермера и Томсона. Волновая функция. Статистическая интерпретация 
волн де Бройля и волновой функции. Соотношение неопределенностей.}

\newpage % ---------------------------------------------------------------------

\emph{6. Волновое уравнение Шрёдингера, физический смысл его решений. 
Частица в потенциальной яме. Линейный гармонический осциллятор.}

\newpage % ---------------------------------------------------------------------

\emph{7. Отражение и прохождение частиц через потенциальный барьер.}

\newpage % ---------------------------------------------------------------------

\emph{8. Операторы физических величин. Квантование момента импульса.}

\newpage % ---------------------------------------------------------------------

\emph{9. Атом водорода по квантовой механике. Уровни и спектры атомов 
щелочных металлов.}

\newpage % ---------------------------------------------------------------------

\emph{10. Магнетизм атомов. Опыты Штерна и Герлаха. Магнитно-механические 
эффекты. Спин электрона. Квантовые числа электрона и тонкая структура 
спектральных термов. Правила отбора. Понятие об уравнении Дирака.}

\newpage % ---------------------------------------------------------------------

\emph{11. Многоэлектронные атомы. Векторная модель многоэлектронного 
атома. Магнитный момент многоэлектронного атома. Фактор Ланде.}

\newpage % ---------------------------------------------------------------------

\emph{12. Простой и сложный эффекты Зеемана. Эффект Пашена -- Бака. 
Электронный парамагнитный резонанс. Сверхтонкая структура спектральных 
линий.}

\newpage % ---------------------------------------------------------------------

\emph{13. Принцип тождественности одинаковых частиц. Фермионы и бозоны. 
Принцип Паули. Строение электронных оболочек. Объяснение периодической 
системы элементов Менделеева. Правила Хунда.}

\newpage % ---------------------------------------------------------------------

\emph{14. Переходы внутренних электронов в атомах. Рентгеновские спектры. 
Закон Мозли. Эффект Оже.}

\newpage % ---------------------------------------------------------------------

\emph{15. Ковалентная и ионная связи. Молекула водорода. Адиабатическое 
приближение. Теория Гайтлера -- Лондона. Многоатомные молекулы. 
Электронные состояния. Принцип Франка-Кондона. Колебательные состояния. 
Вращательные состояния.}

\newpage % ---------------------------------------------------------------------

\emph{16. Молекулярные спектры. Комбинационное рассеяние света.}

\newpage % ---------------------------------------------------------------------

\emph{17. Спонтанное и вынужденное излучение. Коэффициенты Эйнштейна. 
Ширина уровней и спектральных линий. Метастабильные состояния. 
Коэффициент поглощения среды. Лазеры.}

\newpage % ---------------------------------------------------------------------

\emph{18. Система тождественных частиц. Границы применимости классической 
статистики и квантовая статистика. Каноническое распределение. 
Идеальный газ в квантовой статистике. Числа заполнения состояний. 
Статистика Ферми-Дирака. Энергия Ферми. Вырожденный электронный газ. 
Статистика Бозе-Эйнштейна.}
\end{document}
