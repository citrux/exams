\input{../.preambles/00-lectures}
\input{../.preambles/10-russian}
\input{../.preambles/20-math}

\renewcommand{\div}{\mathrm{div}\,}
\newcommand{\grad}{\mathrm{grad}\,}
\newcommand{\header}[1]{\vspace*{.3em}\emph{#1}\vspace*{.2em}}
\newcommand{\D}{\,\Delta}
\newcommand{\pnder}[3]{\frac{\partial^{#1} #2}{\partial #3^{#1}}}
\renewcommand{\kappa}{\varkappa}
\newcommand{\ds}{\displaystyle}
\newcommand{\e}{\mathrm{e}}

\begin{document}
\emph{1. Понятие о ДУЧП, примеры. Методы решения ДУЧП. Типы ДУЧП.}

Уравнение, связывающее неизвестную функцию \( u(x_1, \ldots, x_n \), независимые
переменные \( x_1 \), \ldots, \( x_n \) и частные производные от неизвестной
функции, называется дифференциальным уравнением с частными производными. Оно
имеет вид:
\[
    F\left(x_1, x_2, \ldots, x_n, u, \pder{u}{x_1}, \pder{u}{x_2}, \ldots,
    \pder{u}{x_n}, \ppder{u}{x_1}, \pcder{u}{x_1}{x_2}, \ppder{u}{x_2}, \ldots,
    \pnder{n}{u}{x_m}\right) = 0,
\]
где \( F \) -- заданная функция своих аргументов.

Порядок старшой производной, входящей в уравнение, называется порядком ДУЧП.

\begin{table}[h!]
    \begin{tabular}{m{.48\textwidth}cm{.48\textwidth}}
        \header{Типы ДУЧП}

        Существует 6 способов классификации ДУЧП:
        \begin{enumerate}\itemsep-.4em
            \item порядок уравнения;
            \item число переменных;
            \item линейность (функция и все её производные входят в уравнение
            линейным образом);
            \item однородность;
            \item вид коэффициентов (постоянные либо переменные);
            \item три основных вида линейные ДУЧП II:
            \begin{itemize}\itemsep-.4em
                \item параболические;
                \item эллиптические;
                \item гиперболические.
            \end{itemize}

            Также при переменных коэффициентах возможен <<смешанный>> тип
            уравнения.
        \end{enumerate}
        % -----------------------
        & \hfill &
        % -----------------------
        \header{Методы решения ДУЧП}

        \begin{enumerate}
            \item Метод разделения переменных (метод Фурье);
            \item метод интегральных преобразований;
            \item метод преобразования координат;
            \item метод преобразования зависимой переменной;
            \item численные методы;
            \item методы теории возмущений;
            \item метод функций Грина;
            \item метод интегральных уравнений;
            \item вариационные методы;
            \item метод разложения по собственным функциям.
        \end{enumerate}
    \end{tabular}
\end{table}

\header{Примеры}

\begin{enumerate}
    \item Волновое уравнение:
    \[
        \ppder{u}{t} = c^2\left(\ppder{u}{x} + \ppder{u}{y} +
        \ppder{u}{z}\right);
    \]

    \item уравнение теплопроводности:
    \[
        \pder{u}{t} = \alpha^2\left(\ppder{u}{x} + \ppder{u}{y} +
        \ppder{u}{z}\right);
    \]

    \item уравнение Лапласа:
    \[
        \ppder{u}{x} + \ppder{u}{y} + \ppder{u}{z} = 0.
    \]
\end{enumerate}

\newpage % ---------------------------------------------------------------------

\emph{2. Математическая модель теплопроводности. Некоторые уравнения
диффузионного типа.}

\header{Математическая модель теплопроводности} представляет собой систему четырех
уравнений:
\[
    \left. \begin{array}{rl}
        \text{ДУЧП:} & \ds \pder{u}{t} = \alpha^2\ppder{u}{x}; 
        \vspace*{.4em} \\
        \text{ГУ:} & \left\{ \begin{array}{l}
            u(0, t) = f_1(t), \\
            u(l, t) = f_2(t); 
        \end{array} \right. \\
        \text{НУ:} & u(x, 0) = \phi(x).
    \end{array} \right\}
\]

\header{Некоторые уравнения диффузионного типа:}
\begin{enumerate}
    \item теплообмен через боковую поверхность пропорционально разности
    температур:
    \[
        \pder{u}{t} = \alpha^2\ppder{u}{x} - \beta(u - u_0);
    \]
    
    \item внутренний источник тепла:
    \[
        \pder{u}{t} = \alpha^2\ppder{u}{x} + f(x, t);
    \]
    
    \item уравнение конвективной диффузии:
    \[
        \pder{u}{t} = \alpha^2 \ppder{u}{x} - v\pder{u}{x}.
    \]
\end{enumerate}

\newpage % ---------------------------------------------------------------------

\emph{3. Граничные условия в задачах диффузионного типа.}

В задачах диффузионного типа используются ГУ трёх основных видов:
\begin{enumerate}
    \item ГУ I рода -- заданы значения функции на границе области:
    \[
        \left\{ \begin{array}{l}
            u(0, t) = g_1(t); \\
            u(l, t) = g_2(t).
        \end{array} \right.
    \]
    
    \item ГУ II рода -- задана температура окружающей среды \( g_1(t) \) и
    \( g_2(t) \):
    \[
        \left\{ \begin{array}{l}
            u'_x(0, t) = \cfrac{h}{\kappa}\Big(u(0, t) - g_1(t)\Big); \\
            u'_x(l, t) = \cfrac{h}{\kappa}\Big(u(l, t) - g_2(t)\Big).
        \end{array} \right.
    \]
    
    \item ГУ III рода -- задан поток через границу области (градиент на границе
    области):
    \[
        \left\{ \begin{array}{l}
            u'_x(0, t) = g_1(t); \\
            u'_x(l, t) = g_2(t).
        \end{array} \right.
    \]
\end{enumerate}

\newpage % ---------------------------------------------------------------------

\emph{4. Вывод уравнения теплопроводности.}

Запишем уравнение диффузии: \( \vec{j} = -\kappa\cdot\grad u \), где \( u \)
имеет смысл температуры. Рассмотрим малую окрестность точки наблюдения.
Внутренняя энергия, заключенная в её объеме \( V \) может измениться только за
счет теплообмена, то есть:
\[
    \der{W}{t} = -\oiint\limits_S \vec{j}\cdot\d\vec{S}.
\]

Для интеграла в правой части по теореме Остроградского имеем
    \( \ds \oiint\limits_S \vec{j}\cdot\d\vec{S} = \iiint\limits_V
    \div\vec{j}\d V \),
а в левой: \( \ds W = \iiint_V w\cdot\d V \). Таким образом, имеем:
\[
    \der{}{t}\iiint\limits_V w\d V = \iiint\limits_V \kappa\, \div\grad u\d V.
\]
В силу произвольности выбора элемента объема имеем:
\[
   \pder{w}{t} = \kappa\cdot\Delta u,
\]
где \( \Delta \) -- оператор Лапласа; \( w = \rho\cdot c\cdot u\), где
\( \rho \) -- плотность вещества, а \( c \) -- его удельная теплоемкость. Получаем:
\[
    \rho e\pder{u}{t} = \kappa\Delta u, \quad \text{или} \quad
    \pder{u}{t} = \frac{\kappa}{\rho c}\Delta u.
\]

Обозначая \( \cfrac{\kappa}{\rho c} = \alpha^2 \), получаем уравнение
теплопроводности: \( \ds \pder{u}{t} = \alpha^2\Delta u \), переходящее в
одномерном случае в
\[
    \pder{u}{t} = \alpha^2\ppder{u}{x},
\]
соответствующее распространению тепла в одномерном стержне с теплоизолированной
боковой поверхностью.

В случае, когда потери через боковую поверхность пропорциональны разнице
температур, уравнение принимает вид:
\[
    \pder{u}{t} = \alpha^2\ppder{u}{x} - \beta(u - u_0),
\]
где \( u_0 \) -- температура окружающей среды.

\newpage % ---------------------------------------------------------------------

\emph{5. Метод разделения переменных и его применение в диффузионных задачах.}

Суть метода разделения переменных состоит в том, что частное решение задачи
ищется в виде композиции функций каждой их переменных. Рассмотрим его применение
к задаче диффузионного типа:

\begin{minipage}{.4\textwidth}
\flushleft
\[
    \left. \begin{array}{rl}
        \text{ДУЧП:} & \ds \pder{u}{t} = \alpha^2\ppder{u}{x}; 
        \vspace*{.4em} \\
        \text{ГУ:} & \left\{ \begin{array}{l}
            u(0, t) = f_1(t), \\
            u(l, t) = f_2(t); 
        \end{array} \right. \\
        \text{НУ:} & u(x, 0) = \phi(x).
    \end{array} \right\}
\]
\end{minipage}
\hfill
\begin{minipage}{.56\textwidth}
    Так как в ДУЧП входят две независимые переменные \( x \) и \( t \), то
    частное решение будем искать в виде \( u = X(x)T(t) \).
    
    Подставим его в ДУЧП:
    \[
        X\cdot T' = \alpha^2 T\cdot X''.
    \]
\end{minipage}
    
    Разделив на \( \alpha^2 XT \), имеем:
    \[
        \frac{T'}{\alpha^2 T} = \frac{X''}{X} = k,
    \]
    где \( k \) -- константа разделения.

Получаем два ОДУ: \( X'' - kX = 0 \) и \( T' - k\alpha^2T = 0 \). Из
функциональных соображений решение должно быть ограничено. Это возможно, если
\( k = -\lambda^2 \), где \( \lambda \in (0, +\infty) \). Тогда ОДУ:
\[
    X'' + \lambda^2X = 0, \quad T' + \lambda^2\alpha^2T = 0,
\]
и их решения:
\[
    X = A\sin\lambda x + B\cos\lambda x, \quad T = \e^{-\lambda^2\alpha^2 t}.
\]

Получаем, что
\[
    u = (A\sin\lambda x + B\cos\lambda x)\e^{-\lambda^2\alpha^2 t}.
\]

Найденное решение должно удовлетворять граничным условиям:
\begin{align*}
    & u(0, t) = 0 \Rightarrow A\sin0 + B\cos0 = 0 \Rightarrow B = 0; \\
    & u(l, t) = 0 \Rightarrow A\sin\lambda l = 0 \Rightarrow \lambda l = \pi n
    \Rightarrow \lambda_n = \frac{\pi n}{l}.
\end{align*}

Получаем семейство функций:
\[
    u_n = A_n\sin\frac{\pi n}{l}x \cdot \e^{-\left(\frac{\alpha\pi n}
    {l}\right)^2t},
\]
а общее решение краевой задачи получаем в виде:
\[
    u = \sum\limits_{n = 1}^\infty A_n\sin\frac{\pi n}{l}x \cdot
    \e^{-\left(\frac{\alpha\pi n}{l}\right)^2t}.
\]

Это решение должно удовлетворять начальному условию:
\[
    \phi(x) = \sum\limits_{n = 1}^\infty A_n\sin\frac{\pi n}{l}x \Rightarrow
    A_n = \frac{2}{l}\int\limits_0^l \phi(x)\sin\frac{\pi n}{l}x.
\]

\newpage % ---------------------------------------------------------------------

\emph{6. Преобразование неоднородных ГУ в однородные. Преобразование зависящих
от времени ГУ в нулевые.}

\newpage % ---------------------------------------------------------------------

\emph{7. Задача теплопроводности с производной в ГУ. Задача Штурма-Лиувилля и её
свойства.}

\newpage % ---------------------------------------------------------------------

\emph{8. Преобразование задачи с теплообменом через боковую поверхность к задаче
с теплоизолированной боковой поверхностью.}

\newpage % ---------------------------------------------------------------------

\emph{9. Решение неоднородных ДУЧП методом разложения по собственным функциям.}

\newpage % ---------------------------------------------------------------------

\emph{10. Интегральные преобразования. Спектр функции. Синус- и
косинус-преобразования производных.}

\newpage % ---------------------------------------------------------------------

\emph{11. Решение диффузионной задачи на полупрямой методом
синус-преобразования. Интерпретация решения.}

\newpage % ---------------------------------------------------------------------

\emph{12. Ряд Фурье и его коэффициенты. Теорема Дирихле. Дискретный частотный
спектр периодической функции. Преобразование Фурье.}

\newpage % ---------------------------------------------------------------------

\emph{13. Свойства преобразования Фурье. Решение задачи о распространении тепла
в бесконечном стержне с заданной начальной температурой.}

\newpage % ---------------------------------------------------------------------

\emph{14. Преобразование Лапласа и его свойства. Решение задачи о
теплопроводности в полубесконечной среде.}

\newpage % ---------------------------------------------------------------------

\emph{15. Принцип Дюамеля.}

\newpage % ---------------------------------------------------------------------

\emph{16. Решение задачи конвективного переноса методом преобразования Лапласа.}

\newpage % ---------------------------------------------------------------------

\emph{17. Уравнение колебаний струны и его интуитивная интерпретация.
Замечания.}

\newpage % ---------------------------------------------------------------------

\emph{18. Решение одномерного волнового уравнения с помощью формулы Даламбера.
Примеры применения формулы Даламбера в некоторых конкретных задачах.}

\newpage % ---------------------------------------------------------------------

\emph{19. Пространственно-временная интерпретация формулы Даламбера. Решение
задачи для полубесконечной струны.}

\newpage % ---------------------------------------------------------------------

\emph{20. Волновое уравнение и три типа граничных условий.}

\newpage % ---------------------------------------------------------------------

\emph{21. Решение задачи о колебаниях ограниченной струны методом разделения
переменных.}

\newpage % ---------------------------------------------------------------------

\emph{22. Колебания балки (ДУЧП IV).}

\newpage % ---------------------------------------------------------------------

\emph{23. Переход к безразмерным переменным на примере диффузионной задачи.
Пример преобразования гиперболической задачи к безразмерному виду.}

\newpage % ---------------------------------------------------------------------

\emph{24. Классификация ДУЧП II. Приведение к каноническому виду.}

\newpage % ---------------------------------------------------------------------

\emph{25. Волновое уравнение в свободном пространстве (2- и 3-мерные задачи).}

\newpage % ---------------------------------------------------------------------

\emph{26. Конечные синус- и косинус-преобразования Фурье. Решение краевой задачи
с неоднородным волновым уравнением.}

\newpage % ---------------------------------------------------------------------

\emph{27. Принцип суперпозиции. Разложение смешанной задачи на две более
простые. Разделение переменных и интегральные преобразования как проявление
принципа суперпозиции.}

\newpage % ---------------------------------------------------------------------

\emph{28. Уравнения первого порядка (метод характеристик). Общая стратегия
решения ДУЧП I.}

\newpage % ---------------------------------------------------------------------

\emph{29. Нелинейные ДУЧП I. Вывод и применение закона сохранения к задаче о
дорожном движении.}

\newpage % ---------------------------------------------------------------------

\emph{30. Системы ДУЧП. Решение простейшей линейной системы.}

\newpage % ---------------------------------------------------------------------

\emph{31. Колебания мембраны. Решение задачи на собственные значения для
уравнения Гельмгольца.}

\end{document}
