\chapter{Преобразование неоднородных ГУ в однородные. Преобразование зависящих
от времени ГУ в нулевые.}

\begin{minipage}{.45\textwidth}
    \header{Преобразование неоднородных ГУ в однородные}
    
    Рассмотрим задачу о распространении тепла в стержне, концы которого
    поддерживаются при постоянной температуре:
    \[
        \left. \begin{array}{rl}
            \text{ДУЧП:} & \ds \pder{u}{t} = \alpha^2\ppder{u}{x}; 
            \vspace*{.4em} \\
            \text{ГУ:} & \left\{ \begin{array}{l}
                u(0, t) = k_1, \\
                u(l, t) = k_2; 
            \end{array} \right. \\
            \text{НУ:} & u(x, 0) = \phi(x).
        \end{array} \right\}
    \]
    
    Произведем замену
    \[
        u(x, t) = \frac{xk_2 + (l-x)k_1}{l} + \bar{u}(x, t).
    \]
    Тогда задача с неоднородными ГУ преобразуется в задачу с
    однородными ГУ относительно функции \( \bar{u} \):
    \[
        \left. \begin{array}{rl}
            \text{ДУЧП:} & \ds \pder{\bar{u}}{t} = \alpha^2\ppder{\bar{u}}{x};
            \vspace*{.4em} \\
            \text{ГУ:} & \left\{ \begin{array}{l}
                \bar{u}(0, t) = 0, \\
                \bar{u}(l, t) = 0; 
            \end{array} \right. \\
            \text{НУ:} & \bar{u}(x, 0) = \bar{\phi}(x),
        \end{array} \right\}
    \]
    где \( \ds \bar{\phi}(x) = \phi(x) - \frac{xk_2 + (l-x)k_1}{l} \).
\end{minipage}
\hfill
\begin{minipage}{.48\textwidth}
    \header{Преобразование зависящих от времени ГУ в нулевые}
    \[
        \left. \begin{array}{rl}
            \text{ДУЧП:} & \ds \pder{u}{t} = \alpha^2\ppder{u}{x}; 
            \vspace*{.4em} \\
            \text{ГУ:} & \left\{ \begin{array}{l}
                u(0, t) = g_1(t), \\
                u'_x(l, t) + hu(l, t) = g_2(t); 
            \end{array} \right. \\
            \text{НУ:} & u(x, 0) = \phi(x).
        \end{array} \right\}
    \]
    
    Произведем замену \( u(x, t) = s(x, t) + \bar{u}(x,t) \), где \( s(x, t) =
    a(t)\cdot\left(1 - \cfrac{x}{l}\right) + b(t)\cdot\cfrac{x}{l} \), причем
    \( s(x, t) \) должна удовлетворять ГУ:
    \[
        \begin{array}{l}
            s(0, t) = g_1(t), \\
            s'_x(l, t) + hs(l, t) = g_2(t),
        \end{array}
        \Rightarrow
        \begin{array}{l}
            a(t) = g_1(t), \\
            b(t) = \cfrac{g_1(t) + lg_2(t)}{1 + lh}.
        \end{array}
    \]
    Следовательно, с учетом подстановки имеем преобразованную задачу с нулевыми
    ГУ:
    \[
        \left. \begin{array}{rl}
            \text{ДУЧП:} & \ds \pder{\bar{u}}{t} = \alpha^2\ppder{\bar{u}}{x}
            - \pder{s}{t};
            \vspace*{.4em} \\
            \text{ГУ:} & \left\{ \begin{array}{l}
                \bar{u}(0, t) = 0, \\
                \bar{u}'_x(l, t) + h\bar{u}(l, t) = 0; 
            \end{array} \right. \\
            \text{НУ:} & \bar{u}(x, 0) = \phi(x) - s(x, 0),
        \end{array} \right\}
    \]
\end{minipage}

\newpage
