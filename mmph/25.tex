\chapter{Волновое уравнение в свободном пространстве (2- и 3-мерные задачи).}

\section{Волны в трёхмерном пространстве}
Рассмотрим задачу:

\begin{minipage}{.33\textwidth}
    \begin{align*}
        & \ppder{u}{t} =
        c^2\left(\ppder{u}{x} + \ppder{u}{y} + \ppder{u}{z}\right), \\
        & \left\{ \begin{array}{l}
            u(\vec{r}, 0) = \phi(\vec{r}), \\
            \ds \pder{u}{t}(\vec{r}, 0) = \psi(\vec{r}).
        \end{array} \right.
    \end{align*}
\end{minipage}
\hfill
\begin{minipage}{.6\textwidth}
    Примем, для простоты, \( \phi(\vec{r}) \equiv 0 \).
    
    Оказывается, что решение данной задачи представимо в виде
    \[
        u(\vec{r}, t) = t\midnum{\psi},
    \]
    где \( \midnum{\psi} \) -- среднее значение начального распределения
    \( \psi \) по сфере радиуса \( ct \) с центром в точке \( (x, y, z) \):
\end{minipage}
\[
    \midnum{\psi} = \frac{1}{4\pi c^2t^2} \int\limits_0^\pi
    \int\limits_0^{2\pi} \psi(x + ct\sin\theta\cos\alpha, y + ct\sin\theta
    \sin\alpha, z + ct\cos\theta)c^2t^2\sin\theta\d\theta\d\alpha.
\]

Для получения полного решения рассмотрим теперь задачу с
\( \phi(\vec{r}) \neq 0 \) и \( \psi(\vec{r}) \equiv 0 \). По теореме Стокса,
решение этой задачи может быть найдено дифференцированием решения предыдущей
задачи: \( \ds u(\vec{r}, t) = \pder{}{t}\Bigl(t\cdot\midnum{\phi}\Bigr) \).

Тогда решение исходной задачи может быть записано в виде:
\( \ds u(\vec{r}, t) = t\midnum{\psi} + \pder{}{t}\Bigl(t\midnum{\phi}\Bigr) \).

Полученная формула называется формулой Пуассона и является обобщением формулы
д'Аламбера на трёхмерный случай. Физической интерпретацией формулы Пуассона
является принцип Гюйгенса: решение, распространяющееся из области начального
возмущения, всегда имеет резко очерченный передний фронт, а задний фронт резко
очерчен только в пространствах нечетной размерности: 3, 5, 7, \ldots

\section{Волны в двумерном пространстве}

Рассмотрим задачу:
\begin{align*}
    & \ppder{u}{t} =
    c^2\left(\ppder{u}{x} + \ppder{u}{y} + \ppder{u}{z}\right), \\
    & \left\{ \begin{array}{l}
        u(\vec{r}, 0) = \phi(\vec{r}), \\
        \ds \pder{u}{t}(\vec{r}, 0) = \psi(\vec{r}).
    \end{array} \right.
\end{align*}

Применяя формулу Пуассона в двумерном случае, получим:
\[
    u(x, y, t) = \frac{1}{2\pi c}\left[ \int\limits_0^{2\pi}\int\limits_0^{ct}
    \frac{\psi(r, \alpha)}{\sqrt{c^2t^2 - r^2}}r\d r\d\alpha + \pder{}{t}\left(
    \frac{1}{2\pi c}\int\limits_0^{2\pi}\int\limits_0^{ct}\frac{\phi(r, \alpha)}
    {\sqrt{c^2t^2 - r^2}}r\d r\d\alpha\right)\right].
\]
\newpage
