\chapter{Задача теплопроводности с производной в ГУ. Задача Штурма-Лиувилля и её
свойства.}

\section{Задача с производной в ГУ}
Задача с производной в ГУ соответствует теплообмену с окружающей средой на одном
из концов:
\[
    \left. \begin{array}{rl}
        \text{ДУЧП:} & \ds \pder{u}{t} = \alpha^2\ppder{u}{x}; 
        \vspace*{.4em} \\
        \text{ГУ:} & \left\{ \begin{array}{l}
            u(0, t) = 0, \\
            u'_x(1, t) + hu(t) = 0; 
        \end{array} \right. \\
        \text{НУ:} & u(x, 0) = \phi(x).
    \end{array} \right\}
\]

Разобъем процесс решения на несколько шагов:

\begin{enumerate}
    \item \emph{Преобразование уравнения с частными производными в два ОДУ}.
    
    Подстановка \( u(x, t) = X(x)T(t) \) в уравнение дает:
    \[
        XT' = \alpha^2 X''T.
    \]
    После деления обоих частей этого уравнения на \( \alpha^2 XT \) получаем:
    \[
        \frac{T'}{\alpha T} = \frac{X''}{X}.
    \]
    Поскольку левая часть последнего уравнения зависит только от времени, а
    правая часть зависит только от \( x \) (и так как \( x \) и \( t \)
    независимы), обе части уравнения должны быть константами. Приравнивая обе
    части одной и той же константе \( \mu \), получаем:
    \[
        T' - \mu\alpha^2 T = 0, \quad X'' - \mu X = 0.
    \]

    \item \emph{Определение константы разделения}.
    
    Константа \( \mu \) не должна быть положительной, так как тогда функция
    \( T(t) \) будет экспоненциально расти, что повлечет бесконечный рост
    температуры \( u = XT \). Так же \( \mu \) не должна равняться нулю, так как в этом
    случае получится только нулевое решение: \( u(x, t) \equiv 0 \).
    
    Если положить \( \mu = -\lambda^2 \) и переписать уравнения в виде:
    \[
        T' + \lambda^2\alpha^2 T = 0, \quad X'' + \lambda^2 X = 0.
    \]
    
    Общие решения этих уравнений:
    \[
        X = A\sin\lambda x + B\cos\lambda x, \quad T = \e^{-\lambda^2\alpha^2 t}.
    \]

    Получаем, что
    \[
        u = (A\sin\lambda x + B\cos\lambda x)\e^{-\lambda^2\alpha^2 t}.
    \]

    Найденное решение должно удовлетворять граничным условиям:
    \begin{align*}
        & u(0, t) = 0 \Rightarrow A\sin0 + B\cos0 = 0 \Rightarrow B = 0; \\
        & u'_x(1, t) + hu(1, t) = 0 \Rightarrow A\lambda\cos\lambda +
        hA\sin\lambda = 0 \Rightarrow \tg\lambda = -\lambda/h.
    \end{align*}
    
    Значения \( \lambda_1 \), \( \lambda_2 \), \ldots принято называть
    собственными значениями краевой задачи
    \begin{align*}
        & X'' + \lambda^2 X = 0; \\
        & \left\{ \begin{array}{l}
            X(0) = 0, \\
            X'(1) + hX(1) = 0.
        \end{array} \right.
    \end{align*}
    Решения этой задачи, соответствующие собственным значениям \( \lambda_n \),
    называются собственными функциями \( X_n(x) \):
    \[
        X_n(x) = \sin(\lambda_nx).
    \]
    
    \item \emph{Нахождение фундаментальных решений}.
    
    Имеем бесконечный набор фундаментальных решений:
    \[
        u_n(x, t) = X_n(x)T_n(t) = \e^{-(\lambda_n\alpha)^2t}\sin(\lambda_nx),
    \]
    каждое из которых удовлетворяет ДУ и ГУ, а общее решение задачи получаем в
    виде:
    \[
        u = \sum\limits_{n=1}^\infty A_nX_n(x)T_n(t) = \sum\limits_{n=1}^\infty
        A_n\e^{-(\lambda_n\alpha)^2t}\sin(\lambda_nx).
    \]
    
    Коэффициенты \( A_n \) должны определяться из условия
    \[
        u(x, 0) = x = \sum\limits_{n=1}^\infty A_n\sin(\lambda_nx).
    \]
    
    \item \emph{Разложение начального условия в ряд по собственным функциям}.
    
    Умножим обе части последнего соотношения на \( \sin(\lambda_mx) \) и
    проинтегрируем от 0 до 1. Разрешая относительно \( A_m \) и заменяя \( m \)
    на \( n \), получаем искомый результат:
    \[
        A_n = \frac{2\lambda_n}{\lambda_n - \sin\lambda_n\cos\lambda_n}
        \int\limits_0^1 \xi\sin(\lambda_n\xi)\d\xi.
    \]
\end{enumerate}

\section{Задача Штурма-Лиувилля:}
\begin{multicols}{2}
\vspace*{-4em}
\begin{align*}
    & [p(x)\cdot y']' - q(x)\cdot y + \lambda r(x)\cdot y = 0; \\
    & \left\{ \begin{array}{l}
        \alpha_1 y(0) + \beta_1 y'(0) = 0, \\
        \alpha_2 y(1) + \beta_2 y'(1) = 0.
    \end{array} \right.
\end{align*}
\emph{Свойства} задачи Штурма-Лиувилля:
\begin{enumerate}
    \item Существует бесконечная последовательность собственных значений,
    удовлетворяющих неравенствам
    \[
        \lambda_1 < \lambda_2 < \ldots < \lambda_n < \ldots \to \infty.
    \]
    \item Каждому собственному значению \( \lambda_n \) соответствует
    единственная (с точностью до постоянного множителя) ненулевая собственная
    функция \( y_n(x) \). 
    \item Если \( y_n(x) \) и \( y_m(x) \) -- две различные собственные функции,
    соответствующие различным собственным значениям \( \lambda_n \) и
    \( \lambda_m \), то они ортогональны с весом \( r(x) \) на отрезке
    \( [0, 1] \), то есть удовлетворяют условию:
    \[
        \int\limits_0^1 r(x)y_n(x)y_m(x)\d x = 0.
    \]
\end{enumerate}
\end{multicols}

\newpage
