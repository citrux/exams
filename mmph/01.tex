\section{Понятие о ДУЧП, примеры. Методы решения ДУЧП. Типы ДУЧП.}

Уравнение, связывающее неизвестную функцию \( u(x_1, \ldots, x_n \), независимые
переменные \( x_1 \), \ldots, \( x_n \) и частные производные от неизвестной
функции, называется дифференциальным уравнением с частными производными. Оно
имеет вид:
\[
    F\left(x_1, x_2, \ldots, x_n, u, \pder{u}{x_1}, \pder{u}{x_2}, \ldots,
    \pder{u}{x_n}, \ppder{u}{x_1}, \pcder{u}{x_1}{x_2}, \ppder{u}{x_2}, \ldots,
    \pnder{n}{u}{x_m}\right) = 0,
\]
где \( F \) -- заданная функция своих аргументов.

Порядок старшой производной, входящей в уравнение, называется порядком ДУЧП.

\begin{table}[h!]
    \begin{tabular}{m{.48\textwidth}cm{.48\textwidth}}
        \header{Типы ДУЧП}

        Существует 6 способов классификации ДУЧП:
        \begin{enumerate}\itemsep-.4em
            \item порядок уравнения;
            \item число переменных;
            \item линейность (функция и все её производные входят в уравнение
            линейным образом);
            \item однородность;
            \item вид коэффициентов (постоянные либо переменные);
            \item три основных вида линейные ДУЧП II:
            \begin{itemize}\itemsep-.4em
                \item параболические;
                \item эллиптические;
                \item гиперболические.
            \end{itemize}

            Также при переменных коэффициентах возможен <<смешанный>> тип
            уравнения.
        \end{enumerate}
        % -----------------------
        & \hfill &
        % -----------------------
        \header{Методы решения ДУЧП}

        \begin{enumerate}
            \item Метод разделения переменных (метод Фурье);
            \item метод интегральных преобразований;
            \item метод преобразования координат;
            \item метод преобразования зависимой переменной;
            \item численные методы;
            \item методы теории возмущений;
            \item метод функций Грина;
            \item метод интегральных уравнений;
            \item вариационные методы;
            \item метод разложения по собственным функциям.
        \end{enumerate}
    \end{tabular}
\end{table}

\header{Примеры}

\begin{enumerate}
    \item Волновое уравнение:
    \[
        \ppder{u}{t} = c^2\left(\ppder{u}{x} + \ppder{u}{y} +
        \ppder{u}{z}\right);
    \]

    \item уравнение теплопроводности:
    \[
        \pder{u}{t} = \alpha^2\left(\ppder{u}{x} + \ppder{u}{y} +
        \ppder{u}{z}\right);
    \]

    \item уравнение Лапласа:
    \[
        \ppder{u}{x} + \ppder{u}{y} + \ppder{u}{z} = 0.
    \]
\end{enumerate}

\newpage % ---------------------------------------------------------------------
