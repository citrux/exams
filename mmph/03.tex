\chapter{Граничные условия в задачах диффузионного типа.}

В задачах диффузионного типа используются ГУ трёх основных видов:
\begin{enumerate}
    \item ГУ I рода -- заданы значения функции на границе области:
    \[
        \left\{ \begin{array}{l}
            u(0, t) = g_1(t), \\
            u(l, t) = g_2(t).
        \end{array} \right.
    \]
    
    \item ГУ II рода -- задана температура окружающей среды \( g_1(t) \) и
    \( g_2(t) \) вблизи границ области:
    \[
        \left\{ \begin{array}{l}
            u'_x(0, t) = \cfrac{h}{\kappa}\Big(u(0, t) - g_1(t)\Big), \\
            u'_x(l, t) = \cfrac{h}{\kappa}\Big(u(l, t) - g_2(t)\Big).
        \end{array} \right.
    \]
    
    \item ГУ III рода -- задан поток через границу области (градиент на границе
    области):
    \[
        \left\{ \begin{array}{l}
            u'_x(0, t) = g_1(t), \\
            u'_x(l, t) = g_2(t).
        \end{array} \right.
    \]
\end{enumerate}

\newpage % ---------------------------------------------------------------------
