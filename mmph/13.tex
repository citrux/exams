\chapter{Свойства преобразования Фурье. Решение задачи о распространении тепла
в бесконечном стержне с заданной начальной температурой.}

\emph{Свойства преобразования Фурье}:
\begin{enumerate}
    \item обратимость: \( F^{-1}\bigl[F[f]\bigr] = f \);
    \item линейность: \( F[af + bg] = aF[f] + bF[g] \);
    \item преобразование производных:
    \( \ds
        \pnder{n}{u}{x} \to (\i\xi)^n\cdot F_x[u],\ 
        \pnder{n}{u}{t} \to \pnder{n}{}{t} F_x[u];
    \)
    \item свёртка:    
    \( F[f * g] = F[f]\cdot F[g] \), где
    \[ 
        (f * g)(x) = \frac{1}{\sqrt{2\pi}}\int\limits_{-\infty}^{+\infty}
        f(x-\xi)g(\xi)\d\xi = F^{-1}\bigr[F[f]\cdot F[g]\bigl]
        \text{ -- свёртка}.
    \]
\end{enumerate}

\begin{minipage}{.67\textwidth}
     Рассмотрим теперь \emph{задачу о распространении тепла в бесконечном стержне},
     если задана начальная температура \( u(x, 0) = \phi(x) \).
     
    Сделаем преобразование Фурье по переменной \( x \): \( u \to \bar{u} \).
\end{minipage}
\hfill
\begin{minipage}{.3\textwidth}
    \[
        \left\{ \begin{array}{l}
            \ds \pder{u}{t} = \alpha^2 \ppder{u}{x}; \\
            u(x, 0) = \phi(x).
        \end{array} \right.
    \]
\end{minipage}

\begin{minipage}{.67\textwidth}
    Преобразованная задача:
    
    Её решение: \( \ds \bar{u} = \bar{\phi}(\xi)\cdot\e^{-\alpha^2\xi^2 t} \).
\end{minipage}
\hfill
\begin{minipage}{.3\textwidth}
    \[
        \left\{ \begin{array}{l}
            \ds \pder{u}{t} = \alpha^2 \ppder{u}{x}; \\
            u(x, 0) = \phi(x).
        \end{array} \right.
    \]
\end{minipage}

Обратным преобразованием Фурье получаем ответ:
\[
    u(x, t) = F^{-1}[\bar{\phi}(\xi)\cdot\e^{-\alpha^2\xi^2 t}] =
    F^{-1}[\bar{\phi}(\xi)] * F^{-1}[\e^{-\alpha^2\xi^2 t}] =
    \frac{1}{2\alpha\sqrt{\pi t}} \int\limits_{-\infty}^{+\infty} \phi(s)
    \e^{\frac{-(x-s)^2}{4\alpha^2 t}}\d s.
\]

Функция \( \ds G(x, t) = \frac{1}{2\alpha\sqrt{\pi t}}
\e^{\frac{-x^2}{4\alpha^2 t}} \) называется функцией Грина. Тогда ответ
может быть представлен в виде:
\[
    u(x, t) = \int\limits_{-\infty}^{+\infty} \phi(s)\cdot G(x-s, t)\d s.
\]

\newpage
