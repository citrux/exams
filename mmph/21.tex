\chapter{Решение задачи о колебаниях ограниченной струны методом разделения
переменных.}

В качестве примера рассмотрим задачу с закрепленным концом:

\begin{minipage}{.4\textwidth}
\[
    \left. \begin{array}{rl}
        \text{ДУЧП:} & \ds \ppder{u}{t} = c^2\ppder{u}{x},
        \quad x\in(0, l), \\[.4em]
        \text{ГУ:} & \left\{ \begin{array}{l}
            u(0, t) = 0, \\
            u(l, t) = 0, 
        \end{array} \right. \\
        \text{НУ:} & \left\{ \begin{array}{l}
            u(x, 0) = \phi(x), \\
            \ds \pder{u}{t}(x, 0) = \psi(x).
        \end{array} \right.
    \end{array} \right\}
\]
\end{minipage}
\hfill
\begin{minipage}{.5\textwidth}
    Воспользуемся методом разделения переменных. Будем искать 
    решение вида
    
    \( u(x,t) = X(x)\cdot T(t) \), называемые стоячими волнами.

    Подстановкой в ДУЧП получаем:
    \[
        \frac{T''}{c^2T} = \frac{X''}{X} = -k^2.
    \]
\end{minipage}

\vspace*{.4em}
Получаем ОДУ для \( T \): \( T'' + \lambda^2 c^2 T = 0 \) и задачу
Штурма-Лиувилля для \( X \):
\[
    \left. \begin{array}{l}
    	X'' + k^2 X = 0,\quad x\in[0, l] \\
    	\left\{ \begin{array}{l}
    		X(0) = 0, \\
    		X(l) = 0; \\
    	\end{array} \right.
    \end{array} \right| \Rightarrow X_n = \sin(k_n x), 
    \text{ где } k_n = \frac{\pi n}{l}.
\]

Для \( T \) получаем:
\[ 
	T_n = a_n\cos(\omega_n t) + b_n\sin(\omega_n t), \text{ где }
	\omega_n = k_n\cdot c.
\]

Таким образом, 
\[ 
	u(x, t) = \sum_{n=1}^{\infty}T_n\cdot X_n = \sum_{n=1}^{\infty}
	\left( a_n\cos\frac{\pi nc}{l}t + b_n\sin\frac{\pi nc}{l}t \right)
	\cdot\sin\frac{\pi n}{l}x.
\]

Теперь воспользуемся начальными условиями:
\begin{align*}
	& u(x, 0) = \sum_{n=1}^{\infty} a_n\sin\frac{\pi n}{l}x = \phi(x)
	\Rightarrow a_n = \frac{2}{l} \int\limits_0^l \phi(x)\cdot
	\sin\frac{\pi n}{l}x\d x; \\
	& \pder{u}{t}(x, 0) = \sum_{n=1}^{\infty} \frac{\pi nc}{l}
	b_n\sin\frac{\pi n}{l}x = \psi(x) \Rightarrow b_n = \frac{2}{\pi nc}
	\int\limits_0^l \psi(x)\cdot\sin\frac{\pi n}{l}x\d x.
\end{align*}
В итоге, 
\[
	u(x, t) = \sum_{n=1}^{\infty}\left( a_n\cos\frac{\pi nc}{l}t + 
	b_n\sin\frac{\pi nc}{l}t \right)\sin\frac{\pi n}{l}x
\]
где \( a_n \) и \( b_n \) определяются по приведенным выше формулам.

\newpage
