\chapter{Метод разделения переменных и его применение в диффузионных задачах.}

Суть метода разделения переменных состоит в том, что частное решение задачи
ищется в виде композиции функций каждой их переменных. Рассмотрим его применение
к задаче диффузионного типа:

\begin{minipage}{.4\textwidth}
\[
    \left. \begin{array}{rl}
        \text{ДУЧП:} & \ds \pder{u}{t} = \alpha^2\ppder{u}{x}; 
        \vspace*{.4em} \\
        \text{ГУ:} & \left\{ \begin{array}{l}
            u(0, t) = f_1(t), \\
            u(l, t) = f_2(t); 
        \end{array} \right. \\
        \text{НУ:} & u(x, 0) = \phi(x).
    \end{array} \right\}
\]
\end{minipage}
\hfill
\begin{minipage}{.56\textwidth}
    Так как в ДУЧП входят две независимые переменные \( x \) и \( t \), то
    частное решение будем искать в виде \( u = X(x)T(t) \).
    
    Подставим его в ДУЧП:
    \[
        X\cdot T' = \alpha^2 T\cdot X''.
    \]
\end{minipage}
    
    Разделив на \( \alpha^2 XT \), имеем:
    \[
        \frac{T'}{\alpha^2 T} = \frac{X''}{X} = k,
    \]
    где \( k \) -- константа разделения.

Получаем два ОДУ: \( X'' - kX = 0 \) и \( T' - k\alpha^2T = 0 \). Из
функциональных соображений решение должно быть ограничено. Это возможно, если
\( k = -\lambda^2 \), где \( \lambda \in (0, +\infty) \). Тогда ОДУ:
\[
    X'' + \lambda^2X = 0, \quad T' + \lambda^2\alpha^2T = 0,
\]
и их решения:
\[
    X = A\sin\lambda x + B\cos\lambda x, \quad T = \e^{-\lambda^2\alpha^2 t}.
\]

Получаем, что
\[
    u = (A\sin\lambda x + B\cos\lambda x)\e^{-\lambda^2\alpha^2 t}.
\]

Найденное решение должно удовлетворять граничным условиям:
\begin{align*}
    & u(0, t) = 0 \Rightarrow A\sin0 + B\cos0 = 0 \Rightarrow B = 0; \\
    & u(l, t) = 0 \Rightarrow A\sin\lambda l = 0 \Rightarrow \lambda l = \pi n
    \Rightarrow \lambda_n = \frac{\pi n}{l}.
\end{align*}

Получаем семейство функций:
\[
    u_n = A_n\sin\frac{\pi n}{l}x \cdot \e^{-\left(\frac{\alpha\pi n}
    {l}\right)^2t},
\]
а общее решение краевой задачи получаем в виде:
\[
    u = \sum\limits_{n = 1}^\infty A_n\sin\frac{\pi n}{l}x \cdot
    \e^{-\left(\frac{\alpha\pi n}{l}\right)^2t}.
\]

Это решение должно удовлетворять начальному условию:
\[
    \phi(x) = \sum\limits_{n = 1}^\infty A_n\sin\frac{\pi n}{l}x \Rightarrow
    A_n = \frac{2}{l}\int\limits_0^l \phi(x)\sin\frac{\pi n}{l}x.
\]

\newpage % ---------------------------------------------------------------------

