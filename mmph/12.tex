\chapter{Ряд Фурье и его коэффициенты. Теорема Дирихле. Дискретный частотный
спектр периодической функции. Преобразование Фурье.}

\section{Ряд Фурье}
Функциональный ряд вида \( \ds \frac{a_0}{2} + \sum\limits_{n=1}^\infty
\Bigl(a_n\cos nx + b_n\sin nx\Bigr) \) называется тригонометрическим рядом, а
постоянные числа \( a_0 \), \( a_n \) и \( b_n \) -- коэффициентами
тригонометрического ряда. Если такой ряд сходится, то его сумма -- периодическая
функция с периодом \( 2\pi \),

Пусть функция \( f(x) \) такова, что она представляется тригонометрическим рядом
на интервале \( (-\pi, \pi) \), то есть \( \ds f(x) = \frac{a_0}{2} +
\sum\limits_{n=1}^\infty \Bigl(a_n\cos nx + b_n\sin nx\Bigr) \). Предположим,
что интеграл от функции равен сумме интегралов от членов ряда (это возможно,
если ряд мажорируем, то есть абсолютно сходится ряд коэффициентов). Интегрируя
левую и правую часть от \( -\pi \) до \( \pi \) имеем:
\[
    \int\limits_{-\pi}^\pi f(x)\d x = \int\limits_{-\pi}^\pi \frac{a_0}{2}\d x +
    \sum\limits_{n=1}^\infty \left(a_n\int\limits_{-\pi}^\pi \cos nx\d x + b_n
    \int\limits_{-\pi}^\pi \sin nx\d x\right) = \pi a_0 \Rightarrow
    a_0 = \frac{1}{\pi}\int\limits_{-\pi}^\pi f(x) \d x.
\]

Домножив обе части на \( \cos mx \) и проинтегрировав, получим:
\( \ds a_m = \frac{1}{\pi} \int\limits_{-\pi}^\pi f(x)\cos mx\d x \).

Проводя аналогичные действия с \( \sin mx \), получим:
\( \ds b_m = \frac{1}{\pi} \int\limits_{-\pi}^\pi f(x)\sin mx \d x \).

Коэффициенты, определённые по этим формулам, называются коэффициентами Фурье
функции \( f(x) \), а тригонометрический ряд с такими коэффициентами -- рядом
Фурье функция \( f(x) \).

\section{Теорема Дирихле}
Если \( f(x) \) -- ограниченная периодическая функция, имеющая на каждом периоде
конечное число максимумов, минимумов и точек разрыва, то ряд Фурье функции
\( f(x) \) сходится к \( f(x) \) в каждой точке непрерывности функции и к
среднему арифметическому пределов функции слева и справа в точке разрыва.

\section{Дискретный частотный спектр периодической функции}
Если \( f(x) \) -- периодическая функция, то разложение в ряд Фурье можно
интерпретировать как сопоставление функции \( f(x) \) последовательности
\( \bigl\{ C_n \bigr\} \), где \( C_n = \sqrt{a_n^2 + b_n^2} \).

\section{Преобразование Фурье}
Рассмотрим непериодическую функцию \( f(x) \), определенную на \( (-\infty,
\infty) \) такую, что \( \ds \int\limits_{-\infty}^{+\infty} |f(x)|\d x = Q \).
Тогда в пределе \( l \to \infty \) можно перейти от ряда Фурье на конечном
отрезке \( [-l, l] \) к интегралу Фурье:
\[
    f(x) = \int\lni a(\xi)\cos(\xi x)\d\xi + \int\lni b(\xi)\sin(\xi x)\d\xi,
\]
где \( \ds a(\xi) = \frac{1}{\pi} \int\limits_{-\infty}^{+\infty}
f(x)\cos(\xi x)\d x \), \( \ds b(\xi) = \frac{1}{\pi}
\int\limits_{-\infty}^{+\infty} f(x)\sin(\xi x)\d x \).

Если учесть, что \( \ds \sin x = \frac{\e^{\i x} - \e^{-\i x}}{2\i} \), а
\( \ds \cos x = \frac{\e^{\i x} + \e^{-\i x}}{2} \), то можно записать
комплексный вид интеграла Фурье:
\[
    f(x) = \frac{1}{2\pi}\int\limits_{-\infty}^{+\infty} \left[
    \int\limits_{-\infty}^{+\infty} f(x)\cdot\e^{-\i\xi x}\d x\right]
    \e^{\i\xi x}\d\xi.
\]

Отсюда получаем пару интегральных преобразований:
\[
    F[f] = F(\xi) = \frac{1}{\sqrt{2\pi}}\int\limits_{-\infty}^{+\infty} f(x)
    \e^{-\i\xi x}\d x; \quad
    F^{-1}[F] = f(x) = \frac{1}{\sqrt{2\pi}}\int\limits_{-\infty}^{+\infty} f(x)
    \e^{\i\xi x}\d\xi.
\]
\newpage
