\chapter{Колебания мембраны. Решение задачи на собственные значения для
уравнения Гельмгольца.}

\section{Колебания мембраны}
Найдем колебания \emph{круглой мембраны} с заданными \emph{начальными и 
граничными условиями}. Будем считать, что радиус мембраны равен единице, 
а смещение по границе равно нулю. Пусть \( u(r, \theta, t) \) обозначает 
возвышение точек мембраны под плоскостью равновесия. 
Рассмотрим задачу:

\[
    \left. \begin{array}{rl}
        \text{ДУЧП:} & \ds \pder{u}{t} = c^2\left( \ppder{u}{r} + 
        \frac{1}{r}\pder{u}{r} + \frac{1}{r^2}\ppder{u}{\theta} \right), 
        0 < r < 1, t > 0; \\
        \text{ГУ:} & u = 0 \text{ при } r = 1, t > 0
        \vspace*{.4em} \\
        \text{НУ:} & \left\{ \begin{array}{l}
            u(r,\theta,0) = \phi(r, \theta), \\
            \ds \pder{u}{t}(r,\theta,0) = \psi(r, \theta). 
        \end{array} \right. \\
    \end{array} \right.
\]

Воспользуемся методом разделения переменных и будем искать решение в виде
стоячих волн: 
\[
    u(r, \theta, t) = U(r, \theta)T(t).
\]

Подставим это представление в ДУЧП и получим два уравнения: 
\[
    \Delta U + \lambda^2 U = 0 \text{ (уравнение Гельмгольца)},
\]
\[
    T'' + \lambda^2 c^2 T = 0 \text{ (уравнение гармонических колебаний)}.
\]
где 
\[
    \Delta U =
    \ppder{U}{r} + \frac{1}{r}\pder{U}{r} + \frac{1}{r^2}\ppder{U}{\theta}
    \text{ -- лапласиан в полярных координатах }.
\]

Для периодичности функции \( T(t) \) мы взяли константу разделения 
отрицательной и обозначили \( -\lambda^2 \).

\section{Решение задачи на собственные значения для уравнения Гельмгольца}
С учётом ГУ получаем задачу на собственные значения для уравнения Гельмгольца:
\[
    \Delta U + \lambda^2 U = 0, \quad U(1, \theta) = 0.
\]
Ещё раз разделив переменные \( U(r, \theta) = R(r)\Theta(\theta) \), получим:
\[
    r^2 R'' + rR' + (\lambda^2 r^2 - n^2)R = 0 \text{ (уравнение Бесселя)}
    \quad R(1) = 0,
\]
\[
    \Theta'' + n^2\Theta = 0.
\]

Общее решение уравнения Бесселя имеет вид:
\[
    R(r) = C_1J_n(\lambda r) + C_2Y_n(\lambda r),
\]
где \( J_n(\lambda r) \) -- функция Бесселя первого рода n-го порядка, 
\( Y_n(\lambda r) \) -- функция Бесселя второго рода n-го порядка.

Поскольку функция \( Y_n(\lambda r) \) не ограничена при \( r = 0 \), то 
выберем решение в виде
\[
    R(r) = CJ_n(\lambda r).
\]

Последний шаг в определении \( R(r) \) связан с использованием граничного 
условия \( R(1) = 0 \) для определения всех \( \lambda \). Подставляя 
это условие в \( R(r) = J_n(\lambda r) \), получаем \( J_n(\lambda) = 0 \).
Иначе говоря, для того чтобы функция R(r) обращалась в нуль на границе 
круга, мы должны выбрать константу разделения так, чтобы она была корнем 
уравнения \( J_n(r) = 0 \), т.е. \( \lambda = k_{nm} \), где 
\( k_{nm} \) -- корень m-й степени уравнения \( J_n(r) \) = 0. Тогда 
собственная функция \( U_{nm} \) определяется формулой:
\[ 
    U_{nm}(r, \theta) = J_n(k_{nm}r)
    \left(A\cos(n\theta) + B\sin(n\theta)\right). 
\]

Подставляя полученные соотношения в исходную задачу, получим:
\[
    u(r, \theta, t) = \sum_{n=0}^{\infty}\sum_{m=0}^{\infty}
    J_n(k_{nm}r)\left(A_{n}\cos(n\theta) + B_{n}\sin(n\theta)\right)
    \cdot\left(a_{nm}\cos(k_{nm}ct) + b_{nm}\sin(k_{nm}ct)\right).
\]
