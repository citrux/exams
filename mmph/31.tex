\chapter{Колебания мембраны. Решение задачи на собственные значения для
уравнения Гельмгольца.}

\section{Колебания мембраны}
Найдем колебания \emph{круглой мембраны} с заданными \emph{начальными и 
граничными условиями}. Будем считать, что радиус мембраны равен единице, 
а смещение по границе равно нулю. Пусть \( u(r, \theta, t) \) обозначает 
возвышение точек мембраны под плоскостью равновесия. 
Рассмотрим задачу:

\[
    \left. \begin{array}{rl}
        \text{ДУЧП:} & \ds \pder{u}{t} = c^2\left( \ppder{u}{r} + 
        \frac{1}{r}+\pder{u}{r} + \frac{1}{r^2}\ppder{u}{\theta} \right), 
        0 < r < 1, t > 0; \\
        \text{ГУ:} & u = 0 \text{ при } r = 1, t > 0
        \vspace*{.4em} \\
        \text{НУ:} & \left\{ \begin{array}{l}
            u = f(r, \theta), \\
            \pder{u}{t} = g(r, \theta) \text{ при } t = 0 
        \end{array} \right. \\
    \end{array} \right.
\]

Решение задачи будем искать в виде: 
\[ u(r, \theta, t) = U(r, \theta)T(t) \]
Форма этих колебаний определяется функцией \( U(r, \theta) \), а 
характер осцилляций -- множителем \( T(t) \).

Подставим это представление в волновое уравнение, получим два уравнения: 
\[ \Delta U + \lambda^2 U = 0 \text{ (уравнение Гельмгольца)} \]
\[ T'' + \lambda^2 c^2 T = 0 \text{ (уравненеи гармонических колебаний)} \]
где 
\[ \Delta U = \ppder{U}{r} + \frac{1}{r}\pder{U}{r} + \frac{1}{r^2}\ppder{U}{\theta} \]

Для периодичности функции \( T(t) \) мы взяли константу разделения 
положительной и обозначили \( \lambda^2 \).

Найдём граничные условия для мембраны:
\[ u(1, \theta, t) = U(1, \theta)T(t) = 0, t > 0 \]
или
\[ U(1, \theta) = 0 \]

Для нахождения формы колебаний мембраны, решим получившуюся задачу.

\newpage

\section{Решение задачи на собственные значения для уравнения Гельмгольца}
\[ \Delta U + \lambda^2 U = 0 \]
\[ U(1, \theta) = 0 \]
Сделаем подстановку \( U(r, \theta) = R(r)\Theta(\theta) \), получим:
\[ r^2 R'' + rR' + (\lambda^2 r^2 - n^2)R = 0 \text){(уравнение Бесселя)} \]
\[ R(1) = 0, \]
\[ \Theta'' + n^2\Theta = 0 \]

Общее решение уравнения Бесселя представимо в виде:
\[ R(r) = AJ_n(\lambda r) + BY_n(\lambda r) \]
где \( AJ_n(\lambda r) \) -- функция Бесселя первого рода n-го порядка, 
\( BY_n(\lambda r) \) -- функция Бесселя второго рода n-го порядка.

Поскольку функци \( Y_n(\lambda r) \) не ограничена при \( r = 0 \), 
выберем решение в виде:
\[ R(r) = AJ_n(\lambda r) \]

Последний шаг в определении \( R(r) \) связан с использованием граничного 
условия \( R(1) = 0 \) для определения всех \( \lambda \). Подставляя 
это условие в \( R(r) = J_n(\lambda r) \), получаем \( J_n(\lambda) = 0 \).
Иначе говоря, для того чтобы функция R(r) обращалась в нуль на границе 
круга, мы должны выбрать константу разделения так, чтобы она была корнем 
уравнения \( J_n(r) = 0 \), т.е. \( \lambda = k_{nm} \), где 
\( k_{nm} \) -- корень m-й степени уравнения \( J_n(r) \) = 0. Тогда 
собственная функция \( U_{nm} \) определяется формулой:
\[ 
    U_{nm}(r, \theta) = J_n(k_{nm}r)
    \left[ A\sin(n\theta) + B\cos(n\theta) \right] 
\]

Подставляя полученные соотношения, в исходную задачу, получим:
\[
    u(r, \theta, t) = \sum_{n=0}^{\infty}\sum_{m=0}^{\infty}
    J_n(k_{nm}r)\left[ A\sin(n\theta) + B\cos(n\theta) \right]
    \cdot\left[ A\sin(k_{nm}t) + B\cos(k_{nm}t) \right]
\]
где вторая скобка получается из решения уравнения:
\[ T'' + k^2_{nm}c^2T = 0 \]
