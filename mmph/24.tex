\chapter{Классификация ДУЧП II. Приведение к каноническому виду.}

Рассмотрим ДУЧП II двух переменных вида
\[
    A\ppder{u}{x} + 2B\pcder{u}{x}{y} + C\ppder{u}{y} = F\left(\pder{u}{x},
    \pder{u}{y}, u, x, y\right),
\]
где \( A \), \( B \) и \( C \) -- функции \( x \) и \( y \).
\begin{enumerate}
    \item Если в данной точке \( (x_0, y_0) \) выражение \( B^2 - AC > 0 \),
    то уравнение называется гиперболическим в данной точке.
    \item Если в данной точке \( (x_0, y_0) \) выражение \( B^2 - AC = 0 \),
    то уравнение называется параболическим в данной точке.
    \item Если в данной точке \( (x_0, y_0) \) выражение \( B^2 - AC < 0 \),
    то уравнение называется эллиптическим в данной точке.
\end{enumerate}

Каноническим уравнением гиперболического типа называется уравнение вида
\[
    \pcder{u}{x}{y} = F\left(x, y, u, \pder{u}{x}, \pder{u}{y}\right).
\]

Каноническим уравнением параболического типа называется уравнение вида
\[
    \ppder{u}{x} = F\left(x, y, u, \pder{u}{x}, \pder{u}{y}\right).
\]

Каноническим уравнением эллиптического типа называется уравнение вида
\[
    \ppder{u}{x} + \ppder{u}{y} = F\left(x, y, u, \pder{u}{x}, \pder{u}{y}
    \right).
\]

\vspace*{.6em}
Для приведения ДУЧП II к каноническому виду используется уравнение
характеристик \( A\d y^2 - 2B\d x\d y + C\d x^2 = 0 \), интегралы которого
определяют канонические координаты, то есть координаты в которых ДУЧП
записывается в каноническом виде.
\begin{enumerate}
    \item В случае \( B^2 - AC > 0 \) уравнение характеристик имеет два
    различных интеграла \( \phi(x, y) = C_1,\ \psi(x, y) = C_2 \), и
    непосредственной заменой \( \xi =\phi(x, y), \eta = \psi(x, y) \)
    гиперболическое уравнение приводится к каноническому виду.

    \item В случае \( B^2 - AC = 0 \) уравнение характеристик имеет всего один
    интеграл\\
    \( \phi(x, y) = C_1 \), а канонические координаты определяются
    следующим образом:\\
    \( \xi =\phi(x, y), \eta = \psi(x, y) \), где
    \( \psi(x, y) \) -- любая функция, удовлетворяющая условию: \( \xi_x\eta_y -
    \xi_y\eta_x \ne 0 \).
    
    \item В случае \( B^2 - AC < 0 \) интегралы уравнения характеристик имеют
    вид:\\
    \( \phi(x, y)- \i\psi(x, y) = C_1 \), \( \phi(x, y) + \i\psi(x, y) =
    C_2 \), а канонические координаты определяются так: \( \xi =\phi(x, y),
    \eta = \psi(x, y) \).
\end{enumerate}

\newpage
