\chapter{Решение одномерного волнового уравнения с помощью формулы д'Аламбера.
Примеры применения формулы д'Аламбера в некоторых конкретных задачах.}

Начнём с решения задачи Коши:
\begin{align*}
    \ppder{u}{t} = c^2\ppder{u}{x}, \quad x\in\mathbb{R},\ t > 0, \\
    \left\{ \begin{array}{l}
        u(x, 0) = \phi(x), \\
        \ds \pder{u}{t}(x, 0) = \psi(x).
    \end{array} \right.
\end{align*}

Уравнение \( \ds \ppder{u}{t} - c^2\ppder{u}{x} = 0 \) является уравнением
гиперболического типа. Приведем его к каноническому виду. Уравнение характеристик
имеет вид: \( \d x^2 - c^2\d t^2 = 0 \), и далее:
\[
    \left[ \begin{array}{l}
        \d x - c\d t = 0 \\
        \d x + c\d t = 0
    \end{array} \right.
    \quad\Leftrightarrow\quad
    \left[ \begin{array}{l}
        x - ct = C_1 = \xi \\
        x + ct = C_2 = \eta
    \end{array} \right.
\]

В координатах \( \xi \) и \( \eta \) уравнение приобретает канонический вид:
\( \ds \pcder{u}{\xi}{\eta} = 0 \).
Дважды проинтегрировав его, найдем, что \( u = f(\xi) + g(\eta) =
f(x - ct) + g(x + ct) \).

Теперь воспользуемся начальными условиями:
\[
    \left\{ \begin{array}{l}
        f(x) + g(x) = \phi(x), \\
        -cf'(x) + cg'(x) = \psi(x),
    \end{array} \right.
    \quad\Leftrightarrow\quad
    \left\{ \begin{array}{l}
        f(x) + g(x) = \phi(x), \\
        \ds -f(x) + g(x) = \frac{1}{c}\int\limits_{x_0}^x\psi(\zeta)\d\zeta + K.
    \end{array} \right.
\]

\begin{align*}
    \text{Отсюда, }\quad & f(x) = \frac{1}{2}\phi(x) -
    \frac{1}{2c}\int\limits_{x_0}^x \psi(\zeta)\d\zeta - \frac{K}{2}, \\
    & g(x) = \frac{1}{2}\phi(x) + \frac{1}{2c}\int\limits_{x_0}^x
    \psi(\zeta)\d\zeta + \frac{K}{2}.
\end{align*}

Тогда решение имеет вид
\[
    u(x, t) = \frac{1}{2}\bigl[\phi(x - ct) + \phi(x + ct)\bigr] + \frac{1}{2c}
    \int\limits_{x - ct}^{x + ct} \psi(\zeta)\d\zeta.
\]

Полученное выражение называется формулой д'Аламбера.

\section{Примеры применения формулы д'Аламбера}
\begin{enumerate}
    \item Рассмотрим начальные условия вида 
    \[
        \left\{ \begin{array}{l}
            u(x, 0) = \sin x, \\
            \ds \der{u}{t}(x, 0) = 0.
        \end{array} \right.
    \]
    
    По формуле д'Аламбера получаем решение:
    \[
        u(x, t) = \frac{1}{2}\bigl(\sin(x - ct) + \sin(x + ct)\bigr),
    \]
    которое можно интерпретировать так: начальное смещение струны
    \( u(x, 0) = \sin x \) делится на две одинаковые части, и каждая из частей
    распространяется со скоростью \( c \) в виде бегущих волн.
    Одна из волн перемещается слева направо, а вторая -- в противоположном
    направлении.
    
    \item Рассмотрим начальные условия вида 
    \[
        \left\{ \begin{array}{l}
            u(x, 0) = 0, \\
            \ds \der{u}{t}(x, 0) = \sin x.
        \end{array} \right.
    \]
    
    По формуле д'Аламбера получаем решение:
    \[
        u(x, t) = \frac{1}{2c}\int\limits_{x - ct}^{x + ct} \sin\zeta\d\zeta =
        \frac{1}{2}\bigl(\cos(x - ct) - \cos(x + ct)\bigr),
    \]
    представляющее собой сумму двух бегущих косинусоидальных волн.
\end{enumerate}

\newpage
