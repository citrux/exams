\chapter{Интегральные преобразования. Спектр функции. Синус- и
косинус-преобразования производных.}

\section{Интегральные преобразования}
Интегральным преобразованием называют преобразование, которое ставит в
соответствие функции \( f(t) \) такую функцию \( F(s) \), что
\[
    F(s) = \int\limits_A^B K(s, t)f(t)\d t,
\]
где \( K(s, t) \) -- ядро преобразования. Интегральные преобразования ДУЧП
позволяют избавиться от одной из переменных, переводя ДУЧП в ОДУ или ДУЧП с
меньшим на единицу количеством переменных. ОДУ же переходит в алгебраические
уравнения. Производя обратное преобразование решения полученного уравнения
получаем решение исходного уравнения.

\begin{table}[h!]
    \center
    \caption{Некоторые виды интегральных преобразований}
    \begin{tabular}{|C{.17}|C{.4}|C{.4}|} \hline
        Название & Прямое преобразование & Обратное преобразование \\ \hline
        Синус-преобразование Фурье &
        % используется \( \ds x \) для уменьшения занимаемого места
        \( \ds
            F_s[f] = F(\omega) = \frac{2}{\pi}\int\lni f(t)\sin(\omega t)\d t
        \)
        &
        \( \ds
            F_s^{-1}[F] = f(t) = \int\lni F(\omega)\sin(\omega t)\d\omega
        \) \\ \hline
        % --------------------
        Косинус-преобразование Фурье &
        \( \ds
            F_c[f] = \frac{2}{\pi}\int\lni f(t)\cos(\omega t)\d t
        \)
        &
        \( \ds
            F_c^{-1}[F] = \int\lni F(\omega)\cos(\omega t)\d\omega
        \) \\ \hline
        % --------------------
        Преобразование Фурье &
        \( \ds
            F[f] = \frac{1}{\sqrt{2\pi}}\int\limits_{-\infty}^{+\infty} f(t)
            \e^{-\i\omega t}\d t
        \)
        &
        \( \ds
            F^{-1}[F] = \frac{1}{\sqrt{2\pi}}\int\limits_{-\infty}^{+\infty}
            F(\omega)\e^{\i\omega t}\d\omega
        \) \\ \hline
        % --------------------
        Конечное синус-преобразование &
        \( \ds
            F_s[f] = S_n = \frac{2}{L}\int\limits_0^L f(t)\sin\left(
            \frac{\pi nt}{L}\right)\d t
        \)
        &
        \( \ds
            F_s^{-1}[S_n] = \sum\limits_{n=1}^\infty S_n\sin\left(
            \frac{\pi nt}{L}\right)
        \) \\ \hline
        % --------------------
        Конечное косинус-преобразование &
        \( \ds
            F_c[f] = C_n = \frac{2}{L}\int\limits_0^L f(t)\cos\left(
            \frac{\pi nt}{L}\right)\d t
        \)
        &
        \( \ds
            F_c^{-1}[C_n] = \frac{C_0}{2} + \sum\limits_{n=1}^\infty
            C_n\cos\left(\frac{\pi nt}{L}\right)
        \) \\ \hline
        % --------------------
        Преобразование Лапласа &
        \( \ds
            L[f] = \int\lni f(t)\e^{-st}\d t
        \)
        &
        \( \ds
            L^{-1}[F] = \frac{1}{2\pi\i}\int\limits_{c-\i\infty}^{c+\i\infty}
            F(s)\e^{st}\d s
        \) \\ \hline
        % --------------------
        Преобразование Ханкеля &
        \( \ds
            H[f] = F_n(\xi) = \int\lni tJ_n(t\xi)f(t)\d t
        \)
        &
        \( \ds
            H^{-1}[F_n] = f(t) = \int\lni \xi J_n(t\xi)F_n(\xi)\d\xi
        \) \\ \hline
    \end{tabular}
\end{table}

\section{Спектр функции}
Интегральное преобразование по сути является разложением функции в спектр
(непрерывный в случае бесконечных преобразований и дискретный в случае
конечных). Так, при разложении функции в ряд Фурье
\[
   f(x) = \sum\limits_{n=0}^\infty \Bigl(A_n\cos nx + B_n\sin nx\Bigr)
\]
функция \( S(n) = \sqrt{A_n^2 + B_n^2} \) называется спектром функции. В данном
случае спектр дискретен. Он показывает вклад \( n \)-ой гармоники в функцию
\( f(x) \).

В случае преобразования Фурье имеет место непрерывный спектр:
\[
    S(\omega) = \left|\frac{1}{\sqrt{2\pi}}\int\limits_{-\infty}^{+\infty} f(t)
    \cdot\e^{-\i\omega t}\,dt\right|.
\]

\section{Синус- и косинус-преобразования производных}
\begin{table}[h!]
    \begin{tabular}{C{.47}cC{.47}}
        \( \ds
            F_s[f'] = -\omega F_c[f],
        \)
        \[
            F_s[f''] = \frac{2}{\pi}\omega f(0) - \omega^2 F_s[f].
        \] & \hfill &
        \( \ds
            F_c[f'] = \frac{2}{\pi}f(0) + \omega F_s[f],
        \)
        \[
            F_c[f''] = -\frac{2}{\pi}\omega f(0) - \omega^2 F_c[f].
        \] 
    \end{tabular}
\end{table}
\newpage
