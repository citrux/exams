\chapter{Вывод уравнения теплопроводности.}

Запишем закон Фурье: \( \vec{j} = -\kappa\cdot\grad u \), где \( u \)
имеет смысл температуры. Рассмотрим малую окрестность точки наблюдения.
Внутренняя энергия, заключенная в её объеме \( V \) может измениться только за
счет теплообмена, то есть:
\[
    \der{W}{t} = -\oiint\limits_S \vec{j}\cdot\,d\vec{S}.
\]

Для интеграла в правой части по теореме Остроградского имеем
    \( \ds \oiint\limits_S \vec{j}\cdot\,d\vec{S} = \iiint\limits_V
    \div\vec{j}\,dV \),
а в левой: \( \ds W = \iiint_V w\cdot\,dV \). Таким образом, имеем:
\[
    \der{}{t}\iiint\limits_V w\,dV = \iiint\limits_V \kappa\, \div\grad u\,dV.
\]
В силу произвольности выбора элемента объема имеем:
\[
   \pder{w}{t} = \kappa\cdot\Delta u,
\]
где \( \Delta \) -- оператор Лапласа; \( w = \rho\cdot c\cdot u\), где
\( \rho \) -- плотность вещества, а \( c \) -- его удельная теплоемкость. Получаем:
\[
    \rho e\pder{u}{t} = \kappa\Delta u, \quad \text{или} \quad
    \pder{u}{t} = \frac{\kappa}{\rho c}\Delta u.
\]

Обозначая \( \cfrac{\kappa}{\rho c} = \alpha^2 \), получаем уравнение
теплопроводности: \( \ds \pder{u}{t} = \alpha^2\Delta u \), переходящее в
одномерном случае в
\[
    \pder{u}{t} = \alpha^2\ppder{u}{x},
\]
соответствующее распространению тепла в одномерном стержне с теплоизолированной
боковой поверхностью.

В случае, когда потери через боковую поверхность пропорциональны разнице
температур, уравнение принимает вид:
\[
    \pder{u}{t} = \alpha^2\ppder{u}{x} - \beta(u - u_0),
\]
где \( u_0 \) -- температура окружающей среды.

\newpage % ---------------------------------------------------------------------
