\chapter{Колебания балки (ДУЧП IV).}

Колебания балки описываются ДУЧП четвертого порядка вида
\[
    \ppder{u}{t} = -\alpha^2\pnder{4}{u}{x}.
\]

Рассмотрим колебания свободно опирающейся балки. Из-за того, что балка опирается
свободно, то во-первых \( u(0, t) = u(l, t) = 0 \), а во-вторых, так как концы
не изгибаются, \( \ds \ppder{u}{x}(0, t) = \ppder{u}{x}(l, t) = 0 \). Получаем
смешанную задачу:

\begin{minipage}{.43\textwidth}
\[
    \left. \begin{array}{rl}
        \text{ДУЧП:} & \ds \ppder{u}{t} = -\alpha^2\pnder{4}{u}{x}, \\[.4em]
        \text{ГУ:} & \left\{ \begin{array}{l}
            u(0, t) = u(l, t) = 0, \\
            \ds \ppder{u}{x}(0, t) = \ppder{u}{x}(l, t) = 0;
        \end{array} \right. \\[.4em]
        \text{НУ:} & \left\{ \begin{array}{l}
            u(x, 0) = \phi(x), \\
            \ds \pder{u}{t}(x, 0) = \psi(x); 
        \end{array} \right.
    \end{array} \right\}
\]
\end{minipage}
\hfill
\begin{minipage}{.47\textwidth}
    Воспользуемся методом разделения переменных и будем искать только
    периодические решения:
    \[
        u(x,t) = X(x)\cdot\bigl(a\cos\omega t + b\sin\omega t\bigr).
    \]
\end{minipage}

\vspace*{.4em}
Подставив в ДУЧП, получаем задачу Штурма-Лиувилля:
\[
    \left. \begin{array}{c}
    	\ds X^{(4)} - \frac{\omega^2}{\alpha^2} X = 0, \\
    	\left\{ \begin{array}{l}
    		X(0) = X(l) = 0, \\
    		X''(0) = X''(l) = 0;
    	\end{array} \right.
    \end{array} \right| \Rightarrow X_n = \sin\sqrt{\frac{\omega_n}{\alpha}}x,
    \text{ где } \omega_n = \alpha\left(\frac{\pi n}{l}\right)^2.
\]

Таким образом, \( \ds u(x, t) = \sum_{n=0}^\infty \left[a_n\cos\ 
\alpha\left(\frac{\pi n}{l}\right)^2t + b_n\sin\ 
\alpha\left(\frac{\pi n}{l}\right)^2t\right]\cdot
\sin\left(\frac{\pi nx}{l}\right) \).

Коэффициенты \( a_n \) и \( b_n \) определяются с помощью начальных условий:
\[
    a_n = \frac{2}{l} \int\limits_0^l \phi(x)\sin
    \left(\frac{\pi nx}{l}\right)\d x, \quad
    b_n = \frac{2l}{\alpha\pi^2 n^2} \int\limits_0^l \psi(x)\sin
    \left(\frac{\pi nx}{l}\right)\d x.
\]
\newpage
