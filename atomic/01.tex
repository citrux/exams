\chapter{Тепловое излучение. Закон Релея-Джинса. Гипотеза Планка. 
Кванты излучения. Функция Кирхгофа. Законы Стефана-Больцмана и 
смещения Вина.}

\section{Тепловое излучение. Основные понятие.}
Всё излучение делится на два типа: тепловое излучение и люминесценцию.

Тепловое излучение -- излучение за счет внутренней энергии тела, оно есть
всегда.

Люминесценция -- излучение за счет других видов энергии:
\begin{itemize}
    \item хемилюминесценция -- за счет энергии химических реакций;
    \item электролюминесценция -- за счет воздействия электрическим полем;
    \item фотолюминесценция -- за счет внешнего облучения электромагнитными
        волнами.
\end{itemize}

Основные свойства теплового излучения:
\begin{enumerate}
    \item из всех видов излучения является единственным равновесным (сколько
        энергии в виде электромагнитных волн излучается с единицы площади
        в единицу времени, столько же и поглощается). Это обусловлено тем,
        что интенсивность теплового излучения возрастает с увеличением
        температуры;
    \item немонохроматичность излучаемых электромагнитных волн;
    \item неполяризованность излучаемых электромагнитных волн.
\end{enumerate}

Количественные характеристики теплового излучения:
\begin{enumerate}
    \item энергетическая светимость \( R_T \) (интегральная испускательная
        способность) -- количество энергии излучаемое телом в виде
        электромагнитного излучения с единичной площадки за единицу времени:
        \[
            R_T = \der{W}{S\ dt};
            [R_T] = \left[ \frac{\text{Дж}}{\text{м}^2\cdot\text{с}} \right] =
            \left[ \frac{\text{Вт}}{\text{м}^2} \right].
        \]
    \item спектральная испускательная способность \( r \) -- количество энергии
        излучаемое телом в виде электромагнитного излучения с единичной
        площадки за единицу времени в узкий частотный диапазон
        \( [\omega;\ \omega + d\omega] \).
        \[
            r_\omega = \der{W}{S\ dt\ d\omega} = \der{R}{\omega};
            \quad r_\lambda = \der{R}{\lambda}.
        \]
        \[
            r_\omega\ d\omega = r_\lambda\ d\lambda,\quad r_\omega =
            r_\lambda\der{\lambda}{\omega} = \frac{2\pi c}{\omega^2}r_\lambda.
        \]
    \item поглощающая способность -- безразмерная величина, равная отношению
        поглощенной телом энергии к падающей на это тело энергии в виде
        электромагнитных волн в узком частотном интервале
        \( [\omega;\ \omega + d\omega] \).
        \[
            a_\omega = \der{W_\textit{полг}}{W_\textit{пад}} = [0, \ldots, 1].
        \]
        Если \( a = 1 \) при любой температуре, то такое тело называется
        \emph{абсолютно черным}. Если какая-либо величина описывает это тело,
        то к ней добавляется индексом символ \( ^* \): \( a^*(\omega, T) = 1 \).
    
    Если \( a = 0 \) при любой температуре, то такое тело называется
    \text{абсолютно белым}.
    
    \text{Абсолютно серыми} называются тела, которые имеют постоянный
    коэффициент поглощения во всем частотном интервале:
    \( a^\textit{сер}(\omega, T) = const < 1 \).
\end{enumerate}

\section{Законы теплового излучения}

Закон Кирхгофа -- отношение испускательной к поглощательной способности не
зависит от природы тела и является функцией частоты и температуры, причем эта
функция одинакова для всех тел.

\[
    \left(\frac{r(\omega, T)}{a(\omega, T)}\right)_1 =
    \left(\frac{r(\omega, T)}{a(\omega, T)}\right)_2 =
    \ldots = r^*(\omega, T) = f(\omega, T).
\]

Из этого закона есть два следствия:
\begin{enumerate}
    \item если на какой-то частоте тело излучает больше, то на этой частоте тело
        и поглощает больше:
        \[
            r_1 > r_2;\, a_1 = a_2\frac{r_1}{r_2} > a_2.
        \]
    \item среди всех тел наибольшей испускательной способностью обладает
        абсолютно черное тело:
        \[
            r(\omega, T) = r^*(\omega, T)\cdot a(\omega, T) < r^*(\omega, T).
        \]
\end{enumerate}

Экпериментально был получен вид этой функции:
% график %
С помощью него можно установить 2 закона теплового излучения:
\begin{enumerate}
    \item закон Стефана-Больцмана -- интегральная испускательная способность
        прямо пропорциональна 4-ой степени абсолютной температуры тела;
    \item закон смещения Вина: длина волны, на которую приходится максимум
        спектральной испускательной способности абсолютно чёрного тела обратно
        пропорциональна абсолютной температуре тела.
\end{enumerate}

\section{Закон Релея-Джинса}
Релеем и Джинсом была предпринята попытка определить вид функции
\( r^*(\omega, T) \). Для этого они рассмотрели кубическую полость с абсолютно
чёрными стенками. Для начала, они нашли связь между объёмной плотностью энергии
излучения и излучательной способностью фрагмента стенки:
\begin{gather*}
    dj = cu\frac{d\Omega}{4\pi},\\
    d\Phi = \vec{dj}\cdot\vec{dS} = \frac{cu}{4\pi}\cos\theta\d\Omega\,dS =
    \frac{cu}{4\pi}\cos\theta\sin\theta\d\theta\d\phi\,dS,\\
    d\Phi = \frac{cu}{4\pi}\,dS\int\limits_0^\frac{\pi}{2}\sin\theta\cos\theta
    \d\theta\int\limits_0^{2\pi}\d\phi = \frac{cu\,dS}{4},\\
    r^*(\omega, T) = \frac{d\Phi(\omega, T)}{dS} = \frac{cu(\omega, T)}{4}.
\end{gather*}
Далее они рассмотели стенки полости как совокупность классических гармонических
осцилляторов, которые могут обмениваться энергией с излучением в полости.
Рассмотрим объёмную плотность энергии теплового излучения в полости со стороной
\( l \):
\[
    u(\omega, T) = \frac{dW}{l^3\d\omega} =
    \frac{dN\midnum{W(\omega, T)}}{l^3\d\omega},
\]
где \( dN \) -- число волн в спектральном диапазоне
\( [\omega, \omega+d\omega] \). Для образования стоячих волн должны выполняться
нулевые граничные условия, то есть
\[
    lk = \pi n,\ n_x = \frac{l}{\pi}k_x,\ n_y = \frac{l}{\pi}k_y,\ 
    n_z = \frac{l}{\pi}k_z.
\]
Число волн \( dN \), у которых \( k\in[k, k+dk] \) равно числу целых чисел в 
интервале \( [n, n+dn] \):
\[
    dN = dn_x\,dn_y\,dn_z = \frac{L^3}{\pi^3}\,dk_x\,dk_y\,dk_z =
    2\frac{V}{\pi^3}\frac{4\pi k^2\,dk}{8} = \frac{V}{\pi^2}k^2\,dk =
    \frac{V}{\pi^2}\frac{\omega^2}{c^3}\,d\omega.
\]
Отсюда
\[
    u(\omega, T) =
    \frac{V}{\pi^2}\frac{\omega^2}{c^3}\frac{\midnum{W}}{V} =
    \frac{\omega^2}{\pi^2c^3}\midnum{W}.
\]
Далее Релей и Джинс предположили, что \( \midnum{W} = kT \) согласно
теореме о равномерном распределении энергии по степеням совбоды. Таким образом,
был получен закон Релея-Джинса:
\[
    r^*(\omega, T) = \frac{\omega^2}{4\pi^2c^2}kT.
\]
Однако, полученный вид закона хорошо описывал экспериментальную зависимость лишь
при малых частотах. К тому же, он не объяснял ни закона Стефана-Больцмана, ни
закона смещения Вина.
\section{Гипотеза Планка}
В 1900 году Макс Планк, понимая, что закон Релея-Джинса не содержит логических
ошибок, выдвинул гипотезу о том, что ЭМВ излучаются порциями, или квантами.
Энергия одного такого кванта зависит от частоты \( W_1 = \hbar\omega \).
Тогда осцилляторы могут иметь энергии, кратные энергии кванта:
\( W_n = n\hbar\omega \). Так как в стационарном состоянии энергия описывается
распределением Больцмана, то число осцилляторов с энергией \( W_n \) равно
\( N_n = A\cdot\exp(-W_n/kT) = A\cdot\exp(-n\hbar\omega/kT) \). Усредняя
энергию, получим
\[
    \midnum{W} =
    \frac{\sum\limits_{n=1}^\infty N_nW_n}{\sum\limits_{n=1}^\infty N_n} = 
    \frac{\sum\limits_{n=1}^\infty A\cdot\exp(-\frac{n\hbar\omega}{kT})\cdot
    n\hbar\omega}{\sum\limits_{n=1}^\infty A\cdot\exp(-\frac{n\hbar\omega}{kT})} =
    \frac{\sum\limits_{n=1}^\infty n\exp(-\frac{n\hbar\omega}{kT})}
    {\sum\limits_{n=1}^\infty \exp(-\frac{n\hbar\omega}{kT})}\hbar\omega =
    \frac{\hbar\omega}{\exp(\frac{\hbar\omega}{kT}) - 1}.
\]
Тогда для спектральной излучательной способности абсолютно чёрного тела имеем
\[
    r^*(\omega, T) = \frac{\omega^2}{4\pi^2c^2}
    \frac{\hbar\omega}{\exp(\frac{\hbar\omega}{kT}) - 1} = \frac{\hbar\omega^3}
    {4\pi^2c^2}\frac{1}{\exp(\frac{\hbar\omega}{kT}) - 1}.
\]
Полученная Планком формула хорошо описывает экспериментальную зависимость и из
неё следуют закон Стефана-Больцмана (интегрированием) и закон смещения Вина
(дифференцированием).
\newpage
