\chapter{Молекулярные спектры. Комбинационное рассеяние света.}

\section{Молекулярные спектры}
В отличии от атомных спектров состоящих из отдельных линий, молекулярные 
спектры состоят из отдельных полос, каждый из которых состоит из большого 
числа тесно расположенных линий (отсюда пошло название полосатые спектры). 
В зависимости от того изменение каких либо энергий обуславливается 
испускание молекулой фотона различают три вида полос:
\begin{enumerate}
    \item электронно-колебательные полосы
    \item вращательные полосы
    \item колебательно-вращательные полосы
\end{enumerate}

При электронном переходе \( E'_e \Rightarrow E''_e \) изменяется 
электронная конфигурация оболочки, следовательно изменяется сила 
действующая между ядрами, следовательно меняется и колебательная и 
вращательная энергии. То есть при электронном переходе меняются все три 
составляющие энергии и вместо одной линии соответствующую переходу 
\( E'_e \Rightarrow E''_E \) появляется целая полоса частот. Как 
показывает измерение электронные переходы совершаются за время 
\( t ~ 10^{-15} \) сек., а характерные периоды колебаний ядер молекулы 
имеют \( ~ 10^{-10}, 10^{-15} \). Этот факт является основой принципа 
Франка-Кондона: \emph{электронный переход происходит наиболее вероятно без 
изменения положений ядер молекулы.} 

(рис. 16_1)

Вероятность нахождения атома в данной точке пространства пропорционален 
времени нахождения в окрестности этой точки и соответственно обратно 
пропорционален скорости движения. В окрестности точки поворота, скорости 
значительно меньше, чем в центре, поэтому больше времени атомы в молекулах 
проводят в конфигурации, когда полная энергия практически равна 
потенциальной, а кинетическая примерно равна нулю. Поэтому вероятность 
испускания или поглощения фотона наибольшее тогда, когда ядры неподвижны 
или движутся медленно.

Рассмотрим два электронных перехода: (рис 16_2).

При электронных переходах:
\begin{enumerate}
    \item изменяется электронная конфигурация атома, 
    входящего в молекулу; 
    \item изменяются всех три составляющих энергии молекулы; 
    \item спектр очень <<сложный>>;
    \item наибольшей интенсивностью обладают линии подчиняющиеся принципу 
    Франка-Кондана;
\end{enumerate}

Вращательная полоса возникает при переходах между вращательными уровнями 
(при этом электронная конфигурация и энергия колебаний не изменяется)
\[ 
    \hbar\omega = \Delta E_r = \frac{\hbar^2}{2I} J'(J'+1) - 
    \frac{\hbar^2}{2I} J(J+1)
\]
\[ J' = J + 1 \text{ (в случае испускании фотона)} \]
\[ 
    \omega = \frac{\hbar}{2I}(J+1)(J+2) - \frac{\hbar}{2I}J(J+1) = 
    \frac{\hbar}{I}(J+1)
\]
\[ 
    J = 0, 1, 2, ...
\]
Частоты соответствующие этим квантовым числам соответственно:
\[ 
    \omega_1 = \frac{\hbar}{I}; \quad
    \omega_2 = \frac{2\hbar}{I} = 2\omega_1; \quad
    \omega_3 = \frac{3\hbar}{I} = 3\omega_1; ...
\]

Вращательный спектр состоит из ряда равноотстоящих линий расположенных в 
далёкой инфракрасной области. Измеряя расстояние между линиями мы можем 
определить момент инерции молекулы.

Колебательно-вращательные полосы возникают при изменении колебательной и 
вращательной энергии молекулы (в пределах данной электронной конфигурации 
существует правило отбора \( \Delta v = \pm 1 \).
\[ 
    \hbar\omega = \Delta E_v + \Delta E_R = 
    \left(v' + \frac{1}{2}\right)\omega_v\hbar - 
    \left(v + \frac{1}{2}\right)\omega_v\hbar + 
    \frac{\hbar^2}{2I}J'(J'+1) - \frac{\hbar^2}{2I}J(J+1)
\]

Рассмотрим процесс излучения фотона:
\[ 
    v' = v + 1; \quad
    J' = J \pm 1
\]
\begin{enumerate}
    \item \( J' = J + 1 \)
        \[ \omega = \omega_v + \frac{\hbar}{I}(J+1) \]
        \[ 
            J = 0, 1, 2, ...; \quad
            \Delta\omega_r = \omega_1, \omega_2, \omega_3, ...
        \]
    \item \( J' = J - 1 \)
        \[ 
            \Delta\omega_r = \frac{\hbar}{2I}
            \left(J(J-1) - J(J+1)\right) = -\frac{\hbar}{I}J
        \]
        \[ J = 1, 2, ... \]
        \[ \Delta\omega_r = -\omega, -2\omega, -3\omega \]
        \[ \omega = \omega_v \pm \frac{\hbar}{I}k,\quad k = 1, 2, 3 ... \]
\end{enumerate}

(рис. 16_3)

Колебательная вращательная полоса состоит из совокупности симметричных 
относительно \( \omega_v \) линий отстоящих друг от друга на расстояние 
\( \Delta\omega = \omega_1 = \hbar/I \).

Заметим, что вращательная и колебательно-вращательная полосы наблюдаются 
на опыте, только для не симметричных молекул, то есть для молекул из 
различных атомов. У симметричных молекулы дипольный момент равна нулю, 
что приводит к запретам колебательных и вращательных переходов.

\section{Комбинационное рассеяние света}
Комбинационное рассеяние света или КРС -- это явление заключается в том, 
что в спектре рассеяние возникают при прохождении света через газы, 
жидкости или прозрачные кристаллические тела, помимо первоначальной линии 
появляются новые линии, частоты которых представляют собой комбинации 
частоты падающего света \( \omega_0 \) и частоты \( \omega_i \) 
колебательной или вращательной частот рассеивающих молекул.
\[ \omega = \omega_0 \pm \omega_i \]
\[ 
    \omega = \omega_0 - \omega_i, \lambda = \lambda_0 + \lambda_i 
    \text{ -- красные спутники}
\]
\[ 
    \omega = \omega_0 + \omega_i, \lambda = \lambda_0 - \lambda_i
    \text{ -- фиолетовые спутники}
\]

\emph{Природа возникновения: } это процесс неупругого столкновения фотонов 
с молекулами, при соударении фотон может отдать молекуле или получить от 
неё, только такое количество энергии, которые равны разностям двух 
энергетическим уровням. Если при столкновении энергия молекулы 
увеличивается на \( \Delta E = E'' - E' \), то энергия фотона станет 
равной \( E_0 - \Delta E \) соответственно частота фотона уменьшится на 
величину \( \Delta E / \hbar \) -- возникает красный спутник. Если же 
молекула находилась в возбужденном состоянии с энергией \( E'' \), то 
она в результате столкновения может перейти в \( E' \) и отдать избыток 
энергии фотону -- фиолетовый спутник.

Рассеяние фотона сопровождается переходами молекулы между различными 
колебательными и вращательными уровнями в результате возникает ряд 
симметрично расположенных спутников. 

При обычных температурах согласно распределению Больцмана, число молекул 
в возбужденных состояниях значительно меньше, чем в основном поэтому в 
основном будут происходить процессы поглощения энергии и значит 
интенсивность красных спутников будет выше. При высоких температурах 
они сравняются.

Спектры КРС настолько характерны для молекул, что с их помощью осуществляют 
анализ сложных молекулярных смесей, анализ химическими методами 
которых сильно затруднён или невозможен.


\newpage
