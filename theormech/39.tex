\chapter{Уравнение неразрывности. Вид уравнения неразрывности в декартовой и
криволинейных координатах.}

В основу вывода уравнения неразрывности положим гипотезу, что масса \( m \)
частицы объёмом \( V \) при движении (деформации, вращении, поступательном
движении) не меняется.
\[
    m = \rho V,
\]
где \( \rho \) -- плотность.
\begin{gather*}
    \der{m}{t} =V \der{\rho }{t}+ \rho \der{V}{t} = 0,\\
    \der{\rho }{t}+ \frac{1}{V} \rho \der{V}{t} = 0,
\end{gather*}
но \( dV \) есть изменение объёма частицы, а так как
\( \nabla_is^i = \frac{\delta V}{V} \), где \( \delta V \) есть изменение
объёма частицы, то:
\begin{gather*}
    \der{\rho }{t}+ \rho \frac{\nabla_is^i}{dt} = 0, \\ 
    \der{\rho }{t}+ \rho {\nabla_iv^i } = 0.
\end{gather*}

Последнее уравнение и есть уравнение неразрывности в криволинейных координатах.
Учитывая определение ковариантной производной и полной производной получим
другие формы уравнения неразрывности:
\begin{gather*}
    \pder{\rho }{t} + 
    \pder{\rho }{q^i} v^i + 
    \rho \pder{v^i }{q^i} \rho  + 
    \rho v^k \Gamma^i_{ik} = 0 \\
    \pder{\rho }{t} + 
    \pder{(\rho v^i) }{q^i}   + 
    \rho v^i \Gamma^k_{ki} = 0 \\
    \pder{\rho }{t} + 
    \nabla_i(\rho v^i) = 0
\end{gather*}

В декартовых координатах \( x^i \) ковариантную производную нужно заменить
частной производной:
\begin{gather*}
    \pder{\rho }{t} + 
    \pder{(\rho v^i)}{x^i} = 0 \\
    \der{\rho }{t} + 
    \rho\pder{ v^i}{x^i} = 0    
\end{gather*}       


\newpage
