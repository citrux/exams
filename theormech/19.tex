\chapter{Циклические координаты и циклические интегралы. Связь циклических
интегралов с законами сохранения.}

Обобщенные координаты, которые не входят явно в выражение функции Лагранжа
\( L = L(q_1, \ldots, q_s, \dot{q}_1, \ldots, \dot{q}_s, t) \), называются
\emph{циклическими координатами}. В силу уравнения Лагранжа из
\( \ds \pder{L}{q_k} = 0 \) следует
\( \ds \der{}{t}\left(\pder{L}{\dot{q}_k}\right) = 0 \), откуда
\( \ds \pder{L}{\dot{q}_k} = C_k = \const \). Данные равенства называются
\emph{циклическими интегралами}.

\section{Связь циклических интегралов с законами сохранения}

Рассмотрим прямолинейное движение материальной точки массой \( m \) со скоростью
\( v \). Полагаем потенциальную энергию равной нулю, тогда функция Лагранжа:
\[
    L = T = \frac{mv^2}{2} = \frac{m\dot{x}^2}{2}.
\]
Координата \( x \) является циклической, а соответствующий ей циклический
интеграл имеет вид:
\[
    \pder{T}{\dot{x}} = m\dot{x} = mv = \const.
\]
Это равенство выражает закон сохранения количества движения.

Рассмотрим движение материальной точки массой \( m \) под действием центральной
силы в полярных координатах. Определим кинетическую энергию:
\[
    T = \frac{mv^2}{2} = \frac{m}{2}(v_r^2 + v_\phi^2) = \frac{m}{2}(\dot{r}^2 +
    r^2\dot{\phi}^2).
\]

Потенциальная энергия является функцией \( r \): \( \varPi = f(r) \). Тогда
функция Лагранжа:
\[
    L = T - \varPi = \frac{m}{2}(\dot{r}^2 + r^2\dot{\phi}^2) - f(r).
\]
Угловая координата \( \phi \) является циклической. Соответствующий ей
циклический интеграл имеет вид:
\[
    \pder{L}{\dot{\phi}} = mr^2\dot{\phi} = \const, \quad \text{или} \quad
    mrv_\phi = \const.
\]
Это равенство выражает закон сохранения момента количества движения материальной
точки относительно центра.

\newpage
