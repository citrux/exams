\chapter{Кинетический момент твердого тела относительно неподвижной точки.
Кинетическая энергия твердого тела, совершающего сферическое движение.
Динамические уравнения Эйлера.}

Рассмотрим выражение кинетического момента твердого тела в случае его
сферического движения:
\[
`\vec{K}_O = \sum_{k=1}^N\vec{r}_k\times m_k\vec{v}_k,
\]
где \( \vec{r}_k \) -- радиус-вектор точки \( M_k \) тела относительно начала
координат. Учитывая, что скорость точки равна \( \vec{\omega}\times\vec{r}_k \),
имеем:
\[
    \vec{K}_O = \sum_{k=1}^N\vec{r}_k\times m_k\vec{v}_k =
    \sum_{k=1}^N \vec{r}_k\times m_k(\vec{\omega}\times\vec{r}_k).
\]

Используя формулу для двойного векторного произведения, получим:
\[
    \vec{K}_O = \sum_{k=1}^N m_k r_k^2\vec{\omega} - \sum_{k=1}^N m_k
    (\vec{\omega}\cdot\vec{r}_k)\vec{r}_k = \sum_{k=1}^N m_k(x_k^2 + y_k^2 +
    z_k^2)\vec{\omega} - \sum_{k=1}^N m_k (\omega_k x_k + \omega_k y_k + \omega
    z_k)\vec{r}_k,
\]
где \( \omega_x,\ \omega_y,\ \omega_z \) -- проекции мгновенной угловой скорости
\( \vec{\omega} \) тела на оси \( x_k,\ y_k,\ z_k \) соответственно. Отсюда
получаем следующее выражение для проекций вектора \( \vec{K}_O \) на оси
\( Ox,\ Oy \) и \( Oz \):
\[
    \left\{ \begin{array}{l}
        \ds K_{O_x} = \sum_{k=1}^N m_k(y_k^2 + z_k^2)\omega_x - \sum_{k=1}^N m_k
        x_k y_k \omega_y - \sum_{k=1}^N m_k x_k z_k \omega_z, \\[.4em]
        \ds K_{O_y} = -\sum_{k=1}^N m_k x_k y_k \omega_x + \sum_{k=1}^N m_k
        (x_k^2 + z_k^2)\omega_y  - \sum_{k=1}^N m_k y_k z_k \omega_z, \\[.4em]
        \ds K_{O_z} = -\sum_{k=1}^N m_k x_k z_k \omega_x - \sum_{k=1}^N m_k y_k
        z_k \omega_y + \sum_{k=1}^N m_k (x_k^2 + z_k^2)\omega_z.
    \end{array} \right.
\]
Величины
\[
    I_x = \sum_{k=1}^N m_k(y_k^2 + z_k^2), \quad I_y = \sum_{k=1}^N m_k(x_k^2 +
    z_k^2), \quad I_z = \sum_{k=1}^N m_k(x_k^2 + y_k^2)
\]
называют осевыми моментами инерции, а величины
\[
    I_{xy} = \sum_{k=1}^N m_k x_k y_k, \quad I_{yz} = \sum_{k=1}^N m_k y_k z_k,
    \quad I_{xz} = \sum_{k=1}^N m_k x_k z_k
\]
называют центробежными моментами инерции. Тогда:
\[
    \left\{ \begin{array}{l}
        K_{O_x} = I_x\omega_x - I_{xy}\omega_y - I_{xz}\omega_z, \\
        K_{O_y} = -I_{xy}\omega_x + I_y\omega_y  - I_{yz}\omega_z, \\
        K_{O_z} = -I_{xz}\omega_x - I_{yz}\omega_y + I_z\omega_z.
    \end{array} \right.
\]

Эти формулы можно записать более компактно: \( \vec{K}_O = \hat{I}\vec{\omega}
\). Кинетическую энергию можно представить как кинетическую энергию тела,
вращающегося вокруг оси, параллельной \( \omega \):
\[
    T = \frac{1}{2}\vec{K}_O\cdot\vec{\omega}.
\] 

Для подвижной системы координат, связанной с телом имеем следующий вид правила
моментов:
\[
    \der{\vec{K}_O}{t} + \vec{\omega} \times\vec{K}_O = \vec{M}^e_O.
\]

Расписав векторное произведение
\[
    \vec{\omega}\times\vec{K}_O = \vec{e}_x(\omega_y K_z - \omega_z K_y) +
    \vec{e}_y(\omega_z K_x - \omega_x K_z) + \vec{e}_z(\omega_x K_y -
    \omega_y K_x)
\]
и спроецировав на оси подвижной системы координат, получим:
\[
    \left\{ \begin{array}{l}
        \ds \der{K_x}{t} + (\omega_y K_z - \omega_z K_y) = M^e_x, \\[.4em]
        \ds \der{K_y}{t} + (\omega_z K_x - \omega_x K_z) = M^e_y, \\[.4em]
        \ds \der{K_z}{t} + (\omega_x K_y - \omega_y K_x) = M^e_z.
    \end{array} \right.
\]

Если подвижную систему координат выбрать таким образом, чтобы координатные оси
\( x,\ y,\ \) были главными осями инерции для точки \( O \), то все центробежные
моменты инерции будут равны нулю:
\[
    \left\{ \begin{array}{l}
        \ds I_x\der{\omega_x}{t} + (I_z -  I_y)\omega_y\omega_z = M^e_x, \\[.4em]
        \ds I_y\der{\omega_y}{t} + (I_x -  I_z)\omega_z\omega_x = M^e_y, \\[.4em]
        \ds I_z\der{\omega_z}{t} + (I_y -  I_x)\omega_x\omega_y = M^e_z.
    \end{array} \right.
\] 
Эти уравнения называются динамическими уравнениями Эйлера.

\newpage
