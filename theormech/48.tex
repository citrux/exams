\chapter{Кинетический момент точки и системы относительно центра и оси.
Кинетический момент вращающегося твердого тела относительно оси вращения.
Теорема об изменении кинетического момента для точки.}

Рассмотрим систему материальных точек с массами \( m_1 \), \( m_2 \), \ldots,
\( m_n \), имеющих в данный момент скорости \( v_1 \), \( v_2 \), \ldots,
\( v_n \) относительно инерциальной системы отсчета. Выберем произвольный центр
\( O \). Кинетическим моментом точки \( m_j \) относительно центра \( O \)
называется вектор момента её количества движения относительно этого центра:
\( \vec{K}_{O_j} = \vec{r}_j \times m_j\vec{v}_j \).

Известно, что векторное умножение можно записать через присоединенную матрицу
первого сомножителя -- радиуса вектора \( \vec{r} \). Опуская индекс, запишем
матричное выражение в осях \( xyz \) c началом в \( O \): \( K_O = mRv \), где
\( R \) -- антисимметричная присоединенная матрица вектора \( r \):
\[
    \begin{pmatrix} K_x \\ K_y \\ K_z \end{pmatrix} =
    m\begin{pmatrix} 0 & -z & y \\ z & 0 & -x \\ -y & x & 0 \end{pmatrix}
    \begin{pmatrix} \dot{x} & \dot{y} & \dot{z} \end{pmatrix} = 
    m\begin{pmatrix} y\dot{z} - z\dot{y} \\ z\dot{x} - x\dot{z} \\
    x\dot{y} - y\dot{x} \end{pmatrix}.
\]
 
Проекция кинетического момента на ось называются кинетическим моментом точки
относительно оси. Он вычисляется либо аналитически, либо как момент силы
относительно оси. Момент дает только касательная составляющая вектора
\( \vec{q} \): \( K_Z = \pm q_\tau h \).

Момент обращается в ноль, если вектор количества движения лежит в одной
плоскости с осью. Кинетическим моментом системы относительно центра \( O \)
называется главный момент количеств движений точек системы относительно этого
центра: \( \vec{K}_O = \sum \vec{K}_{O_j} = \sum m_j \vec{r}_j \times
\vec{v}_j \) или в матричной форме:
\[
    \begin{pmatrix} K_x \\ K_y \\ K_z \end{pmatrix} = \sum_j m_j
    \begin{pmatrix} y_j\dot{z}_j - z_j\dot{y}_j \\ z_j\dot{x}_j - x_j\dot{z}_j
    \\ x_j\dot{y}_j - y_j\dot{x}_j \end{pmatrix}
\]
Вычислим кинетический момент вращающегося вокруг оси тела. Пусть угловая
скорость тела будет \( \omega \), тогда для любой точки отстоящей от оси на
расстояние \( h_k \), скорость будет \( v_k = \omega h_k \), а момент
относительно \( Oz \):
\[
    \vec{m}_k(m_k\vec{v}_k) = m_k v_k h_k = m_k\omega h_k^2.
\]

Подставляя это соотношение в \( \vec{K}_X = \sum \vec{m}_X(m_k\vec{v}_k) \),
\( \vec{K}_Y = \sum \vec{m}_0(m_k\vec{v}_k) \), \( \vec{K}_Z = \sum \vec{m}_Z
(m_k\vec{v}_k) \) получаем:
\[
    K_Z = \sum \vec{m}_Z(m_k\vec{v}_k) = \sum (m_k h_k^2)\vec{\omega}
\] 
или используя формулу \( I_Z = \sum m_k h_k^2 \) (\( I \ge 0 \)):
\( \vec{K}_Z = I_Z\vec{\omega} \).
 
Кинетический момент вращающегося тела относительно оси вращения равен
произведению момента инерции тела относительно этой оси на угловую скорость
тела.

Если система состоит из нескольких тел, то:
\[
    \vec{K}_Z = I_{1Z}\vec{\omega}_1 + I_{2Z}\vec{\omega}_2 + \ldots +
    I_{NZ}\vec{\omega}_N = \sum_{i=1}^N I_{iZ}\vec{\omega}_i.
\]
 
Теорема моментов для одной точки \( \ds \der{}{t}\bigl[\vec{m}_0(m\vec{v})\bigr]
= \vec{m}_0(\vec{F}) \) справедлива для каждой точки системы. Для одной точки:
\[
    \der{}{t}\bigl[\vec{m}_0(m_k\vec{v}_k)\bigr] = \vec{m}_0(\vec{F}^e_k) +
    \vec{m}_0(\vec{F}^i_k).
\]
Для всех точек:
\begin{gather*}
    \der{}{t}\left[\sum\vec{m}_0(m_k\vec{v}_k)\right] =
    \sum\vec{m}_0(\vec{F}^e_k) + \sum\vec{m}_0(\vec{F}^i_k), \\
    \der{\vec{K_0}}{t} = \sum\vec{m}_0(\vec{F}^e_k).
\end{gather*}

\newpage
