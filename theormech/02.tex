\chapter{Общее уравнение динамики (принцип д'Аламбера-Лагранжа).}

Согласно \textit{принципу д'Аламбера} всякую материальную точку можно считать
покоящейся, если дополнить все действующие силы силой инерции.

Согласно  \textit{принципу Лагранжа} всякую несвободную материальную точку
можно рассматривать как свободную, если заменить связи силами реакций.

Если система состоит из \( n \) материальных точек со связями и находится в
силовых полях, то уравнение движения \( j \)-ой материальной точки с учётом
принципов д'Аламбера и Лагранжа имеет вид:
\[
    \underbrace{\vec{F}_{j}}_{
        \substack{\textbf{равнодействующая сил,} \\
        \textbf{обусловленных полями}}}
    + 
    \underbrace{\vec{N}_{j} }_{
            \substack{\textbf{равнодействующая} \\ \textbf{реакций связей}}
            }
    +
    \underbrace{\vec{\Phi}_{j}}_{
            \textbf{сила инерции}
            }   
    = 0.
\]

Домножим эти уравнения скалярно на виртуальные перемещения
\( \delta\vec{r}_{j} \) и просуммируем по \( j \). В силу произвольности выбора
виртуальных перемещений результат будет эквивалентен записанным выше
уравнениям:
\[
    \sum\limits_{j=1}^n \vec{F}_{j}\cdot\delta\vec{r}_{j}+
    \sum\limits_{j=1}^n \vec{N}_{j}\cdot\delta\vec{r}_{j}+
    \sum\limits_{j=1}^n \vec{\Phi}_{j}\cdot\delta\vec{r}_{j}
    = 0.
\]  
    
Вспоминая определение виртуальной работы получаем
\emph{общее уравнение динамики}:
\[
    \sum\limits_{j=1}^n \underbrace{\delta A^e_j}_{
        \substack{\textbf{работа} \\ \textbf{силовых полей}}    
        }
    + 
    \sum\limits_{j=1}^n\underbrace{ \delta A^N_j}_{
            \substack{\textbf{работа} \\ \textbf{сил реакции}}
            }
    +
    \sum\limits_{j=1}^n\underbrace{ \delta A^\text{и}_j}_{
            \substack{\textbf{работа} \\ \textbf{силы инерции}}
            }   
    = 0.
\]

Для случая идеальных связей \( \delta A^N_j = 0 \) и общее уравнение динамики
принимает вид:
\[
    \sum\limits_{j=1}^n \delta A^e_j +
    \sum\limits_{j=1}^n \delta A^\text{и}_j
    =0.
\]

\newpage % -------------------------------------------------------------------
