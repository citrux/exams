\chapter{Принцип наименьшего действия Мопертюи-Лагранжа в форме Якоби.}

Действие по Лагранжу
\[
    W = \int\limits_{0}^t 2T\,dt,
\]
где \( T \) - кинетическая энергия. Принцип наименьшего действия
Мопертюи-Лагранжа:
\[
    \Delta W = 0.
\]

Якоби дал другую формулировку принципа. Кинетическая энергия \( n \)
материальных точек
\[
    T = \frac{1}{2}\sum\limits_{i=1}^{n} m_i v_i^2 = 
    \frac{1}{2}\sum\limits_{i=1}^{n} m_i \left(\der{s_i}{t}\right)^2 = 
    \frac{1}{2}\sum\limits_{i=1}^{n} m_i \der{s_i^2}{t^2}.
\]
Откуда 
\[
    dt = \sqrt{\frac{\sum\limits_{i=1}^{n} m_i\,ds_i^2}{2T}} \text{ и }
    W = \int\limits_{A}^B \sqrt{{2\sum\limits_{i=1}^{n} m_i T\,ds_i^2}}.
\]
Принцип наименьшего действия в формулировке Лагранжа применим лишь к системам
с потенциальными силовыми полями:
\[
    W = \int\limits_{A}^B \sqrt{{2\sum\limits_{i=1}^{n} m_i {(h-\Pi)}\,ds_i^2}},
\]
где \( h \) -- полная энергия, \( \Pi = \Pi(q_1, \dots, q_s) \)~--
потенциальная энергия.

Если ввести представления о фундаментальном метрическом тензоре \( g_{ij} \),
как о тензоре, определяющем метрику криволинейного пространства 
(использовано правило Эйнштейна для суммирования):
\[
    ds_i^2 = g_{jk}dq_jdq_k,
\]
то 
\[
    W = \int\limits_{A}^B
    \sqrt{{2\sum\limits_{i=1}^{n} m_i (h-\Pi) g_{jk}\,dq_j\,dq_k}}
\]
и, если ввести новый метрический тензор
\[
    g_{jk}^* = g_{jk}\left(1-\frac{\Pi}{h}\right),
\]
то движение можно рассматривать в отсутствие сил в искривлённом пространстве.
То есть механика может быть бессиловой:
\[
    W = \int\limits_{A}^B
    \sqrt{{2\sum\limits_{i=1}^{n} m_i h g_{jk}^*\,dq_j\,dq_k}}.
\] 
\textbf{Принцип наименьшего действия в формулировке Якоби:}
\emph{действие \( W \) имеет стационарное значение по сравнению со всякими
другими близкими движениями между теми же самыми начальным и
конечным положениями и с тем же самым постоянным значением \( h \) энергии,
что и в действительном движении.}
\newpage
