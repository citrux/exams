\chapter{Обобщенные силы. Способы вычисления обобщенных сил. Понятие об
идеальных связях. Вычисление обобщенных сил в случае движения механической
системы в поле потенциальных сил.}

Предположим, что механическая система имеет \( s \) степеней свободы, то есть её
положение определяется \( s \) обобщенными координатами:
\( \vec{r}_j = \vec{r}_j(q_1, q_2, \ldots, q_s, t) \). Отношение работы
\( \delta A_{q_j} \) к приращению обобщенной координаты \( \delta q_j \) назовём
обобщенной силой, соответствующей координате \( q_j \) и обозначим \( Q_j \):
\[
    Q_j = \frac{\delta A_{q_j}}{\delta q_j} = \sum_{k=1}^N F_k\pder{r_k}{q_j}.
\]

\section{Способы вычисления обобщенных сил}
\begin{enumerate}
    \item Вычисление по определению:
    \[
        Q_j = \sum_{k=1}^N F_k\pder{r_k}{q_j} = \sum_{k=1}^N \left(F_{kx}
        \pder{x_k}{q_j} + F_{ky}\pder{y_k}{q_j} + F_{kz}\pder{z_k}{q_j}\right).
    \]
    
    \item Для вычисления обобщенной силы \( Q_1 \) дадим такое виртуальное
    перемещение, при котором все вариации обобщенных координат, кроме
    \( \delta q_1 \), равны нулю:\\
    \( \delta q_2 = \delta q_3 = \ldots = \delta q_s = 0 \),
    \( \delta q_1 \ne 0 \). Получим: \( \delta A_1 = Q_1\delta q_1 \).
    
    \item Если силы консервативны, будут справедливы равенства:
    \[
        F_{kx} = -\pder{\varPi}{x_k}, \quad F_{ky} = -\pder{\varPi}{y_k}, \quad
        F_{kz} = -\pder{\varPi}{z_k},
    \]
    где \( \varPi \) -- потенциальная энергия системы. Вычисляем обобщенные силы:
    \[
        Q_j = -\sum_{k=1}^N \left(\pder{\varPi}{x_k}\pder{x_k}{q_j} +
        \pder{\varPi}{y_k}\pder{y_k}{q_j} + \pder{\varPi}{z_k}\pder{z_k}{q_j}
        \right).
    \]
\end{enumerate}

\section{Понятие об идеальных связях}
Связи называются \emph{идеальными}, если сумма работ всех реакций связей на
любом виртуальном перемещении системы равна нулю. Обозначим через
\( \vec{R}_k \) равнодействующую всех реакций связей, приложенных к
\( k \)-точке. Тогда условие идеальности связей будет иметь вид:
\( \ds \sum_{k=1}^N \vec{R}_k\delta\vec{r}_k = 0 \).

\newpage
