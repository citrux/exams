\renewcommand{\labelenumi}{\Roman{enumi}}

\chapter{Основные законы динамики материальной точки. Законы Ньютона.
Дифференциальные уравнения движения материальной точки. Две основные задачи
динамики точки.}

В основе динамики материальной точки лежат три закона Ньютона:
\begin{enumerate}
    \item закон Ньютона (закон инерции):

    всякое тело сохраняет состояние покоя или равномерного прямолинейного
    движения, пока и поскольку приложенные силы не заставят его изменить это
    состояние;
    
    \item закон Ньютона:

    изменение движения пропорционально приложенной движущей силе и происходит
    в направлении линии действия этой силы;

    \item закон Ньютона:
    
    действию всегда соответствует равное ему и противоположно направленное
    противодействие, то есть действия двух тел друг на друга всегда равны и
    направлены в противоположные стороны.

\end{enumerate}
\renewcommand{\labelenumi}{\arabic{enumi}.}

Эти три закона предполагают существование ``абсолютного времени'' и установлены
для движений материальной точки по отношению к ``абсолютно неподвижной'' системе
координат, а согласно принципу Галилея -- и по отношению к любой инерциальной
(галилеевой) системе отсчета.

\emph{Дифференциальное уравнение движения} материальной точки в векторном виде:
\[
    m\dder{\vec{r}}{t} = \vec{F}.
\]

\section{Две основные задачи динамики материальной точки:}
\emph{Первая задача} (прямая задача):
    
дано движение материальной точки заданной массы, то есть известны координаты
точки как функции времени -- кинематические уравнения движения; требуется найти
силу, действующую на точку.

\emph{Вторая задача} (обратная задача):
    
дана сила, приложенная к материальной точке заданной массы; требуется найти
движение точки, то есть кинематические уравнения движения.

\newpage
