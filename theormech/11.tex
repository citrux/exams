\chapter{Физический смысл дивергенции вектора смещения. Понятие о тензоре
скоростей деформации.}
Определим физический смысл \( \nabla_is^i = \pder{s^i}{x^i} = \div\vec{s} =
\theta \). Рассмотрим малый параллелепипед со сторонами \( \delta x^1 \),
\( \delta x^2 \), \( \delta x^3 \), после деформации они станут:
\begin{gather*}
	\delta\hat{x}^1 = \left(1 + \pder{s^1}{x^1}\right)\delta x^1\\
	\delta\hat{x}^2 = \left(1 + \pder{s^2}{x^2}\right)\delta x^2\\
	\delta\hat{x}^3 = \left(1 + \pder{s^3}{x^3}\right)\delta x^3
\end{gather*}
сравнивая объёмы параллелепипеда до и после деформации \( V \) и
\( \hat{V} = V + \delta V \) получим:
\[
	\nabla_is^i = \frac{\delta V}{V},
\]
то есть дивергенция вектора смещения имеет смысл относительного изменения
объёма частицы сплошной среды.

Так как вектор смещения есть перемещение частицы сплошной среды за малый
промежуток времени \( dt \) тензор деформации есть бесконечно малая величина
порядка \( dt \). Отношение тензора деформации к \( dt \) после изменения
порядка интегрирования, даст \emph{тензор скоростей деформации}:
\[
	\frac{\nabla_is^j}{dt} = \nabla_iv^j,
\]
где \( v^j \) - скорость движения частиц сплошной среды, взятая по Эйлеру.
\newpage
