\chapter{Структура кинетической энергии в обобщенных координатах. Матрица
инерционных коэффициентов уравнений Лагранжа в случае стационарных связей.}

Запишем кинетическую энергию материальной системы:
\( \ds T = \frac{1}{2}\sum_{k=1}^N m_k v_k^2 \). Воспользуемся тождеством:
\( v_k^2 = \vec{v}_k\cdot\vec{v}_k \) и заменим вектор скорости точки
\( \vec{v}_k \) по формуле:
\[
    \vec{v}_k =\der{\vec{r}_k}{t} = \sum_{j=1}^s \pder{\vec{r}_k}{q_j}\dot{q}_j
    + \pder{\vec{r}_k}{t},
\]
тогда кинетическая энергия примет вид: \( \ds T = \frac{1}{2}\sum_{k=1}^N m_k
\left(\sum_{j=1}^s \pder{\vec{r}_k}{q_j}\dot{q}_j +
\pder{\vec{r}_k}{t}\right)^2 \).

Возведем скобку в квадрат и сгруппируем отдельно члены второй степени
относительно обобщенных скоростей \( \vec{q}_j \), члены первой степени
относительно тех же величин и члены, не содержащие скорости:
\begin{align*}
    T & = \frac{1}{2} \sum_{k=1}^N m_k\left( \sum_{i=1}^s \pder{\vec{r}_k}{q_i}
    \dot{q}_i + \pder{\vec{r}_k}{t} \right) \cdot \left( \sum_{j=1}^s
    \pder{\vec{r}_k}{q_j}\dot{q}_j + \pder{\vec{r}_k}{t} \right) = \\
    & =\frac{1}{2}\sum_{k=1}^N m_k\left( \sum_{i=1}^s\sum_{j=1}^s
    \pder{\vec{r}_k}{q_i}\pder{\vec{r}_k}{q_j}\dot{q}_i\dot{q}_j + 2\sum_{j=1}^s
    \pder{\vec{r}_k}{q_j}\pder{\vec{r}_k}{t}\dot{q}_j + \pder{\vec{r}_k}{t}
    \pder{\vec{r}_k}{t}\right).
\end{align*}

Обозначим \( \ds T_0 = \frac{1}{2}\sum_{k=1}^N m_k\pder{\vec{r}_k}{t}
\pder{\vec{r}_k}{t} \), \( \ds a_{ij} = \sum_{k=1}^N m_k \pder{\vec{r}_k}{q_i}
\pder{\vec{r}_k}{q_j} \), \( \ds b_j = \sum_{k=1}^N m_k \pder{\vec{r}_k}{q_j}
\pder{\vec{r}_k}{t} \), тогда кинетическую энергию можно записать в виде:
\( \ds T = \frac{1}{2} \sum_i\sum_j a_{ij}\dot{q}_i\dot{q}_j +
\sum_j b_j\dot{q}_j + T_0 \).

Обозначая \( \ds T_1 = \sum_j b_j\dot{q}_j \),
\( \ds T_2 = \frac{1}{2} \sum_i\sum_j a_{ij}\dot{q}_i\dot{q}_j \), получим
\( T = T_0 + T_1 + T_2 \).

Коэффициенты \( a_{ij} \) называются матрицей инерционных коэффициентов, причем
эта матрица является симметричной: \( a_{ij} = a_{ji} \).

В случае стационарных связей \( T_0 = T_1 = 0 \), \( T = T_2 \).

% было написано, что нет матрицы инерционных коэффициентов в случае стационарных
% связей

\newpage
