\chapter{Количество движения системы и способы его вычисления. Теорема об
изменении количества движения системы в дифференциальной и конечной
формах. Законы сохранения количества движения системы.}

Количеством движения системы материальных точек \( \vec{Q} \) называется
векторная сумма количеств движений отдельных точек системы:
\( \ds \vec{Q} = \sum_j m_j\vec{v}_j \).

Количество движения системы можно выразить через массу системы и скорость центра масс:
\[
    \vec{Q} = M\vec{v}_c.
\]

\section{Теорема об изменении количества движения системы}
Запишем теорему об изменении количества движения для \( i \)-точки:
\[
    \der{(m_j\vec{v}_j)}{t} = \vec{F}^e_j + \vec{F}^i_j.
\]

Сложив уравнения для всех точек системы, получим:
\[
    \der{\vec{Q}}{t} = \sum \vec{F}^e_j + \sum \vec{F}^i_j = \sum \vec{F}^e_j.
\]

Производная по времени от количества движения системы равна векторной сумме всех
внешних сил, действующих на систему:
\[
    \der{\vec{Q}}{t} = \sum \vec{F}^e_k.
\]

В проекциях на оси координат имеем:
\[
    \der{Q_x}{t} = \sum F^e_{kx}, \quad \der{Q_y}{t} = \sum F^e_{ky}, \quad
    \der{Q_z}{t} = \sum F^e_{kz}.
\]

\section{Законы сохранения количества движения}
\begin{enumerate}
    \item Если главный вектор всех внешних сил системы равен нулю:
    \( \sum \vec{F}^e_k = 0 \), то количество движения системы постоянно по
    величине и направлению: \( \vec{Q} = \const \).
    \item Если проекция главного вектора всех внешних сил системы на какую-либо
    ось равна нулю: \( \sum \vec{F}^e_{kx} = 0 \), то проекция количества
    движения системы на эту ось является постоянной величиной:
    \( Q_x = \const \).
\end{enumerate}

\newpage
