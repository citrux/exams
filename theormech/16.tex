\chapter{Основные законы (теоремы) механики для материальной точки. Теорема об
изменении момента импульса (момента количества движения, кинетического
момента). Закон сохранения момента импульса.}

Основными теоремами механики материальной точки являются:
\begin{enumerate}
    \item теорема об изменении количества движения (и соответствующий ей закон
    сохранения импульса);
    \item теорема об изменении момента количества движения (и соответствующий ей
    закон сохранения момента импульса);
    \item теорема об изменении кинетической энергии.
\end{enumerate}

Запишем основное уравнение динамики: \( \ds m\der{\vec{v}}{t} = \vec{F} \) и
умножим его векторно на радиус-вектор точки \( \vec{r} \), определяющий положение
материальной точки относительно какой-либо точки \( O \), которую будем называть
центром:
\[
    \vec{r}\times m\der{\vec{v}}{t} = \vec{r}\times\vec{F}.
\]

Преобразуем левую часть:
\( \ds \vec{r}\times m\der{\vec{v}}{t} = \der{}{t}(\vec{r}\times m\vec{v}) \).
Зная, что \( \ds\der{\vec{r}}{t} = \vec{v} \) и векторное произведение
параллельных векторов \( \vec{v}\times m\vec{v} \) равно нулю, получим:
\[
    \vec{v}\times m\der{\vec{v}}{t} = \der{}{t}(\vec{r}\times m\vec{v}), \quad
    \text{откуда:} \quad
    \der{}{t}(\vec{r}\times m\vec{v}) = \vec{r}\times\vec{F}.
\]

Вектор \( \vec{K}_O = \vec{r}\times m\vec{v} \) называется моментом количества
движения материальной точки относительно центра (точки \( O \)). Вектор
\( \vec{M}_O = \vec{r}\times\vec{F} \) -- момент приложенной к точке силы
относительно центра.

Таким образом: \( \ds \der{\vec{K}_O}{t} = \vec{M}_O \). Это уравнение выражает
теорему об изменении момента количества движения материальной точки: производная
по времени от момента количества движения материальной точки относительно
какого-либо центра равна моменту силы, приложенной к точке, относительно того же
центра.

Если сумма моментов приложенных сил равна нулю \( \sum\vec{M}_O = 0 \), то
момент количества движения постоянен \( \vec{K}_O = \const \) и движение
происходит в одной плоскости -- закон сохранения момента импульса.

\newpage % ---------------------------------------------------------------------
