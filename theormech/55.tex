\chapter{Теорема об изменении кинетической энергии материальной точки и системы
в дифференциальной и конечной формах. Случай абсолютно твердого тела.}

\section{Теорема об изменении кинетической энергии материальной точки и системы
в дифференциальной и конечной формах}

Рассмотрим материальную точку с массой \( m \), перемещающуюся из положения \( M_0 \), 
где она имеет скорость \( v_0 \), в положение \( M_1 \), где её скорость \( v_1 \).

Для получения искомой зависимости обратимся к выражающему основной закон динамики 
уравнению \( m\vec{a} \sum\vec{F_k} \). Проектируя обе его части на касательную 
\( M\tau \) к траектории точки \( M \), направленную в сторону движения, получим
\[ ma_\tau = \sum F_{k\tau} \]

Входящее сюда касательное ускорение точки представим в виде
\[ a_\tau = \frac{dv}{dt} = \frac{dv}{ds}\frac{ds}{dt} = v\frac{dv}{ds} \]

В результате найдём, что
\[ mv\frac{dv}{ds} = \sum F_{k\tau} \]

Умножим обе части этого равенства на \( ds \) и внесём \( m \) под знак 
дифференциала. Тогда, замечая, что \( F_{k\tau}ds = dA_k \), где 
\( dA_k \) -- элементарная работа силы \( \vec{F_k} \), получим выражение 
\emph{теоремы об изменении кинетической энергии точки в дифференциальной 
форме}:
\[ d\left( \frac{mv^2}{2} \right) = \sum dA_k \]

Проинтегрируем теперь обе части этого равенства в пределах, соответствующих 
значениям переменных в точка \( M_0 \) и \( M_1 \), найдём окончательно:
\[ \frac{mv^2_1}{2} - \frac{mv^2_0}{2} = \sum A_{(M_0 M_1)} \]

Последнее уравнение выражает теорему об изменении кинетической 
энергии точки в конечном виде: \emph{изменение кинетической энергии точки при 
некотором её перемещении равно алгебраической сумме работ всех действующих 
на точку сил на том же перемещении}.

Теорема об изменении кинетической энергии для точки будет справедлива 
для любой точки системы
\[ d\left( \frac{mv^2}{2} \right) = dA^e_k + dA^i_k \]
Составляя такие уравнения для всех точек системы и складывая их 
почленно получим:
\[ d\left( \frac{mv^2}{2} \right) = \sum dA^e_k + \sum dA^i_k \]
проинтегрировав получаем:
\[ 
	\frac{mv^2_1}{2} - \frac{mv^2_0}{2} = 
	\sum A^e_k + \sum A^i_k 
\]
теорема об изменении кинетической энергии в конечном виде: 
\emph{изменение кинетической энергии системы при некотором её 
конечном перемещении равно сумме работ на этом перемещении всех 
приложенных к системе внешних и внутренних сил}.

\section{Случай абсолютно твердого тела}

Работа внутренних сил в абсолютно твёрдом теле равна нулю:
\[ \sum A^i_k = 0\]
следовательно, формулу можно переписать в виде:
\[ \frac{mv^2_1}{2} - \frac{mv^2_0}{2} = \sum A^e_k \] 

\newpage
