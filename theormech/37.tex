\chapter{Понятие об обобщенных координатах. Понятие о связях. Классификация
связей. Возможные (виртуальные) перемещения. Число степеней свободы
механической системы.}

Перемещения точек несвободной механической системы не могут быть совершенно
произвольными, так как они ограничены имеющимися связями. При таком условии
положение точек системы определяется заданием только независимых координат.
Обобщенные координаты системы -- независимые параметры любой физической
размерности, однозначно определяющие положение механической системы в
пространстве.

\section{Понятие о связях}

По Ньютону связи -- это тела, которые ограничивают движение других тел.

По Лагранжу связи -- это условия, которые ограничивают движение рассматриваемых
механических систем в пространстве.

\emph{Возможные перемещения} -- бесконечно малые, воображаемые перемещения,
допускаемые связями в данный момент времени.

\emph{Числом степеней свободы механической системы} \( S \) называется число
возможных перемещений. В общем случае, если механическая система состоит из
\( N \) точек, то \( S~=~3N \).

\section{Классификация связей}
Связи делятся на удерживающие (двусторонние) и неудерживающие (односторонние).
В свою очередь, удерживающие связи делятся на голономные (интегрируемые,
геометрические) и неголономные (дифференциальные, кинематические).

Так же все связи можно разделить на зависящие от времени (реономные) и
независящие от времени (склерономные).

\newpage
