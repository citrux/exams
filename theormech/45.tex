\chapter{Количество движения точки и системы. Элементарный и полный импульс
силы. Теорема об изменении количества движения точки в дифференциальной и
конечной формах. Теорема импульсов.}

Количеством движения материальной точки называется векторная физическая
величина, равная произведению массы точки на её скорость:
\[
    \vec{q} = m\vec{v}.
\]

Для механической системы количество движения -- это сумма количеств движения
каждой материальной точки, входящей в систему:
\[
    \vec{Q} = \sum_j \vec{q}_j = \sum_j m_j\vec{v}_j.
\]

Элементарный импульс силы есть \( \d\vec{S} = \vec{F}\d t \). Проинтегрировав,
получим полный импульс силы:
\[
    \vec{S} = \int\limits_{t_1}^{t_2} \vec{F}\d t.
\]

\section{Теорема об изменении количества движения}
Изменение количества движения точки равно импульсу равнодействующей сил,
действующих на точку:
\begin{align*}
    \d\vec{q} = \vec{F}\d t & \text{ -- дифференциальный вид}, \\
    \vec{Q}_2 - \vec{Q}_1 = \int\limits_{t_1}^{t_2} \vec{F}\d t = \vec{S} &
    \text{ -- интегральный вид (теорема импульсов)}.
\end{align*}

\newpage
