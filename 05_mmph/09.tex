\question{Решение неоднородных ДУЧП методом разложения по собственным функциям.}

Рассмотрим неоднородную задачу:
\[
    \left. \begin{array}{rl}
        \text{ДУЧП:} & \ds \pder{u}{t} = \alpha^2\ppder{u}{x} + f(x, t),
        \vspace*{.4em} \\
        \text{ГУ:} & \left\{ \begin{array}{l}
            \alpha_1\left.\pder{u}{x}\right|_{0, t} + \beta_1 u(0, t) = 0, \\
            \alpha_2\left.\pder{u}{x}\right|_{l, t} + \beta_2 u(l, t) = 0,
        \end{array} \right. \\
        \text{НУ:} & u(x, 0) = \phi(x).
    \end{array} \right\}
\]

Идея метода разложения по собственным функциям состоит в следующем:
решение задачи ищется в виде
\( \ds u(x, t) = \sum\limits_{n=1}^\infty T_n(t)X_n(x) \), где
\( X_n(x) \) -- собственные функции задачи Штурма-Лиувилля:
\begin{align*}
    & X'' + \lambda^2 X = 0, \\
    & \left\{ \begin{array}{l}
        \alpha_1 X'(0) + \beta_1 X(0) = 0, \\
        \alpha_2 X'(l) + \beta_2 X(l) = 0.
    \end{array} \right.
\end{align*}

Подстановкой вида решения в ДУЧП определяется вид функции \( T_n(t) \) при
помощи получаемых ОДУ. Коэффициенты разложения \( u(x, 0) = \phi(x) \) по
системе \( \bigl\{ X_n \bigr\} \) выполняют роль начальных условий в этих ОДУ.
    
\newpage
