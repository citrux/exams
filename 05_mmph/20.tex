\question{Волновое уравнение и три типа граничных условий.}

Обычно различают ГУ трех типов:
\begin{enumerate}
    \item Заданное движение концов (ГУ I рода):
    \[
        \left\{ \begin{array}{l}
            u(0, t) = g_1(t), \\
            u(l, t) = g_2(t).
        \end{array} \right.
    \]
    
    \item Заданные силы на концах (ГУ II рода):
    \[
        \left\{ \begin{array}{l}
            \ds \pder{u}{x}(0, t) = g_1(t), \\
            \ds \pder{u}{x}(l, t) = g_2(t).
        \end{array} \right.
    \]
    
    Так как вертикальная проекция силы натяжения струны равна \( \ds T\cdot
    \pder{u}{x} \), то такие граничные условия описывают вертикальные
    составляющие сил, действующих на концы.
    
    \item Упругое закрепление в граничных точках (ГУ III рода):
    \[
        \left\{ \begin{array}{l}
            \ds \pder{u}{x}(0, t) - k_1u(0, t) = g_1(t), \\
            \ds \pder{u}{x}(l, t) +k_2u(l,t) = g_2(t).
        \end{array} \right.
    \]
    
    Это условие очень похоже на условие второго типа, если учесть, что \( ku \)
    имеет смысл возвращающей упругой силы при малых деформациях.
\end{enumerate}
\newpage
