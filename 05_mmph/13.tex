\chapter{Свойства преобразования Фурье. Решение задачи о распространении тепла
в бесконечном стержне с заданной начальной температурой.}

\section{Свойства преобразования Фурье}
\begin{enumerate}
    \item обратимость: \( \mathcal{F}^{-1}\bigl[\mathcal{F}[f]\bigr] = f \);
    \item линейность:
        \( \mathcal{F}[af + bg] = a\mathcal{F}[f] + b\mathcal{F}[g] \);
    \item преобразование производных:
    \( \ds
        \mathcal{F}_x\left[\pnder{n}{u}{x}\right] =
            (\i\xi)^n\mathcal{F}_x[u],\ 
        \mathcal{F}_x\left[\pnder{n}{u}{t}\right]=
            \pnder{n}{}{t}\mathcal{F}_x[u];
    \)
    \item свёртка:    
    \( \mathcal{F}[f * g] = \mathcal{F}[f]\cdot \mathcal{F}[g] \), где
    \[ 
        (f * g)(x) = \frac{1}{\sqrt{2\pi}}\int\limits_{-\infty}^{+\infty}
        f(x-\xi)g(\xi)\d\xi \text{ -- свёртка}.
    \]
\end{enumerate}

\section{Задача о распространении тепла в бесконечном стержне}
\begin{minipage}{.67\textwidth}
    Рассмотрим теперь \emph{задачу о распространении тепла в бесконечном
    стержне}, если задана начальная температура \( u(x, 0) = \phi(x) \).

    Сделаем преобразование Фурье по переменной \( x \): \( u \to \bar{u} \).
\end{minipage}
\hfill
\begin{minipage}{.3\textwidth}
    \[
        \left\{ \begin{array}{l}
            \ds \pder{u}{t} = \alpha^2 \ppder{u}{x}, \\
            u(x, 0) = \phi(x).
        \end{array} \right.
    \]
\end{minipage}

\begin{minipage}{.67\textwidth}
    Преобразованная задача:
    
    Её решение: \( \ds \bar{u} = \bar{\phi}(\xi)\cdot\e^{-\alpha^2\xi^2 t} \).
\end{minipage}
\hfill
\begin{minipage}{.3\textwidth}
    \[
        \left\{ \begin{array}{l}
            \ds \der{u}{t} = -\alpha^2\xi^2\bar{u}, \\
            \bar{u}(0) = \bar{\phi}(\xi).
        \end{array} \right.
    \]
\end{minipage}

Обратным преобразованием Фурье получаем ответ:
\[
    u(x, t) = \mathcal{F}^{-1}[\bar{\phi}(\xi)\cdot\e^{-\alpha^2\xi^2 t}] =
    \mathcal{F}^{-1}[\bar{\phi}(\xi)]*\mathcal{F}^{-1}[\e^{-\alpha^2\xi^2 t}]=
    \frac{1}{2\alpha\sqrt{\pi t}} \int\limits_{-\infty}^{+\infty} \phi(s)
    \e^{\frac{-(x-s)^2}{4\alpha^2 t}}\d s.
\]

Функция \( \ds G(x, t) = \frac{1}{2\alpha\sqrt{\pi t}}
\e^{\frac{-x^2}{4\alpha^2 t}} \) называется функцией Грина данной задачи. Тогда
ответ может быть представлен в виде:
\[
    u(x, t) = \int\limits_{-\infty}^{+\infty} \phi(s)\cdot G(x-s, t)\d s.
\]

\newpage
