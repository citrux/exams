\chapter{Системы ДУЧП. Решение простейшей линейной системы.}

Рассмотрим задачу Коши для системы, содержащей два уравнения и два начальных
условия:
\[
    \left\{ \begin{array}{rl}
        \ds \text{ДУЧП 1 } & \pder{u_1}{t} + 8\pder{u_2}{x} = 0, \\[.4em]
        \ds \text{ДУЧП 2 } & \pder{u_2}{t} + 2\pder{u_1}{x} = 0, \\
        \text{НУ 1 } & u_1(x, 0) = f(x), \\
        \text{НУ 2 } & u_2(x, 0) = g(x).
    \end{array} \right.
\]

Запишем эту же систему уравнений в матричной форме:
\[
    \begin{bmatrix} \ds \pder{u_1}{t} \\[.6em] \ds \pder{u_2}{t} \end{bmatrix} +
    \begin{bmatrix} 0 & 8 \\[-.5em] 2 & 0 \end{bmatrix}
    \begin{bmatrix} \ds \pder{u_1}{x} \\[.6em] \ds \pder{u_2}{x} \end{bmatrix} =
    \begin{bmatrix} 0 \\[-.5em] 0 \end{bmatrix}
\]
или \( u'_t + Au'_x = 0 \), где
\[
    A = \begin{bmatrix} 0 & 8 \\[-.5em] 2 & 0 \end{bmatrix}, \quad
    u'_t =
    \begin{bmatrix} \ds \pder{u_1}{t} \\[.6em] \ds \pder{u_2}{t} \end{bmatrix},
    \quad u'_x =
    \begin{bmatrix} \ds \pder{u_1}{x} \\[.6em] \ds \pder{u_2}{x} \end{bmatrix},
    \quad 0 = \begin{bmatrix} 0 \\[-.5em] 0 \end{bmatrix}.
\]

Введем новые неизвестные функции
\( \ds v = \begin{bmatrix} v_1 \\[-.5em] v_2 \end{bmatrix} \)
с помощью преобразования \( u = Pv \), где \( P \) -- матрица, по столбцам
которой стоят собственные векторы матрицы \( A \). Продифференцировав обе части
этого соотношения, получаем
\[
    \pder{u}{t} = P\pder{v}{t}, \quad \pder{u}{x} = P\pder{v}{x}.
\]

Подставляя эти соотношения в систему уравнений получаем: \( Pv'_t + APv'_x = 0 \).

Если умножить обе части последнего соотношения на обратную к \( P \) матрицу
\( P^{-1} \), то получим \( v'_t + P^{-1}APv'_x = 0 \), или
\[
    v'_t + \Lambda v'_x = 0,\quad \text{ где }
    \Lambda = \begin{bmatrix} 4 & 0 \\[-.5em] 0 & -4 \end{bmatrix}.
\]

Получилась система из двух несвязанных уравнений, которые можно решать
независимо:
\[
    \pder{v_1}{t} + 4\pder{v_1}{x} = 0,\quad \pder{v_2}{t} - 4\pder{v_2}{x} = 0.
\]
Как известно, решениями этих уравнений являются бегущие волны:
\[
    v_1(x, t) = \phi(x - 4t),\quad v_2(x, t) = \psi(x + 4t),
\]
где \( \phi \) и \( \psi \) -- произвольные дифференцируемые функции.

Теперь находим \( u \) из соотношения \( u = Pv \):
\[
    u = \begin{bmatrix} 2 & -2 \\[-.5em] 1 & 1 \end{bmatrix}
    \begin{bmatrix} v_1 \\[-.5em] v_2 \end{bmatrix} =
    \begin{bmatrix} 2 & -2 \\[-.5em] 1 & 1 \end{bmatrix}
    \begin{bmatrix} \phi(x - 4t) \\[-.5em] \psi(x + 4t) \end{bmatrix} =
    \begin{bmatrix} 2\phi(x-4t) - 2\psi(x+4t) \\[-.5em]
    \phi(x-4t) + \psi(x+4t) \end{bmatrix}.
\]

Таким образом, общее решение системы:
\[
    \left\{
    \begin{array}{l}
        u_1(x, t) = 2\phi(x-4t) - 2\psi(x+4t), \\
        u_2(x, t) = \phi(x-4t) + \psi(x+4t).
    \end{array}
    \right.
\]

Подставляя его в начальные условия, получаем:
\[
    \phi(x) = \frac{1}{4}\bigl(f(x) + 2g(x)\bigr), \quad
    \psi(x) = \frac{1}{4}\bigl(2g(x) - f(x)\bigr).
\]

Окончательно:
\[
    \left\{
    \begin{array}{l}
    u_1(x, t) = 2\phi(x-4t) - 2\psi(x+4t) = \frac{1}{2}\bigl(f(x-4t) + 2g(x-4t)
    \bigr) - \frac{1}{2}\bigl(2g(x+4t) - f(x+4t)\bigr), \\
    u_2(x, t) = \phi(x-4t) + \psi(x+4t) = \frac{1}{4}\bigl(f(x-4t) + 2g(x-4t)
    \bigr) + \frac{1}{4}\bigl(2g(x+4t) - f(x+4t)\bigr).
    \end{array}
    \right.
\]

\newpage
