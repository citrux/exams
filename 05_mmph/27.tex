\chapter{Принцип суперпозиции. Разложение смешанной задачи на две более
простые. Разделение переменных и интегральные преобразования как проявление
принципа суперпозиции.}

Принцип суперпозиции состоит в том, что результирующий отклик системы на
воздействие может быть представлен в виде суммы откликов на каждое элементарное
воздействие, на которые можно разбить исходное воздействие.

\section{Разложение смешанной задачи на две более простые}
Рассмотрим задачу:
\[
    \left. \begin{array}{l}
        \ds \pder{u}{t} = \ppder{u}{x} + \sin\pi x, \\[.4em]
        \left\{ \begin{array}{l}
            u(0, t) = 0, \\
            u(1, t) = 0, 
        \end{array} \right. \\
        u(x, 0) = \sin2\pi x.
    \end{array} \right\}
\]

Вместо непосредственного решения поставленной задачи рассмотрим две подзадачи:
\[
    \left. \begin{array}{l}
        \ds \pder{u}{t} = \ppder{u}{x} + \sin\pi x, \\[.4em]
        \left\{ \begin{array}{l}
            u(0, t) = 0, \\
            u(1, t) = 0, 
        \end{array} \right. \\
        u(x, 0) = 0.
    \end{array} \right\}
    \qquad \text{ и } \qquad
    \left. \begin{array}{l}
        \ds \pder{u}{t} = \ppder{u}{x}, \\[.4em]
        \left\{ \begin{array}{l}
            u(0, t) = 0, \\
            u(1, t) = 0, 
        \end{array} \right. \\
        u(x, 0) = \sin2\pi x.
    \end{array} \right\}
\]

Очевидно, что сумма решений этих двух подзадач будет являться решением исходной
задачи, как это имеет место в неоднородных ОДУ при представлении решений в виде
частного решения неоднородного уравнения и общего решения однородного. Таким
образом,
\[
    u(x, t) = \frac{1}{\pi^2}\Bigl(1 - \e^{-\pi^2 t}\Bigr)\sin\pi x +
    \e^{-4\pi^2 t}\sin2\pi x.
\]

Из вышеприведенной аналогии можно заключить, что это справедливо для любого вида
неоднородности и начального условия.

\section{Разделение переменных и интегральные преобразования как проявление
принципа суперпозиции}

Метод разделения переменных является проявлением принципа суперпозиции:
разложение в ряд по начальным функциям воздействия \( f \) и отклика \( u \)
позволяет определить соотношения между \( f_n \) и \( u_n \).

В случае интегральных преобразований (например, преобразования Фурье) решение
находится как суперпозиция откликов на точечные источники (интегрирование
функции Грина).

\newpage
