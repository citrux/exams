\chapter{Уравнение колебаний струны и его интуитивная интерпретация.
Замечания.}

Уравнение колебания струны имеет вид:
\[
    \ppder{u}{t} = \alpha^2\ppder{u}{x},
\]
где левая часть определяет вертикальное ускорение струны, правая -- кривизну.

Это уравнение может быть интерпретировано следующим образом: ускорение струны,
обусловленное её натяжением, тем больше, чем больше вогнутость струны в данной
точке.

Рассмотрим замечания:
\begin{enumerate}
    \item Уравнение \( \ds \ppder{u}{t} = \alpha^2\ppder{u}{x} \) описывает также
    продольные и крутильные колебания стержня. Оно вообще описывает любой
    волновой процесс в упругой среде в одномерном случае, если пренебречь
    внешним воздействием и потерями энергии.
    
    \item Если линейная плотность струны \( \rho(x) \) зависит от координаты, то
    уравнение несколько изменяет вид:
    \[
        \ppder{u}{t} = \pder{}{x}\left(\alpha^2(x)\pder{u}{x}\right).
    \]
    
    \item Так как производная по времени является производной второго порядка,
    то для решения задач требуются два начальных условия: \( u(x, 0) \) и
    \( \ds \pder{u}{t}(x, 0) \).
    
    \item Для рассмотрения квазистационарных процессов в электрических цепях
    пользуются телеграфными уравнениями:
    \begin{align*}
        \pder{i}{x} + c\pder{u}{t} + gu = 0, \\
        \pder{u}{x} + l\pder{i}{t} + ri = 0,
    \end{align*}
    где \( c \) и \( l \) -- погонные ёмкость и индуктивность, \( r \) и \( g \)
    -- погонные сопротивление и проводимость утечки.
    
    Из этих уравнений нетрудно получить два других:
    \begin{align*}
        \ppder{i}{x} = lc\ppder{i}{t} + (cr + gl)\pder{i}{t} + gri, \\
        \ppder{u}{x} = lc\ppder{u}{t} + (cr + gl)\pder{u}{t} + gru,
    \end{align*}
    которые также называются телеграфными. В случае, когда \( g \) и \( r \)
    пренебрежимо малы, эти уравнения переходят в волновые:
    \[
        \ppder{i}{t} = \alpha^2\ppder{i}{x}, \quad \ppder{u}{t} =
        \alpha^2\ppder{u}{x}, \text{ где } \alpha^2 = \frac{1}{lc}.
    \]
\end{enumerate}

\newpage
