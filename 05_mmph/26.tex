\question{Конечные синус- и косинус-преобразования Фурье. Решение краевой задачи
с неоднородным волновым уравнением.}

Конечные синус- и косинус-преобразования определяются следующим образом:
\[
    S_n = \frac{2}{l} \int\limits_0^l f(x)\cdot\sin\frac{\pi nx}{l}\d x
    \quad \text{и} \quad
    C_n = \frac{2}{l} \int\limits_0^l f(x)\cdot\cos\frac{\pi nx}{l}\d x,
\]
а обратные преобразования:
\[
    f(x) = \sum_{n=1}^\infty S_n\sin\frac{\pi nx}{l}
    \quad \text{и} \quad
    f(x) = \frac{C_0}{2} + \sum_{n=1}^\infty C_n\cos\frac{\pi nx}{l}.
\]

Очевидно, что эти преобразования -- ни что иное, как коэффициенты \( b_n \) и
\( a_n \) в разложении функции в ряд Фурье, а обратные преобразования -- это
ряды Фурье по синусам и косинусам.

Для решения ДУЧП полезно знать как преобразуются производные при описанных выше
преобразованиях:
\begin{align*}
    S_x\left[\pnder{m}{u}{t}\right] & = \nder{m}{}{t}S_x[u], \quad
    S_x\left[\ppder{u}{x}\right] = -\left(\frac{\pi n}{l}\right)^2 S_x[u] +
    \frac{2\pi n}{l^2}\Bigl(u(0, t) + (-1)^{n+1}u(l, t)\Bigr); \\
    C_x\left[\pnder{m}{u}{t}\right] & = \nder{m}{}{t}C_x[u], \quad
    C_x\left[\ppder{u}{x}\right] = -\left(\frac{\pi n}{l}\right)^2 C_x[u] -
    \frac{2}{l}\left(\pder{u}{x}(0, t) + (-1)^{n+1}\pder{u}{x}(l, t)\right).
\end{align*}

Решим теперь краевую задачу с неоднородным волновым уравнением:

\begin{minipage}{.3\textwidth}
    \[
        \left. \begin{array}{l}
            \ds \ppder{u}{t} = \ppder{u}{x} + \sin\pi x, \\[.4em]
            \left\{ \begin{array}{l}
                u(0, t) = 0, \\
                u(1, t) = 0,
            \end{array} \right. \\
            \left\{ \begin{array}{l}
                u(x, 0) = 1, \\
                \ds \pder{u}{t}(x, 0) = 0,.
            \end{array} \right.
        \end{array} \right\}
    \]
\end{minipage}
\hfill
\begin{minipage}{.65\textwidth}
    Воспользуемся синус-преобразованием по переменной \( x \) и получим:
    \[
        \dder{S_n}{t} = -(\pi n)^2 S_n + (2\pi n)(0 + (-1)^{n+1}\cdot 0) + \delta_{1n},
    \]
    где \( \delta_{ij} \) -- символ Кронекера.

    Получаем семейство задач Коши.
\end{minipage}

\begin{align*}
    \text{При } n = 1 \text{ имеем: } & S_1'' + \pi^2S_1 = 1, \quad S_1'(0) = 0,
    \quad S_1(0) = \frac{4}{\pi}; \\
    & S_1(t) = \left(\frac{4}{\pi} - \frac{1}{\pi^2}\right)\cdot\cos(\pi t) +
    \frac{1}{\pi^2}. \\
    \text{При } n \ne 1 \text{ имеем: } & S_n'' + (\pi n)^2 S_n = 1,
    \quad S_n'(0) = 0, \quad S_n(0) =
    \left\{ \begin{array}{rl}
        \ds \frac{4}{\pi n}, & \text{ при нечётном } n, \\
        0, & \text{ при чётном } n;
    \end{array} \right. \\
    & S_n(t) = \left\{ \begin{array}{rl}
        \ds \frac{4}{\pi n}\cos(\pi nt), & \text{ при нечётном } n, \\
        0, & \text{ при чётном } n.
    \end{array} \right.
\end{align*}

В итоге:
\[
    u(x, t) = \left(\left(\frac{4}{\pi} - \frac{1}{\pi^2}\right)\cos\pi t
    + \frac{1}{\pi^2}\right)\sin\pi x + \frac{4}{\pi}\sum_{n=1}^\infty
    \frac{1}{2n+1}\cos(2n + 1)\pi t\cdot\sin(2n + 1)\pi x.
\]

\newpage
