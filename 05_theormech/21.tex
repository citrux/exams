\chapter{Полная система уравнений, описывающая движение частицы сплошной
среды. Уравнение состояния. Простейшие модели сплошных сред: линейное
упругое тело, идеальная жидкость.}

Система уравнений движения частицы сплошной среды состоит из
\begin{itemize}
\item уравнения неразрывности
\[
    \der{\rho }{t}+ \rho {\nabla_iv^i } = 0,
\]
\item уравнения поступательного движения частицы сплошной среды:
\[
    \rho  \der{v^i}{t}  = \rho f^i + \pder{ \sigma^{ij}}{x^j},
\]
\item закона парности касательных напряжений
\[
    \sigma^{ij} = \sigma^{ji}.
\]
\end{itemize}

Систему в таком виде решать ещё нельзя: здесь 7 уравнений и 13 неизвестных.
Чтобы решить задачу дополним её ещё 3 неизвестными -- компонентами вектора
перемещений: \( s^{i} \), выразим через вектор перемещений тензор деформаций
\( e^j_i  = \nabla_i s^i \) и тензор скоростей деформаций
\( {de^j_i}/{dt} = \dot{e}^j_i \). Если теперь ввести представления о модели
сплошной среды -- \emph{уравнения состояния} -- выразить зависимость тензора
напряжений от тензора деформаций и тензора скоростей деформаций
\[
    \sigma^{kl} = f^{kli}_j(e^j_i, \dot{e}^j_i),
\]
то система станет полной (16 уравнений, 16 неизвестных).

В качестве примера моделей приведём две:
\begin{itemize}
\item \emph{идеальная жидкость} -- среда для которой вектор напряжений
\( \sigma^{kl}n_k \) в любой точке параллелен \( n_k \), где \( n_k \) -- вектор
нормали к произвольной площадке. Тензор напряжений такой среды 
\( \sigma^{kl} = - p g^{kl} \). Уравнения движения такой среды носят название
\emph{уравнений Эйлера} и имеют вид:
\[
    \left\{\begin{array}{l}
    \der{\rho}{t} + \rho {\nabla_iv^i } = 0, \\
    \rho\der{v^i}{t}  = \rho f^i - g^{ij}\pder{p}{x^j};
    \end{array}\right.    
\]
последнее уравнение в векторном виде
\[
    \rho\der{\vec{v}}{t} = \rho\vec{f} - \gradient p.
\]

\item \emph{линейное упругое тело} -- среда, в которой тензор напряжений есть
линейная функция тензора деформаций (подчиняется закону Гука):
\[
    \sigma^{kl} = A^{klij}e_{ij}.
\]
\end{itemize}

\newpage
