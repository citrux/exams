\question{Механическая система (изменяемая и неизменяемая). Масса системы. Центр
масс и его координаты. Статические моменты массы системы относительно
полюса и плоскости. Статические моменты массы относительно центра масс и
плоскостей, проходящих через центр масс.}

Механической системой называется совокупность материальных точек. Если при
движении взаимное расположение всех материальных точек системы не изменяется, то
мы имеем дело с неизменяемой механической системой. Примером такой системы
является абсолютно твердое тело. В противном случае система называется
изменяемой. Таковыми, например, являются жидкости и газы.

Основными характеристиками механической системы являются её масса
\( \ds M = \sum_i m_i \) и центр масс \( \ds \vec{r}_c =
\frac{\sum\limits_i m_i\vec{r}_i}{M} \).

\subquestion{Статические моменты массы системы}
Статический момент массы системы -- векторная физическая величина,
характеризующая распределение массы тела:
\[
    \vec{K} = \int \vec{r}\,dm.
\]
Определённый выше статический момент массы есть момент относительно полюса.
Также момент массы можно определить относительно плоскости:
\[
    \vec{K} = \vec{n}\int \rho\,dm,
\]
где \( \vec{n} \) -- вектор номали к плоскости, а \( \rho \) -- расстояние
от плоскости до элемента \( dm \).

Положение центра масс системы определяется как
\[
    \vec{r}_c = \frac{\int \vec{r}\,dm}{\int dm},
\]
откуда
\[]
    \int\vec{r}_c\,dm = \int \vec{r}\,dm,\quad
    \int(\vec{r} - \vec{r}_c)\,dm = 0.
\]
Таким образом, статический момент массы относительно центра масс равен нулю.
Следовательно и моменты относительно плоскостей, проходящих через центр масс
равны нулю.

\newpage % ---------------------------------------------------------------------
