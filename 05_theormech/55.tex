\question{Теорема об изменении кинетической энергии материальной точки и системы
в дифференциальной и конечной формах. Случай абсолютно твердого тела.}

\subquestion{Теорема об изменении кинетической энергии материальной точки и системы
в дифференциальной и конечной формах}

Рассмотрим материальную точку с массой \( m \), перемещающуюся из положения
\( M_0 \), где она имеет скорость \( v_0 \), в положение \( M_1 \), где её
скорость \( v_1 \).

Для получения искомой зависимости обратимся к выражающему основной закон
динамики уравнению
\[
	m\vec{a} = \sum_k \vec{F_k}.
\]
Проектируя обе его части на касательную к траектории точки \( M \),
направленную в сторону движения, получим:
\[
    ma_\tau = \sum_k F_{k\tau}.
\]
Входящее сюда касательное ускорение точки представим в виде
\[
    a_\tau = \der{v}{t} = \der{v}{s}\der{s}{t} = v\der{v}{s}.
\]
В результате найдём, что
\[
    mv\der{v}{s} = \sum_k F_{k\tau}.
\]
Умножим обе части этого равенства на \( ds \) и внесём \( m \) под знак 
дифференциала. Тогда, замечая, что \( F_{k\tau}ds = dA_k \), где 
\( dA_k \) -- элементарная работа силы \( \vec{F_k} \), получим выражение 
\emph{теоремы об изменении кинетической энергии точки в дифференциальной 
форме}:
\[
    d\left( \frac{mv^2}{2} \right) = \sum_k dA_k
\]

Проинтегрируем теперь обе части этого равенства в пределах, соответствующих 
значениям переменных в точка \( M_0 \) и \( M_1 \), найдём окончательно:
\[
    \frac{mv^2_1}{2} - \frac{mv^2_0}{2} = \sum_k A_k.
\]
Последнее уравнение выражает теорему об изменении кинетической 
энергии точки в конечном виде: \emph{изменение кинетической энергии точки при 
некотором её перемещении равно алгебраической сумме работ всех действующих 
на точку сил на том же перемещении}.

Теперь перейдём к рассмотрению механических систем. Величина
\[
	T = \sum_j \frac{m_jv_j^2}{2}
\]
называется кинетической энергией механической системы. Теорема об изменении
кинетической энергии для точки будет справедлива для любой точки системы:
\[
    d\left(\frac{m_j v_j^2}{2}\right) = dA^e_j + dA^i_j.
\]
Составляя такие уравнения для всех точек системы и складывая их, получим:
\[
    dT = \sum_j d\left(\frac{m_jv_j^2}{2}\right) =
    \sum_j dA^e_j + \sum_j dA^i_j.
\]
Проинтегрируем:
\[ 
	\Delta T = \sum_j A^e_j + \sum_j A^i_j.
\]
Теорема об изменении кинетической энергии в конечном виде: 
\emph{изменение кинетической энергии системы при некотором её 
конечном перемещении равно сумме работ на этом перемещении всех 
приложенных к системе внешних и внутренних сил}.

\subquestion{Случай абсолютно твердого тела}
Работа внутренних сил в абсолютно твёрдом теле равна нулю:
\[
	\sum A^i_k = 0.
\]
Следовательно, формулу можно переписать в виде:
\[
    \Delta T = \sum A^e_k,
\] 
то есть изменение кинетической энергии абсолютно твёрдого тела равно работе внешних сил.
\newpage
