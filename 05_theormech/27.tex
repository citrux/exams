\chapter{Исследование движения материальной точки в центрально-симметричном
поле сил. Первые интегралы уравнений движения. Понятие об эффективной
потенциальной энергии. Интегрирование уравнений движения.}

Рассмотрим движение материальной точки в центральном поле. Такое поле является
потенциальным:
\[
    U = U(r),\quad \vec{F} = -\gradient U(r) = -\der{U}{r}\frac{\vec{r}}{r}.
\]
Нетрудно видеть, что момент этой силы относительно центра равен нулю:
\[
    \vec{M} = \vec{r}\times\vec{F} = -\der{U}{r}\frac{\vec{r}\times\vec{r}}{r}
    = 0.
\]
Следовательно, момент импульса точки \( \vec{K} \) остаётся постоянным, а её
движение происходит в одной плоскости. Так как поле потенциально, то и
механическая энергия точки остаётся постоянной.

Определим энергию точки при таком движении. Введём полярные координаты
\( (r, \phi) \). В них:
\[
    E = \frac{m}{2}(\dot{r}^2 + r^2\dot{\phi}^2) + U(r).
\]
Но
\[
    \frac{mr^2\dot{\phi}^2}{2} = \frac{K^2}{2mr^2},
\]
поэтому
\[
    E = \frac{m\dot{r}^2}{2} + \frac{K^2}{2mr^2} + U(r).
\]
Величина
\[
    \frac{K^2}{2mr^2} + U(r)
\]
называется эффективной потенциальной энергией \( U_\textit{э} \). Таким образом,
энергия может быть представлена в виде
\[
    E = \frac{m\dot{r}^2}{2} + U_\textit{э}(r).
\]
Отсюда нетрудно получить
\begin{gather*}
    \dot{r} = \der{r}{t} = \sqrt{\frac{2}{m}(E-U_\textit{э}(r))},\\
    t = \int\frac{dr}{\sqrt{\frac{2}{m}(E-U_\textit{э}(r))}} + \const,\\
    d\phi = \frac{K}{mr^2}\,dt\Rightarrow \phi =
    \int\frac{dr}{r^2\sqrt{2m(E-U_\textit{э}(r))}} + \const.
\end{gather*}
\newpage
