\chapter{Понятие об абсолютной и ковариантной производных.}

Рассмотрим произвольное векторное поле \( \vec{a} \). Как и ранее определим его
в ковариантном базисе и попытаемся найти от него частные производные по
координатам:
\begin{gather*}
    \pder{\vec{a}}{q^i} = 
    \pder{(a^j \vec{e}_{j})}{q^i} = 
    \pder{a^j}{q^i}\vec{e}_{j} + \pder{ \vec{e}_{j}}{q^i} a^j =
    \pder{a^j}{q^i}\vec{e}_{j} + \Gamma^k_{ij} \vec{e}_{k} a^j =\\=
    \pder{a^k}{q^i}\vec{e}_{k} + \Gamma^k_{ij} \vec{e}_{k} a^j =
    \left(\pder{a^k}{q^i} + a^j \Gamma^k_{ij}\right)\vec{e}_{k}.
\end{gather*}

Выражение в скобках даёт контрвариантные компоненты частной производной от
векторного поля по координатам, носит название \emph{ковариантной производной
контрвариантных компонент вектора } \( \vec{a} \) и обозначается:
\[
    \nabla_i a^k = a^k_{,i} = \pder{a^k}{q^i} + a^j \Gamma^k_{ij}.
\]
    
Дифференциал от вектора \( \vec{a} \) носит название абсолютного дифференциала
и обозначается \( \delta  \vec{a} \) или \( D  \vec{a} \), если \( \vec{a} \)
функция координат \( q^i \) и параметра \( t \), то с учётом, что базисные
векторы \( \vec{e}_{k} \) явно от \( t \) не зависят:
\begin{gather*}
    \delta  \vec{a} \equiv \delta a^k \vec{e}_{k}= 
    \pder{\vec{a}}{t} dt + \pder{ \vec{a}}{q^i} dq^i = 
    \pder{a^k}{t} \vec{e}_{k} dt  + \nabla_i a^k \vec{e}_{k} dq^i =
    \pder{a^k}{t}\vec{e}_{k} dt  + \left(\pder{a^k}{q^i}dq^i +
    a^j \Gamma^k_{ij} dq^i\right) \vec{e}_{k} = \\ =
    \left(\pder{a^k}{t} dt+\pder{a^k}{q^i}dq^i +
    a^j \Gamma^k_{ij} dq^i\right) \vec{e}_{k} = 
    \left(d{a^k} + a^j \Gamma^k_{ij} dq^i\right) \vec{e}_{k}.
\end{gather*}
    
Если \( q^i \) - функции параметра \( t \), то компоненты отношения
дифференциалов \( \delta\vec{a}/dt \) \( \delta a^k/dt \) носят название
\emph{абсолютной производной контрвариантных компонент вектора}.

\newpage
