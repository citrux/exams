\chapter{Уравнения Лагранжа 1-го рода. Понятие об избыточных координатах.
Неопределенные множители Лагранжа. Полная система уравнений, описывающая
движение механической системы с избыточными координатами. Примеры применения.}

В основу вывода уравнений Лагранжа 1-го рода положим общее уравнение динамики:
\[
    \sum\limits_{j=1}^N \Bigl(-m_j\ddot{\vec{r}}_j + \vec{F}^a_j + \vec{R}_j
    \Bigr)\cdot\delta\vec{r}_j = 0,
\]
где \( m_j\ddot{\vec{r}}_j \) -- силы инерции, \( \vec{F}^a_j \) -- активные
силы, \( \vec{R}_j \) -- связи.

% преобразования

В результате преобразований в обобщенных координатах \( q_1 \), \( q_2 \),
\ldots, \( q_k \) (\( k \) не равно числу степеней свободы \( s \)), получим:
\[
    \sum\limits_{i=1}^k \left[\der{}{t}\left(\pder{T}{\dot{q}_i}\right) -
    \pder{T}{q_i} - Q_i - D_i\right]\delta q_i = 0,
\]
где \( Q_i \) -- обобщенные активные силы, \( D_i \) -- обобщенные реакции
связей. Так как выражение в скобках и \( \delta q_i \) независимы, то:
\[
    \der{}{t}\left(\pder{T}{\dot{q}_i}\right) - \pder{T}{q_i} - Q_i - D_i = 0.
\]

% далее пошел набор формул

\[
    f_j(q_1, q_2, \ldots, q_k, t) = 0, \quad \text{ где } j = 1, 2, \ldots, k-s.
\]

Условие идеальности связей: \( \ds \sum_{l=1}^k D_l\delta q_l = 0 \).
Подставляем \( D_l \), получаем:
\[
    \sum\limits_{l=1}^k \pder{f_j}{q_l}\delta q_l = 0.
\]

Получаем систему уравнений:
\[
    \left\{ \begin{array}{l}
       \ds\pder{f_1}{q_1}\delta q_1+\ldots+\pder{f_1}{q_k}\delta q_k =0,\\[.6em]
       \ds \pder{f_2}{q_1}\delta q_1 + \ldots + \pder{f_2}{q_k}\delta q_k = 0,\\
       \ldots\ldots \\
       \ds \pder{f_{k-s}}{q_1}\delta q_1 + \ldots + \pder{f_{k-s}}{q_k}\delta
       q_k = 0, \\
       D_1\delta q_1 + \ldots + D_k \delta q_k = 0.
    \end{array} \right.
\]

Домножим первые \( k - s \) уравнений на \( \lambda_1(t) \), \( \lambda_2(t) \),
\ldots, \( \lambda_{k-s}(t) \), называемые множителями Лагранжа, и вычтем их из
последнего:
\[
    \left[D_1 - \sum_{n=1}^{k-s}\lambda_n\pder{f_n}{q_1}\right]\delta q_1 +
    \left[D_2 - \sum_{n=1}^{k-s}\lambda_n\pder{f_n}{q_2}\right]\delta q_2 +
    \ldots +
    \left[D_k - \sum_{n=1}^{k-s}\lambda_n\pder{f_n}{q_k}\right]\delta q_k = 0.
\]

Так как \( \delta q_i \) не зависят от выражений в скобках, то видно, что
выражения в скобках должны быть равны нулю, откуда
\[
    D_m = \sum_{n=1}^{k-s} \lambda_n \pder{f_n}{q_m}.
\]

В итоге, получаем систему уравнений:
\[
    \left\{ \begin{array}{l}
        \ds \der{}{t}\left(\pder{T}{\dot{q}_i}\right) - \pder{T}{q_i} =
        Q_i + \sum_{n=1}^{k-s} \lambda_n\pder{f_n}{q_i}; \\
        f_j(q_1, q_2, \ldots, q_k, t) = 0, \quad j = 1, 2, \ldots, k - s.
    \end{array} \right.
\]

\newpage
