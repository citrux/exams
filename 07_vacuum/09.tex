\question{Уравнения движения электрона в неоднородном аксиально-симметричном
  электрическом поле. Основное уравнение параксиальной электроники}

Уравнение движения
\( \displaystyle
  \der{\vec{p}}{t} = \vec{F}
\)
в цилиндрической системе координат примет вид:
\[
  \left\{
    \begin{array}{l}
      m(\ddot{r} - r\dot{\phi}^2) = F_r, \\
      m(r\ddot{\phi} + 2\dot{r}\dot{\phi}) = F_\phi, \\
      m\ddot{z} = F_z.
    \end{array}
  \right.
\]

Для электрона в неоднородном аксиально-симметричном электрическом поле
(\( \dot{\phi} = \ddot{\phi} = 0 \)) эта система упрощается:
\[
  \left\{
    \begin{array}{l}
      \displaystyle m\ddot{r} = F_r = -e\pder{U}{r}, \\[.6em]
      \displaystyle m\ddot{z} = F_z = -e\pder{U}{z}.
    \end{array}
  \right.
\]

Ограничимся случаем, когда частица движется в параксиальной области. Тогда с
учетом \eqref{eq08.2.15} система примет вид:
\begin{equation}
  \left\{
    \begin{array}{l}
      m\ddot{r} = e\dfrac{U_0''}{2}r, \\
      m\ddot{z} = -eU_0' + q\dfrac{r^2}{4}U_0'''.
    \end{array}
  \right.
  \label{eq09system}
\end{equation}

Скорость частицы вдоль оси \( 0z \) найдем из закона сохранения энергии:
\[
  mv^2 / 2 = -eU; \qquad
    v = \der{z}{t} = \sqrt{-2\frac{e}{m}U_0}.
\]

Используем это выражение для исключения из \eqref{eq09system} явную зависимость
от времени:
\[
  \der{}{t} = \der{}{z}\der{z}{t} = \sqrt{-2\frac{e}{m}U_0}\cdot\der{}{z}; \quad
    \dder{}{t} = \der{z}{t}\der{}{z} \left( \der{z}{t}\der{}{z} \right) =
    -2\frac{e}{m}\sqrt{U_0}\der{}{z} \left( \sqrt{U_0}\der{}{z} \right).
\]

Подставляя этот оператор в первое уравнение системы \eqref{eq09system},
получаем:
\begin{equation}
  \der{}{z} \left( \sqrt{U_0(z)}\ \der{r}{z} \right) = -\frac{1}{4}
    \frac{U_0''(z)}{\sqrt{U_0(z)}} r,
  \label{eq09equation}
\end{equation}
или, раскрывая оператор,
\begin{equation}
  \dder{r}{z} + \frac{1}{2} \frac{U_0'(z)}{U_0(z)} \der{r}{z} + \frac{1}{4}
    \frac{U_0''(z)}{U_0(z)} r = 0.
  \label{eq09paraxial}
\end{equation}

Уравнение \eqref{eq09paraxial} называется основным уравнением параксиальной
электроники. Его анализ позволяет сделать ряд выводов о траекториях электронов:
\begin{enumerate}
  \item поскольку в уравнение не входят ни заряд, ни масса частицы, то
    траектории любых заряженных частиц, движущихся с нерелятивистской скоростью
    в приосевой области совпадают при одинаковых \( U_0(z) \);
  \item траектории частиц обратимы, то есть при изменении знака \( U_0(z) \) вид
    уравнения не изменяется и частицы будут лететь по тем же траекториям, только
    в обратном направлении;
  \item это уравнение однородно относительно \( r \). Таким образом, при
    введении новой координатной зависимости \( R = r\cdot\const \), вид
    уравнения не изменится~-- форма траекторий останется прежней. Это позволяет
    изучать траектории частиц на масштабных моделях;
  \item это уравнение однородно и относительно потенциалов. Если изменить
    \( U_0 \) на постоянную величину \( \Phi_0 = U_0\cdot\const \), то вид
    уравнения не изменится~-- форма траекторий останется прежней. Это позволяет
    изучать траектории частиц при небольших величинах потенциалов;
  \item траектории параксиальной частицы определяется только распределением
    потенциала на оси симметрии.
\end{enumerate}
