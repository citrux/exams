\question{Эмиссия электронов при наличии внешнего тормозящего поля. Определение
температуры катода по величине эмиссионного тока}

При наличии внешнего тормозящего поля увеличивается глубина потенциальной ямы, в
которой находится электрон в металле. Можно сказать, что эффективное значение
работы выхода в этом случае увеличивается на величину \( eU \). Тогда, учитывая
эту поправку в уравнении Ричардсона-Дэшмана, получим
\[
    j = CT^2\exp\left(-\frac{A+eU}{kT}\right) =
    j_0\exp\left(-\frac{eU}{kT}\right).
\]

Логарифмируя это выражение, получим
\[
    \frac{eU}{kT} = \ln\frac{j_0}{j},
\]
откуда
\[
    T = \frac{eU}{k\ln\frac{j_0}{j}}.
\]
Таким образом, зная значения тока при двух различных значениях задерживающего
напряжения, можно определить температуру катода.

