\question{Оператор момента импульса}

В классической механике моментом импульса материальной точки называют векторное
произведение радиус-вектора этой точки на её импульс:
\[
    \vec{L} = \vec{r}\times\vec{p}.
\]

Аналогично можно ввести оператор момента импульса:
\[
    \hat{\vec{L}} = \hat{\vec{r}}\times\hat{\vec{p}}.
\]

Распишем этот оператор покоординатно:
\[
    \hat{\vec{L}} =
    \begin{vmatrix}
        \vec{e}_x & \vec{e}_y & \vec{e}_z \\
        x         & y         & z         \\
        \hat{p}_x & \hat{p}_y & \hat{p}_z \\
    \end{vmatrix} =
    (y\hat{p}_z - z\hat{p}_y)\vec{e}_x +
    (z\hat{p}_x - x\hat{p}_z)\vec{e}_y +
    (x\hat{p}_y - y\hat{p}_x)\vec{e}_z.
\]
Учитывая, что
\[
    \hat{p}_\alpha = -i\hbar\pder{}{\alpha}, \quad (\alpha \in \{ x,y,z \})
\]
для проекций момента импульса на декартовы оси получим:
\[
    \left\{
        \begin{array}{l}
            \hat{L}_x = -i\hbar\left( y\pder{}{z} - z\pder{}{y} \right),\\
            \hat{L}_y = -i\hbar\left( z\pder{}{x} - x\pder{}{z} \right),\\
            \hat{L}_z = -i\hbar\left( x\pder{}{y} - y\pder{}{x} \right).\\
        \end{array}
    \right.
\]
Запишем теперь коммутационные соотношения для проекций момента:
\begin{gather*}
    \left[\hat{L}_x, \hat{L}_y\right] = \hat{L}_x\hat{L}_y - \hat{L}_y\hat{L}_x=
    (y\hat{p}_z - z\hat{p}_y)(z\hat{p}_x - x\hat{p}_z) -
    (z\hat{p}_x - x\hat{p}_z)(y\hat{p}_z - z\hat{p}_y) =\\
    =(y\hat{p}_z z\hat{p}_x - xy\hat{p}_z^2 - z^2\hat{p}_y\hat{p}_x +
    z\hat{p}_y x\hat{p}_z) -
    (z\hat{p}_x y\hat{p}_z - xy\hat{p}_z^2 - z^2\hat{p}_x\hat{p}_y +
    x\hat{p}_z z\hat{p}_y) =\\
    =x\hat{p}_y [z, \hat{p}_z] - y\hat{p}_x[z, \hat{p}_z]=
    [z, \hat{p}_z]\hat{L}_z = i\hbar\hat{L}_z.
\end{gather*}
Циклически переставляя координаты получим и 2 других коммутационных соотношения:
\begin{align*}
    & \left[\hat{L}_y, \hat{L}_z\right] = i\hbar\hat{L}_x,\\
    & \left[\hat{L}_z, \hat{L}_x\right] = i\hbar\hat{L}_y.
\end{align*}
Определим также следующий коммутатор:
\begin{gather*}
    \left[ \hat{L}^2, \hat{L}_x \right] = \left[ \hat{L}_x^2, \hat{L}_x \right]
    + \left[ \hat{L}_y^2, \hat{L}_x \right] +
    \left[ \hat{L}_z^2, \hat{L}_x \right] =
    \hat{L}_y\hat{L}_y\hat{L}_x - \hat{L}_x\hat{L}_y\hat{L}_y +
    \hat{L}_z\hat{L}_z\hat{L}_x - \hat{L}_x\hat{L}_z\hat{L}_z
\end{gather*}
Учтём теперь
\[
    \hat{L}_x\hat{L}_y = \hat{L}_y\hat{L}_x + i\hbar\hat{L}_z, \quad
    \hat{L}_x\hat{L}_z = \hat{L}_z\hat{L}_x - i\hbar\hat{L}_y
\]
и получим
\begin{gather*}
    \left[ \hat{L}^2, \hat{L}_x \right] =
    \hat{L}_y\hat{L}_x\hat{L}_y - i\hbar\hat{L}_y\hat{L}_z -
    \hat{L}_y\hat{L}_x\hat{L}_y - i\hbar\hat{L}_z\hat{L}_y +
    \hat{L}_z\hat{L}_x\hat{L}_z + i\hbar\hat{L}_z\hat{L}_y -
    \hat{L}_x\hat{L}_z\hat{L}_z + i\hbar\hat{L}_y\hat{L}_z = 0.
\end{gather*}
Совершенно аналогично можно показать, что для двух других проекций этот
коммутатор также будет равен нулю. Отсюда можно сделать вывод, что одновременно
могут быть определены лишь одна из проекций момента и его величина. Поэтому
имеет смысл рассмотреть этот оператор в полярных координатах. В них операторы
проекций момента принимают вид
\begin{align*}
    & \hat{L}_x = i\hbar\left(\sin\phi\pder{}{\theta} +
        \ctg\theta\cos\phi\pder{}{\phi}\right), \\
    & \hat{L}_y = -i\hbar\left(\cos\phi\pder{}{\theta} -
        \ctg\theta\sin\phi\pder{}{\phi}\right), \\
    & \hat{L}_z = -i\hbar\pder{}{\phi}.
\end{align*}
Квадрату момента импульса в сферических координатах соответствует оператор
\[
    \hat{L}^2 = -\hbar^2\Delta_{\theta,\phi},
\]
где
\[
    \Delta_{\theta,\phi} = \frac{1}{\sin\theta}\pder{}{\theta}
        \left(\sin\theta\pder{}{\theta}\right) +
        \frac{1}{\sin^2\theta}\ppder{}{\phi}
\]
сферическая часть оператора Лапласа.

Определим теперь собственные значения операторов \( \hat{L}_z \) и
\( \hat{L}^2 \). Начнём с проекции:
\[
    \hat{L}_z\psi = L_z\psi,
\]
\[
    -i\hbar\pder{\psi}{\phi} = L_z\psi,
\]
\[
    \pder{\psi}{\phi} - i\frac{L_z}{\hbar}\psi = 0.
\]
\[
    \psi = Ae^{i\frac{L_z}{\hbar}\phi}.
\]
Полученная функция должна иметь период \( 2\pi \), поэтому
\[
    \psi = Ae^{im\phi},
\]
\[
    L_z = m\hbar,\quad m \in \mathbb{Z}.
\]
Теперь рассмотрим квадрат момента:
\[
    \hat{L}^2\psi = L^2\psi,
\]
\[
    -\hbar^2\Delta_{\theta,\phi}\psi = L^2\psi,
\]
\[
    \Delta_{\theta,\phi}\psi + (\frac{L}{\hbar})^2\psi = 0.
\]
Обозначив \( (L/\hbar)^2 = \lambda \) получим уравнение сферических функций:
\[
    \frac{1}{\sin\theta}\pder{}{\theta}
        \left(\sin\theta\pder{\psi}{\theta}\right) +
        \frac{1}{\sin^2\theta}\ppder{\psi}{\phi} + \lambda\psi = 0,
\]
решениями которого являются сферические функции \( Y_{lm} \), а собственные
значения \( l(l+1),\ l\in\mathbb Z \). Собственные значения модуля момента
импульса равны \( L = \hbar\sqrt{l(l+1)} \).
