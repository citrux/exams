\question{Стационарные состояния}
Состояния, в которых энергия системы имеет определённые значения, называются
стационарными состояниями системы. Они описываются волновыми функциями
\( \Psi_n \), являющимися собственными функциями оператора Гамильтона
\( \hat{H} \), то есть удовлетворяющими уравнению
\[
    \hat{H}\Psi_n = E_n\Psi_n,
\]
где \( E_n \) -- собственные значения энергии. Волновое уравнение для функции
\( \Psi_n \) имеет вид
\[
    i\hbar\pder{\Psi_n}{t} = E_n\Psi_n,
\]
решение которого
\[
    \Psi_n = e^{-\frac{i}{\hbar}E_nt}\psi_n(q),
\]
где \( \psi_n \) -- функция только от координат.

