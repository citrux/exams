\question{Дифференцирование по времени в квантовой механике}

В квантовой механике невозможно говорить о производной величины по времени в
классическом смысле, так как величина, точно определённая в некоторый момент
времени, совершенно не определена в последующие. Поэтому в качестве производной
по времени принимается величина, среднее значение которой равно производной
по времени срежднего значения:
\[
    \average{\dot{f}} = \dot{\average{f}}.
\]
Исходя из этого определения нетрудно получить выражение для оператора
\( \hat{\dot{f}} \), соответствующего величине \( \dot{f} \):
\[
    \average{\dot{f}} = \dot{\average{f}} = \der{}{t}\int\psi^*\hat{f}\psi dq =
    \int\psi^*\pder{\hat{f}}{t}\psi dq + \int\pder{\psi^*}{t}\hat{f}\psi dq +
    \int\psi^*\hat{f}\pder{\psi}{t} dq.
\]
Выражая производные волновых функций из уравнения Шрёдингера и подставляя,
получаем
\[
    \average{\dot{f}} =  \int\psi^*\pder{\hat{f}}{t}\psi dq
    +\frac{i}{\hbar}  \int(\hat{H}^*\psi^*)\hat{f}\psi dq -
    \frac{i}{\hbar} \int\psi^*\hat{f}(\hat{H}\psi) dq.
\]
Поскольку оператор \( \hat{H} \) -- эрмитов, то
\[
    \int(\hat{H}^*\psi^*)\hat{f}\psi dq = \int \psi^*\hat{H}\hat{f}\psi dq. 
\]
Таким образом, имеем:
\[
    \average{\dot{f}} =  \int\psi^*\left(\pder{\hat{f}}{t} +
    \frac{i}{h}\hat{H}\hat{f} - \frac{i}{h}\hat{f}\hat{H} \right)\psi dq.
\]
Выражение, стоящее в скобках представляет собой искомый оператор
\[
    \hat{\dot{f}} = \pder{\hat{f}}{t} +
    \frac{i}{h}\left(\hat{H}\hat{f} - \hat{f}\hat{H}\right).
\]

