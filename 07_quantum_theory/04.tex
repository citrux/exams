\question{Собственные значения момента}
\label{q:04}
Определим собственные значения операторов \( \hat{L}_z \) и
\( \hat{L}^2 \). Начнём с проекции:
\[
    \hat{L}_z\psi = L_z\psi,
\]
\[
    -i\hbar\pder{\psi}{\phi} = L_z\psi,
\]
\[
    \pder{\psi}{\phi} - i\frac{L_z}{\hbar}\psi = 0.
\]
\[
    \psi = Ae^{i\frac{L_z}{\hbar}\phi}.
\]
Полученная функция должна иметь период \( 2\pi \), поэтому
\[
    \psi = Ae^{im\phi},
\]
\[
    L_z = m\hbar,\quad m \in \mathbb{Z}.
\]
Теперь рассмотрим квадрат момента:
\[
    \hat{L}^2\psi = L^2\psi,
\]
\[
    -\hbar^2\Delta_{\theta,\phi}\psi = L^2\psi,
\]
\[
    \Delta_{\theta,\phi}\psi + (\frac{L}{\hbar})^2\psi = 0.
\]
Обозначив \( (L/\hbar)^2 = \lambda \) получим уравнение сферических функций:
\[
    \frac{1}{\sin\theta}\pder{}{\theta}
        \left(\sin\theta\pder{\psi}{\theta}\right) +
        \frac{1}{\sin^2\theta}\ppder{\psi}{\phi} + \lambda\psi = 0,
\]
решениями которого являются сферические функции \( Y_{lm} \), а собственные
значения \( l(l+1),\ l\in\mathbb Z \). Собственные значения модуля момента
импульса равны \( L = \hbar\sqrt{l(l+1)} \).

