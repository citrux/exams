\question{Атом водорода}
Запишем уравнение Шредингера для электрона в кулоновском потенциале протона:
\[
	-\frac{\hbar^2}{2m}\Delta\psi - \frac{e^2}{r}\psi = E\psi,
\]
где \( e^2 = q_e^2/4\pi\eps_0 \). Так как рассматриваемая система обладает
сферической симметрией, то лапласиан запишем в сферических координатах:
\[
	\Delta = \frac{1}{r^2}\pder{}{r}\left(r^2\pder{}{r}\right) +
	\frac{1}{r^2\sin\theta}\pder{}{\theta}\left(\sin\theta\pder{}{\theta}\right) +
	\frac{1}{r^2}\ppder{}{\phi}.
\]
Подставив, получим
\[
	\left[\pder{}{r}\left(r^2\pder{}{r}\right) +
	\frac{1}{\sin\theta}\pder{}{\theta}\left(\sin\theta\pder{}{\theta}\right) +
	\ppder{}{\phi}\right]\psi =
	-\frac{2mr^2}{\hbar^2}\left(\frac{e^2}{r}+E\right)\psi,
\]
или
\[
	\pder{}{r}\left(r^2\pder{\psi}{r}\right) +
	\left[
	\frac{1}{\sin\theta}\pder{}{\theta}\left(\sin\theta\pder{}{\theta}\right) +
	\ppder{}{\phi}\right]\psi =
	-\frac{2mr^2}{\hbar^2}\left(\frac{e^2}{r}+E\right)\psi,
\]
Так как электрон движется в поле центральных сил, то система обладает
поворотной симметрией. С этим типом симметрии связан закон сохранения момента 
импульса. Так как момент импульса сохраняется, то он коммутирует
с гамильтонианом и энергия и момент импульса могут быть определены
одновременно. Поэтому \( \psi \) -- также собственная функция момента.
Тогда во втором слагаемом в левой части
\[
	\left[
	\frac{1}{\sin\theta}\pder{}{\theta}\left(\sin\theta\pder{}{\theta}\right) +
	\ppder{}{\phi}\right]\psi = -\frac{1}{\hbar^2}\hat{L}^2\psi = -l(l+1)\psi.
\]
Подставив это, получим
\[
	\pder{}{r}\left(r^2\pder{\psi}{r}\right) =
	\left[-\frac{2mr^2}{\hbar^2}\left(\frac{e^2}{r}+E\right) + l(l+1)\right]\psi
\]
Введём вспомогательную функцию \( \gamma = r\psi \). Тогда
\[
	\pder{\psi}{r} = \pder{}{r}\left(\frac{\gamma}{r}\right) =
	\frac{1}{r}\pder{\gamma}{r} - \frac{1}{r^2}\gamma,
\]
\[
	\pder{}{r}\left(r^2\pder{\psi}{r}\right) = \pder{}{r}\left(
	r\pder{\gamma}{r} - \gamma\right) = \pder{\gamma}{r} + r\ppder{\gamma}{r} -
	\pder{\gamma}{r} = r\ppder{\gamma}{r}.
\]
Подставим обратно и получим
\[
	\ppder{\gamma}{r} =
	-\frac{2m}{\hbar^2}
	\left(\frac{e^2}{r} - \frac{\hbar^2l(l+1)}{2mr^2} + E\right)\gamma.
\]
Теперь сделаем замены:
\[
	r = \frac{\hbar^2}{me^2}\rho,\quad E = \frac{me^4}{2\hbar^2}\eps,\quad
	f = \frac{me^2}{\hbar^2}\gamma
\]
и получим
\[
	\ppder{f}{r} = - \left(\eps + \frac{2}{\rho} -
	\frac{l(l+1)}{\rho^2}\right)f.
\]
Есть мнение, что решение стоит искать в виде
\[
	f = e^{-\alpha\rho}\sum_{n=0}^{\infty} a_n\rho^n.
\]
Подставим его в уравнение:
\[
	\left( \alpha^2 \sum_{n=0}^{\infty} a_n\rho^n -
	2\alpha\sum_{n=0}^{\infty} (n+1)a_{n+1}\rho^n +
	\sum_{n=0}^{\infty} (n+1)(n+2)a_{n+2}\rho^n \right)e^{-\alpha\rho}
	= - \left(\eps + \frac{2}{\rho} - \frac{l(l+1)}{\rho^2}\right)
	e^{-\alpha\rho}\sum_{n=0}^{\infty} a_n\rho^n,
\]
\[
	\alpha^2 \sum_{n=0}^{\infty} a_n\rho^n -
	2\alpha\sum_{n=0}^{\infty} (n+1)a_{n+1}\rho^n +
	\sum_{n=0}^{\infty} (n+1)(n+2)a_{n+2}\rho^n =
	\eps\sum_{n=0}^{\infty} a_n\rho^n +
	2\sum_{n=-1}^{\infty} a_{n+1}\rho^n -
	l(l+1)\sum_{n=-2}^{\infty} a_{n+2}\rho^n.
\]
Приводя подобные, получаем:
\begin{align*}
	n=-2:&\quad l(l+1)a_0 = 0,\\
	n=-1:&\quad 2a_0 - l(l+1)a_1 = 0,\\
	n\ge0:&\quad \alpha^2 a_n - 2\alpha(n+1)a_{n+1}+
	(n+1)(n+2)a_{n+2} = \eps a_n +
	2 a_{n+1} - l(l+1)a_{n+2}.
\end{align*}