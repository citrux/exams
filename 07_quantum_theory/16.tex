\documentclass{minimal}
\usepackage[utf8]{inputenc}
\usepackage[russian]{babel}
\usepackage[derivative]{hedmaths}
\begin{document}
%\question{Атом водорода}
Запишем уравнение Шредингера для электрона в кулоновском потенциале протона:
\[
    -\frac{\hbar^2}{2m}\Delta\psi - \frac{e^2}{r}\psi = E\psi,
\]
где \( e^2 = q_e^2/4\pi\eps_0 \). Так как рассматриваемая система обладает
сферической симметрией, то лапласиан запишем в сферических координатах:
\[
    \Delta = \frac{1}{r^2}\pder{}{r}\left(r^2\pder{}{r}\right) +
    \frac{1}{r^2\sin\theta}\pder{}{\theta}\left(\sin\theta\pder{}{\theta}\right) +
    \frac{1}{r^2\sin^2\theta}\ppder{}{\phi}.
\]
Подставив, получим
\[
    \left[\pder{}{r}\left(r^2\pder{}{r}\right) +
    \frac{1}{\sin\theta}\pder{}{\theta}\left(\sin\theta\pder{}{\theta}\right) +
    \frac{1}{\sin^2\theta}\ppder{}{\phi}\right]\psi =
    -\frac{2mr^2}{\hbar^2}\left(\frac{e^2}{r}+E\right)\psi,
\]
или
\[
    \pder{}{r}\left(r^2\pder{\psi}{r}\right) +
    \left[
    \frac{1}{\sin\theta}\pder{}{\theta}\left(\sin\theta\pder{}{\theta}\right) +
    \frac{1}{\sin^2\theta}\ppder{}{\phi}\right]\psi =
    -\frac{2mr^2}{\hbar^2}\left(\frac{e^2}{r}+E\right)\psi,
\]
Так как электрон движется в поле центральных сил, то система обладает
поворотной симметрией. С этим типом симметрии связан закон сохранения момента
импульса. Так как момент импульса сохраняется, то он коммутирует
с гамильтонианом и энергия и момент импульса могут быть определены
одновременно. Поэтому \( \psi \) -- также собственная функция момента.
Тогда во втором слагаемом в левой части
\[
    \left[
    \frac{1}{\sin\theta}\pder{}{\theta}\left(\sin\theta\pder{}{\theta}\right) +
    \frac{1}{\sin^2\theta}\ppder{}{\phi}\right]\psi =
    -\frac{1}{\hbar^2}\hat{L}^2\psi = -l(l+1)\psi.
\]
Подставив это, получим
\[
    \pder{}{r}\left(r^2\pder{\psi}{r}\right) =
    \left[-\frac{2mr^2}{\hbar^2}\left(\frac{e^2}{r}+E\right)+l(l+1)\right]\psi.
\]
Теперь сделаем замены:
\[
    r = \frac{\hbar^2}{me^2}\rho,\quad E = \frac{me^4}{2\hbar^2}\eps
\]
и получим
\[
    \pder{}{\rho}\left(\rho^2\pder{\psi}{\rho}\right) =
    \left[-\frac{2}{\rho}-\eps+l(l+1)\right]\rho^2\psi.
\]
Введём вспомогательную функцию \( \gamma = \rho\psi \). Тогда
\[
    \pder{\psi}{\rho} = \pder{}{\rho}\left(\frac{\gamma}{\rho}\right) =
    \frac{1}{\rho}\pder{\gamma}{\rho} - \frac{1}{\rho^2}\gamma,
\]
\[
    \pder{}{\rho}\left(\rho^2\pder{\psi}{\rho}\right) = \pder{}{\rho}\left(
    \rho\pder{\gamma}{\rho} - \gamma\right) = \pder{\gamma}{\rho} + \rho\ppder{\gamma}{\rho} -
    \pder{\gamma}{\rho} = \rho\ppder{\gamma}{\rho}.
\]
Подставим обратно и получим
\[
    \ppder{\gamma}{\rho} =
    \left[-\frac{2}{\rho}-\eps+l(l+1)\right]\gamma.
\]
Есть мнение, что решение стоит искать в виде
\[
    \gamma = e^{-\alpha\rho}\sum_{k=0}^{\infty} a_k\rho^k.
\]
Подставим его в уравнение:
\[
    \alpha^2 \sum_{k=0}^{\infty} a_k\rho^k -
    2\alpha\left(\sum_{k=0}^{\infty} a_k\rho^k\right)' +
    \left(\sum_{k=0}^{\infty} a_k\rho^k \right)''
    = - \left(\eps + \frac{2}{\rho} - \frac{l(l+1)}{\rho^2}\right)
    \sum_{k=0}^{\infty} a_k\rho^k,
\]
\[
    \alpha^2 \sum_{k=0}^{\infty} a_k\rho^k -
    2\alpha\sum_{k=0}^{\infty} (k+1)a_{k+1}\rho^k +
    \sum_{k=0}^{\infty} (k+1)(n+2)a_{k+2}\rho^k =
    -\eps\sum_{k=0}^{\infty} a_k\rho^k -
    2\sum_{k=-1}^{\infty} a_{k+1}\rho^k +
    l(l+1)\sum_{k=-2}^{\infty} a_{k+2}\rho^k.
\]
Приводя подобные, получаем:
\begin{align*}
    k=-2:&\quad la_0 = 0,\\
    k=-1:&\quad 2a_0 - l(l+1)a_1 = 0\text{ или } a_0 = 0,\ la_1 = 0 ,\\
    k\ge0:&\quad \alpha^2 a_k - 2\alpha(k+1)a_{k+1}+
    (k+1)(k+2)a_{k+2} = -\eps a_k - 2 a_{k+1} + l(l+1)a_{n+2}.
\end{align*}
На \( \alpha \) мы можем наложить ограничение. Для упрощени положим, что \( \alpha^2 = -\eps \). Тогда
\[
    a_{k+2} = \frac{2(\alpha(k+1) - 1)}{(k+1)(k+2) - l(l+1)} a_{k+1},
\]
или, уменьшив индексы на 1
\[
    a_{k+1} = \frac{2(\alpha k - 1)}{k(k+1) - l(l+1)} a_k,
\]
При \( k\gg l \) имеем
\[
    a_{k+1} \approx \frac{2\alpha}{k+1} a_k \approx \frac{(2\alpha)^{k+1}}{(k+1)!}
\]
Нетрудно видеть, что это коэффициенты разложения в ряд \( e^{2\alpha\rho} \). Произведение полученного ряда на экспоненту \( e^{-\alpha\rho} \) всё равно даст растущую экспоненту, что не соответствует реальности. Однако, если \( a = 1/n \), то
\[
	a_{n+1} = \frac{2(\alpha n - 1)}{k(k+1) - l(l+1)} a_n = 0,
\]
и все последующие коэффициенты равны нулю. Поэтому для получения волновых функций связанных состояний следует потребовать
\[
	\alpha = \frac{1}{n},\quad n\in\mathbb{N}.
\]
Отсюда получааем энергии и волновые функции стационарных состояний электрона в атоме водорода
\[
	\eps_n = -\frac{1}{n^2},\quad \psi = \frac{1}{\rho}e^{-\rho/n}\sum_{k=l+1}^{n} a_k\rho^k,\text{ где }a_{k} = \frac{2}{n}\frac{k-1-n}{k(k-1) - l(l+1)} a_{k-1},
\]
Если вернуться к энергии стационарного состояния \( E \), то выражение получится следующим
\[
	E = -\frac{me^4}{2\hbar^2}\frac{1}{n^2}.
\]
\end{document}
