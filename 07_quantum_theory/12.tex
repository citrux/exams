\question{Линейный гармонический осциллятор}
Потенциаьная энергия линейного гармонического осциллятора
\[
    U(x) = \frac{kx^2}{2} = \frac{m\omega^2x^2}{2}.
\]
Уравнение Шрёдингера имеет вид
\begin{equation}
    -\frac{\hbar^2}{2m}\dder{\psi}{x} + \frac{m\omega^2x^2}{2}\psi = E\psi
    \label{eq:12:schrodinger}
\end{equation}
Сделаем замену:
\begin{equation}
    \xi = \sqrt{\frac{m\omega}{\hbar}}x
    \label{eq:12:replacement}
\end{equation}
\[
    -\frac{\hbar^2}{2m}\frac{m\omega}{\hbar}\dder{\psi}{\xi} +
    \frac{m\omega^2}{2}\frac{\hbar}{m\omega}\xi^2\psi = E\psi,
\]
\begin{equation}
    \dder{\psi}{\xi} - \xi^2\psi + \lambda\psi = 0,
    \label{eq:12}
\end{equation}
где
\[
    \lambda = \frac{2E}{\hbar\omega}.
\]
Будем искать решение в виде
\begin{equation}
    \psi = e^{\beta \xi^2}\sum_{n=0}^{\infty} a_n\xi^n:
    \label{eq:12:solution}
\end{equation}
\[
    e^{\beta\xi^2}\left[ \left(\sum_{n=0}^{\infty} a_n\xi^n\right)''
    + 4\beta\xi\left(\sum_{n=0}^{\infty} a_n\xi^n\right)' +
    2\beta(1+2\beta\xi^2)\sum_{n=0}^{\infty} a_n\xi^n\right] -
    \xi^2e^{\beta \xi^2}\sum_{n=0}^{\infty} a_n\xi^n +
    \lambda e^{\beta \xi^2}\sum_{n=0}^{\infty} a_n\xi^n = 0,
\]
\[
    \sum_{n=0}^{\infty} (n+2)(n+1)a_{n+2}\xi^n
    + 4\beta\xi\sum_{n=0}^{\infty} (n+1)a_{n+1}\xi^n +
    (2\beta+\lambda+(4\beta^2-1)\xi^2)\sum_{n=0}^{\infty} a_n\xi^n = 0.
\]
\[
    \sum_{n=0}^{\infty} [(n+2)(n+1)a_{n+2} + (2\beta+\lambda) a_n] \xi^n +
    \sum_{n=1}^{\infty} 4\beta na_n\xi^n +
    (4\beta^2-1)\sum_{n=2}^{\infty} a_{n-2}\xi^n = 0.
\]
Отсюда,
\begin{align*}
    n=0: &\quad 2a_2 + (2\beta+\lambda)a_0 = 0, \\
    n=1: &\quad 6a_3 + (2\beta+\lambda)a_1 + 4\beta a_1 = 0,\\
    n\ge2: &\quad (n+2)(n+1)a_{n+2} + (2\beta+\lambda) a_n + 4\beta na_n +
    (4\beta^2-1)a_{n-2} = 0.
\end{align*}
\[
    a_2 = -\frac{2\beta+\lambda}{2}a_0,\quad
    a_3 = -\frac{6\beta+\lambda}{6}a_1,\quad
    a_{n+2} = -\frac{[(4n+2)\beta+\lambda] a_n + (4\beta^2-1)a_{n-2}}
    {(n+2)(n+1)}
\]
При больших \(n\) имеем
\[
    a_{n+2} \approx -\frac{\beta a_n}{n}
\]
следовательно ряд
\[
    \sum_{n=0}^{\infty} a_n\xi^n \sim e^{-\beta \xi^2}.
\]
Но отсюда \( \psi \) на бесконечности будет конечной величиной, что не
соответствует физической действительности. Поэтому следует оборвать ряд.
Для того, чтобы \( a_{n+1}\) и все последующие члены были равны нулю необходимо
\begin{gather*}
    \frac{[(4n+2)\beta+\lambda] a_n + (4\beta^2-1)a_{n-2}}{(n+2)(n+1)} = 
    -a_{n+2} = 0, \\
    \frac{[(4n+6)\beta+\lambda] a_{n+1} + (4\beta^2-1)a_{n-1}}{(n+3)(n+2)} = 
    -a_{n+3} = 0.
\end{gather*}
Отсюда,
\[
    4\beta^2-1 = 0 \text{ и } (4n+2)\beta + \lambda_n = 0. 
\]
Для заданной степени полинома \( n \) равенство возможно лишь при условии
\[
    \beta = -0.5,
\]
так как в противном случае экспнента приведёт к расходящемуся на бесконечности
решению. Следовательно,
\[
    \lambda_n = 2n+1
\]
и уравнение преобразуется к виду
\[
    P_n'' - 2\xi P_n' + 2nP_n = 0.
\]
Это уравнение полиномов Эрмита. Отсюда получаем собственные функции:
\[
    \psi_n = H_n(\xi)e^{-\frac{\xi^2}{2}},
\]
где \( \xi \) связано с \( x \) соотношением (\ref{eq:12:replacement}).
Собственные значения энергии определяются через \( \lambda \):
\[
    E = \frac{\hbar\omega}{2}\lambda = \hbar\omega\left(n + \frac{1}{2}\right).
\]

