\question{Несостоятельность классических теорий для микромира}

Классические механика и электродинамика при попытке применить их к объяснению
атомных явлений приводят к результатам, находящимся в резком противоречии с
опытом:
\begin{enumerate}
    \item согласно классической электродинамике атом неустойчив, так как
        движение электрона в ограниченной области является ускоренным, а
        ускоренно движущаяся заряженная частица излучает;
    \item также классическая электродинамика не может верно объяснить
        закономерности фотоэффекта и эффекта Комптона;
    \item с точки зрения классической механики частицы обладают точно
        определённой скоростью и движутся по точно определённым траекториям, что
        противоречит наблюдаемой дифракции электронов;
    \item классическая механика никак не может объяснить туннельный эффект --
        преодолевание частицей потенциального барьера.
\end{enumerate}
