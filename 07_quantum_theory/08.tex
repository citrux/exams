\question{Сложение моментов}

Рассмотрим систему, состоящую из двух слабо взаимодействующих частей. При полном
пренебрежении взаимодествием для каждой из них справедлив закон сохранения
момента импульса, а полный момент всей системы можно рассматривать как сумму
моментов её частей:
\[
    \hat{\vec{L}} = \hat{\vec{L}}_1 + \hat{\vec{L}}_2.
\]
Для проекций момента закон сложения очевиден:
\[
    \hat{L}_z = \hat{L}_{1z} + \hat{L}_{2z}.
\]
Следовательно для квантовых чисел выполняется соотношение
\[
    m = m_1 + m_2.
\]
Так как эти числа могут принимать следующие значения
\[
    m_1 \in \{-l_1, -l_1 + 1, \ldots l_1 \}, \quad
    m_2 \in \{-l_2, -l_2 + 1, \ldots l_2 \},
\]
то их сумма принимает значения из множества
\[
    \{ -l_1-l_2, -l_1-l_2+1, \ldots,  l_1+l_2 \}.
\]
Так как при целых значениях \( l_1 \) и \( l_2 \)  для любого значения
результирующего момента существует нулевая проекция на выделенное направление
\( m = 0 \), то число способов получить её определяет число возможных значений
числа \( l \), связанного с полным моментом. Пусть \( l_2 \le l_1 \), тогда
всего таких способов \( 2l_2+1 \). Следовательно, число \( l \) так же принимает
одно из \( 2l_2+1 \) значений, причем максимальное из них равно \( l_1 + l_2 \),
так как максимальное значение \( m \) равно \( l_1+l_2 \). Поэтому
\[
    l \in \{ l_1+l_2, l_1+l_2-1, \ldots, |l_1-l_2| \}. 
\]

